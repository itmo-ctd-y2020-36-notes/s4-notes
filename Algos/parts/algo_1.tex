\chapter{Паросочетания и потоки}

Мы начнём с паросочетаний и потоков.

\section{Паросочетания}
Паросочетание графа --- это набор его ребер, не имеющих общих вершин.

\subsection{Паросочетание в двудольном графе}
Паросочетания в двудольных графах ищутся гораздо проще, чем в произвольных.
Алгоритмы решают эту задачку понемногу. 
Берется маленькое (пустое) паросочетание и расширяется с помощью дополняющих цепочек.

Дополняющая цепочка --- это путь, который начинается и заканчивается в вершинах вне паросочетания, и ребра в нём чередуются (принадлежит - не принадлежит пар. сочетанию).
Из такой цепочки можно получить паросочетание большего размера (на 1).

Алгоритм начинает строить цепочки и удаляет дополняющие цепочки такой заменой.
\begin{theorem}
    В графе нет дополняющего пути тогда и только тогда, когда паросочетание максимально.
\end{theorem}
Это утверждение сформулировано для произвольных графов: для двудольных и не очень.
А в чём проблема? 
Проблема в поиске дополняющих путей.

В двудольном графе это делается просто. 
Это алгоритм Куна.

Двудольный граф можно хранить не как нормальный граф.
''Алгоритм КУНА. Это не связано с аниме никак!''
''Код, на самом деле, за пять копеек.''
Асимптотика: $O(nm)$.
% fn dfs(v):
% if mark[v]: return False;
% mark[v] = True

% for u \in G[v]:
%     if p[u] = \emptyset or dfs(p[u]):
%         p[u] = v
%         return True

% return False

% for v = 1..n
    % fill(mark, False)
    % dfs(v)

Алгоритм никогда не освобождает вершины --- если вершшина попала в пар. соч., она там останется, можно dfs из неё не запускать.
Чуть сложнее: если dfs однажда не нашел доп. пути из одной вершины, он никогда его не найдёт --- можно его не запускать.


Немного о полных сочетаниях.
\begin{theorem}
В двудольном графе $G_{n,n}$ существует полное паросочетание в том и только том случае, когда для любого подмножества $A$ вершин одной доли выполнено $|N(A)| \geqslant |A|$. 
\end{theorem}

Паросочетания позволяют решать кучу прикольных задач.

\subsection{Вершинное покрытие}
Вершинное покрытие в графе --- это набор вершин, таких, что никаие две вершины не лежат на одном ребре.
Это понятие двойственное понятию паросочетания.
Мы ищем минимальное по мощности вершинное покрытие. 
Это $NP$-полная задача в произвольном графе.
В двудольном, однако, всё очень мило.

Размер покрытия в графе всегда не превосходит размер вершинного покрытия: $M \leqslant S$.

