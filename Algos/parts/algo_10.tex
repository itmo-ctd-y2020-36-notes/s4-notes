\section{Тоерия чисел}

\subsection{GCD}
$gcd(a, b) = gcd(a \mod b, b)$


Алгоритм Евклида
\begin{lstlisting}[mathescape=true]
gcd(a, b):
    if b == 0:
        return a
    else:
        return gcd(b, a % b)
\end{lstlisting}

Мы будем считать алгоритмы полиномиальными, если они работают за полином от логаримфма от введенных чисел.

Время работы $\mathcal{O} (\log (a + b))$.

На самом деле, с помощью алгоритма Евклида можно решать дифантовые уравнения.

$ax + by = c$.
$d = gcd(a ,b )$.

$a, b:~~~ y a + (x - y k) b = d \longrightarrow b, a \% b:~~~ x b + y(a - kb) = d$.

\subsection{Китайская теорема об остатках}

Пусть $a \in [0..nm - 1]$, ~~~ $n, m$~--- взаимно простые, ~~~
$a_1 = a \% n$, ~~ $ b_2 = a\% m$\\
$a \leftrightarrow (a_1, a_2)$

\subsection{Вычисления по модулю}

Хотим замкнуть все свои вычисления в кольце остатков по модулю $m$.

$(a + b) \% n$,~~~ $a + (-a) = 0$.

Складывать, вычитать, умножать --- легко. Делить тоже можно.

$a \perp n$~~~~~ $a \cdot a^{-1} = 1 (\mathrm{mod} n) \iff a * x = 1 + n * y$.

\subsection{Простые числа}

Умеем легко проверять простоту за $\mathcal{ O} (\sqrt {n})$.
