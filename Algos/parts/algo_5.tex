Рассмотрим такие графы, в которых все $c_{uv} = 1$

Можем находить все непересекающиеся пути между двумя вершинами. Если $c_{uv}$ маленькие, например 1 или 2, и в нём нет параллельных потокв, можно разрешить параллельные рёбра.

Алгоритмы:
\begin{table}[htpb]
    \centering
    \caption{Асимптотики}
    \label{tab:label}
    \begin{tabular}{cccc}
        & $c_{uv}\in \R$ & $c_{uv} = 1$ & $c_{uv}=1$ и нет параллельных\\
        ФФ & $O(Fm)$ & $O(m^2)$ & $O(nm)$\\
        ЭК & $O(m^2n)$ & $O(m^2)$ & $O(nm)$\\
        Диниц & $O(mn^2)$ & $O(m^{\frac{3}{2}})$ & $O(\min\{\sqrt{m}, n^{\frac{2}{3}} \}m)$
    \end{tabular}
\end{table}

\begin{statement}
    Диниц не делаеть $n$ фаз. $c_{uv}=1 \implies $ Фаз $\leqslant \sqrt{m}$

    Нашли пути длин $1, 2, 3, \ldots, \sqrt{m} $

    $F_{max} - F$ разбивается на непересекающиеся пути $s \to t$

    Так как всего рёбер $m$ И каждое ребро участвует только в одном пути, то количество таких путей будет $\leqslant \sqrt{m} $
\end{statement}

\begin{statement}
    $c_{uv}$ и нет параллельных рёбер, то $O(n^{\frac{2}{3}})$
\end{statement}

\subsection{Вспомним паросочетания}

Был двудольный граф. Мы в нём искали паросоетания, потом содили его к потоку.

Если просто натравить Диница, то уже получится лучше $O(nm)$. Давайте ещё лучше оценим.

\begin{statement}
    $O(\sqrt{m} )$ фаз для диница и паросочетаний.

    $F_{\max} - F$ можно декомпозировать на пути.
\end{statement}

Финальная асимптотика нахождения паросочетания с помощью Диница -- $O(\sqrt{n}m)$

\subsection{Push-relabel}

\begin{definition}
    Предпоток:

    $f_{uv} \leqslant c_{uv}$, но баланс $b_u = \sum_v f_{uv} \geqslant 0$

    Слои: $l[v] $ -- уровень вершины.

    $l[s] = n\quad l[t] = 0$ Это константно. $l[v] = 0$ дл явсех остальных вершин, потом растёт.
\end{definition}

\begin{property}
    Если ребро очень сильно уходит вниз. $l[u] \leqslant l[v] -2 \implies f_{uv} = c_{uv}$

    Все рёбра, которые идут сильно вниз, они насыщены.
\end{property}

Изначально в алгоритме свйоство не выполняется для рёбер из $s$.Насытим все эти рёбра

Две операции:
\begin{itemize}
    \item push. $l[u] = l[v] - 1\quad f_{uv} += \min(b_v, \p c_{uv})$
    \item relabel. $\p c_{uv} > 0\quad l[v] = \min(l[u]) + 1$
\end{itemize}

\begin{enumerate}
    \item Все вершины с балансом $b_v = 0 \implies F_{\max}$
    \item На любом пути от $s$ до $t$ бужет хотя бы одно насыщенное ребро. Значит насыщенные рёбра образуют разрез. Значит поток максимален.
\end{enumerate}

Давайте теперь время на всё это.

$O(n)$ на один realabel. $O(nm)$ на все.

Насыщающее проталкивание: Каждый раз поднимает уровень. Таких не больше $n$ для каждого ребра. Все вместе они тоже дают $O(nm)$

Остались ненасыщающие, их может быть много, с ними нужно аккуратно.

Заведём баланс как сумму уровней вершин с ненулевым балансом. $\Phi = \sum_{b_n >0} l(u)$

relabel: $n^2 \cdot n$. Насыщающий push $nm\cdot n$. Ненасыщающий $-1$. Последних $\leqslant n^2m$ соответственно.
насыщающий push

Если взять хорошу очередь и в правильном порядку релаксировать, то будет $O(n^3)$ вместо $O(n^2m)$

Ещё можно $O(n \sqrt{m} )$

relabel не поднимает потенциал больше, чем на $n$, а проталкивание бесплатное.
% последняя строчка, которую можно двигать