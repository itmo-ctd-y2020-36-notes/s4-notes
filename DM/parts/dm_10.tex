\subsection{А обязательно ли разрешать компьютеру зависать?}

\begin{definition}
    Язык программирования называется \textbf{полным} для $A$, если для любого перечислимого языка из $A$ можно задать его описание на этом языке.

    Иными словами, любую вычислимую функцию можно задать при помощи этого языка программирования.
\end{definition}

\begin{definition}
    Язык называется \textbf{вычислимым} (компилируемым), если по описанию и словы мы можем сказать подходит или нет.
\end{definition}

\begin{definition}
    Язык программирования \textbf{независающий}, если $\forall$ программы и $\forall$ слова он не зависнет.
\end{definition}

\begin{theorem}
    Не существует полного для разрешимых языков вычислимого, не зависающего метаязыка описания.


    То есть, не существует вычилимой нумерации всех всюду определенных вычислимых функций.
\end{theorem}

\begin{proof}
    Предположим, что она (нумирация) существует. Вот эта нумерация $f_1, f_2, f_3, \dots$.
    \begin{tabular}{c|c c c c}
         & $x_1$ & $x_2$ & $x_3$ & $x_4$  \\\hline
        $f_1$ &&&&\\
        $f_2$ &&&&\\
        $f_3$ &&&&\\
        $f_4$ &&&&
    \end{tabular}

    $g(i) = f_i(x_i) + 1$, ~~~ $g$~--- всюдуопределенная, вычислимая.\\
    $g(i)  = ! f_i(x_i)$
\end{proof}

\begin{corollary}
    На конечном автомате нельзя интерпретировать конечный автомат.
\end{corollary}


\subsection{Разрешимость}
У нас есть пример неразрешимого языка. Это язык $U = \{ \langle p, x \rangle \mid p(x) = 1\}$.
А если мы хоти проверить еще какой-то язык? Можно ли куда-то замести под ковер рассужения?

\begin{definition}
    $m$~--- сведение (исторически many to one reduction, но в реальности это оказалось не удачным, поэтому называют mappng сведение).

    Говорят язык $A \leqslant_m B$, если существует всюдуопределенная вычислимая функция $f$, такая, что $x \in A \iff f(x) \in B$.
\end{definition}

\begin{theorem}
    Если $A$ не разрешимый, $A \leqslant_m B \implies B$~--- не разрешимый.
\end{theorem}

\begin{proof}
    Предположим $B$ разрешимый, то есть есть $\mathrm{in} B(x)$.
    Тогда $A$ разрешается программой
\begin{lstlisting}[mathescape=true]
inA(x):
    return inB(f(x))
\end{lstlisting}
\end{proof}

\begin{example}
    [Задача останова]

    Пусть $HALT = \{ p\mid p(\epsilon) \text{ не зависает } \}$.

    Сведем $U$ к $HALT$.
\begin{lstlisting}[mathescape=true]
f ($\langle p, x \rangle$):
    return "q(y):
        def p = ...
        if p(x) != 1:
            while True:
                pass
    "
\end{lstlisting}

Заметим, что $f$ всюду определнная и вычислимая.

$f$ является $m$ сведением U к HALT.

$q \in HALT \iff \langle p, x \rangle \in U$.
\end{example}

Разминка перед Т. У-Р.
Пусть мы хотим проанализировать поведение программы.

$A = \{ p \mid p \text{ чему-то удовлетворяет, но не зависает}\}$.
Тогда $A$ не зазрешим.

\begin{proof}
    Также напишим
\begin{lstlisting}[mathescape=true]
f($\langle p, x\rangle$):
    return "q(y):
        if p(x) = 1:
            сделай то, что удовлетворяет $A$
        else: while True: pass
    "
\end{lstlisting}
\end{proof}

\subsection{Теорема Успенского-Райса}
$E = $ множества перечислимых языков.

c: $\Sigma \equiv $ char,~ s: $\Sigma^* \equiv $ string,~
L $\subset \Sigma^* \equiv $ set<string> = lang,~
$2^{\Sigma^*} \equiv $ set<lang>,~
$E \subset 2^{\Sigma^*} \equiv $ set<lang>.

Свойство перечислимых языков:\\
$A \subset E \equiv $ set<lang>\\
$L \in A \to L $ удовлетворяет $A$\\
$L \not\in A \to L$ не удовлетворяет $A$.\\
$L(A) = \{ p \mid L(p) \in A\}$\\
$L(p) = \{ x \mid p(x) =1 \}$.

\begin{theorem}[Успенского-Райса]
    Язык любого нетривиального свойства перечислимых языков не разрешим.

    $A = \O~~~~L(A) = \O$, ~~~~~ $A = E ~~~~ L(A) = \Sigma^{*}$.
    Есть $A\neq \O$, $A \neq E \implies L(A) $ не разрешим.
\end{theorem}

\begin{proof}
    Пусть $\O \not\in A$.
    Пусть какой-то язык $X \in A$.

    $L(A)$~--- разрешим $in A(p)$. Напишем следующий код, разрешающий $U$.
\begin{lstlisting}[mathescape=true]
in U($\langle p, x \rangle$):
    q = "q(y):
            if p(x) = 1:
                return inX(y)
            else:
                return 0
        "
    return inA(q)
\end{lstlisting}
Утверждается, что получился разрешитель для $U$.
Если $p(x) = 1$, то $L(q) = X \in A$.
А если $p(x) = 0$ или $p(x)$ зависает, то $L(q) \O$.

Таким образом, $\langle p, x \rangle \in U \iff q \in L(A)$.

То есть $f$ $m$--сводит $U$ к $L(A)$.
\end{proof}
