\subsection{Теорема о рекурсии}

Бывают программы, что
\begin{lstlisting}[mathescape=true]
q - разрещитель
    f(x): $\dots$
    if q(f):
        while True: pass
    else:
        return 1
\end{lstlisting}

\begin{example}[игрушечный] Что выведет эта инструкция?
\begin{verbatim}
    Написать 2 раза, второй раз в ковычках:
    "Написать 2 раза, второй раз в ковычках:"
\end{verbatim}

Действительно, получится:
\begin{verbatim}
    Написать 2 раза, второй раз в ковычках:
    "Написать 2 раза, второй раз в ковычках:"
\end{verbatim}
\end{example}

То есть в каком-то месте программы мы можем вывести исходный код программы.

\begin{lstlisting}[mathescape=true]
f(...):
.
.   getSource()
.
.
\end{lstlisting}
Программа выведет:
\begin{verbatim}
.
getSource():
.
.
.
.
\end{verbatim}

Наиболее интересное применение --- quine~--- программа, которая выводит свой исходный код.

Как же это делать?
\begin{lstlisting}[mathescape=true]
f(...):
.
.
.   getSource()
.

getSource():
    s $\longleftarrow$ getAuxSource()
    return s + "getAuxSource():" +
    "   return \"" + s + "\""

getAuxSource():
    return "f(...):
            ...
            getSource():
                s $\longleftarrow$ getAuxSource()
                return s + \"getAuxSource():\" +
                    \"   return \\\"\" + s + \"\\\"\"
            "
\end{lstlisting}
\begin{note}
Заметим, что весь код программы состоит из того, что выведет getAuxSource, плюс определение getSource.
\end{note}


\begin{theorem}[О рекурсии]
    Пусть $V(x, y)$~-- вычислимая функция $\implies \exists $ вычислимая  функция $f$ :  $f(y) =  V (f, y)$.
\end{theorem}

Благодаря этой теореме можно проще доказывать неразрешимость некоторых языков.

\begin{example}
    $\mathrm{HALT} = \{ p \mid p(\varepsilon) \neq \bot \}$ неразрешим

    \begin{proof}
        Пусть $h$~--- разрещитель $\mathrm{HALT}$.

\begin{lstlisting}[mathescape=true]
p():
    if h(p) == 1 :
        while True: pass
    else:
        return 1
\end{lstlisting}
    \end{proof}
\end{example}


\begin{example}[Теорема Успенского--Райса]
    $A \subseteq RE, ~~~A \neq \O,~~~ A \neq RE$.

    $L(A )  = \{ p \mid L(p) \in A \}$~--- неразрешим.

    \begin{proof}
        Пусть $inL_A$~--- разрешитель.

Пусть $\underbrace{M}_{inM(x)}\in A,~~~ \underbrace{N}_{inN(x)} \not\in A$.

\begin{lstlisting}[mathescape=true]
f(x):
    if ($\mathrm{inL}_A$(getSource())):
        return inN(x)
    else:
        return inM(x)

\end{lstlisting}
    \end{proof}
\begin{note}
    Такая подростковая программа: послушаем родителей и сделаем наоборот.
\end{note}
\end{example}

\subsubsection*{Еще примеры невычислимых функций}

\begin{example}
    $K(x)$~--- Колмогоровская сложность\\
    $K(x) = \min$ длинна программы $p$, что $p() = x$.

    К сожалению, нельзя написать программу, которая бы оптимально кодировалоа строку.
\end{example}

\begin{statement}
    $K(x)$~--- невычислима
\end{statement}
\begin{proof}
    Пусть $K(x)$ вычислима, напишем код.
\begin{lstlisting}[mathescape=true]
p():
    for x \in $\Sigma^{*}$:
        if k(x) > |getSource()|:
            return x
\end{lstlisting}
Мы нашли программу, у котороой колмогоровская сложность больше, чем сложность программы $p$, которая без проблем вывродит эту строчку.
\end{proof}

\begin{example}
    [Busy Beaver (BB(n))]
    Эта функция принимает число n и возвращает максимальное число шагов, которое делает программы длинны $n$ (из $n$ строчек или символов).
\end{example}

\begin{statement}
    $BB(n)$ невычилима. Напишем программу, которая будет работать дольше.
\begin{lstlisting}[mathescape=true]
f():
    for i = 0 .. BB(|getSource()|):
        pass
    return
\end{lstlisting}
\end{statement}
