\subsection{Проблемы вычислимости}
Когда мы говорим про вычислимость, мы представляем программу на \textit{c++}, \textit{python}, \textit{псевдоязыке}. Но с математикикой это не очень удобно связывать.

\begin{problem}[Проблема Соответствия Поста]
    Есть число $n$. Далее следует $n$ строк с числами $a_1, b_1, a_2, b_2, \dots, a_n, b_n$.

    Задача, вывести \texttt{YES} или \texttt{NO} и сертификат, если существует последовательность индесов такая, что $a_{i_1}, a_{i_2},a_{i_3}, \dots a_{i_n} = b_{j_1}, b_{j_2}, b_{j_3}, \dots, b_{j_n}$.

\end{problem}

\begin{problem}[Проблема полимомно]

Полимино --- плоская геометрическая фигура, состоящая из $n$ одноклеточных квадратов, соединенных по сторонам.

Дано $n$, $w, h$ --- количество  типов полимино, высота и ширина.
Далее $n$ фигурок полимино.

Требуется понять, можно ли замостить четверть плоскости такими фигурами.

\end{problem}

\begin{definition}

[Машиана тьюринга].
$\Sigma$~--- входной алфавит, $\Pi \supset \Sigma$.
$B  \in \Pi \setminus \Sigma$~--- пробел.

$Q$~--- множество состояний.
$Y\in Q$, $N \in Q, Y\neq N$.
$S \in Q$~--- стартовое состояние.

$\delta: \left( \setminus \{ Y, N\} \right) \times \Pi \to Q \times P \times \{ \leftarrow, \rightarrow\}$.

Мгновенное описание Машины Тьюринга $\alpha \#_p \beta$.

$\vdash$~--- переходит за один шаг.
$\alpha \#_p \beta \vdash \xi \#_q \eta$, если \\
$\begin{cases}
    \alpha = \xi \alpha, \eta = a d \beta, ~~~ \delta(p, c) = \langle q, d, \leftarrow\rangle,\\
    \xi = \alpha d, \eta = \p \beta, ~~~\delta(q, d, \rightarrow),\\
    \xi = \alpha, \eta = d \p \beta, ~~~\delta(q, d, \cdot)
\end{cases}$
\end{definition}

Сейчас мы попробуем приблизить машину тьюринга к современному компьютору.\\
М.Т. $\to$ многодорожечная  М.Т. $\to$ многоленточная М.Т $\underset{\text{яма с крокодилами и Тезис Т.Ч.}}{\longrightarrow}$. современный компьютор.

А потом мы попробуем упростить нашу машину тьюринга.\\
Двух счетчиковая машина $\leftarrow$ трех счетчиковая машина $\leftarrow$ двустековая машина $\leftarrow$ М.Т.

А также мы докажем эфкивавалентность М.Т.\begin{enumerate}
    \item Машине Мароква,
    \item Грамматикам нулевого порядка,
    \item Кл. Автомат.
\end{enumerate}

\begin{definition}
    Многодорожечная Машина Тьюринга

    Пусть у нас есть не одна дорожка, а несколько, то есть мы можем не портить исходное слово.
\end{definition}

\begin{statement}
    М.Д.М.Т. $\approx$ М.Т.

\end{statement}

\begin{proof}
    Действительно, $\p \Pi = \Pi^k$.
\end{proof}

\begin{statement}
    Можно считать, что лента односторонне бесконечная и нет пробелов $B$.
\end{statement}
\begin{proof}
    Чтобы не ставить пробелы добавим сурогатный пробел $\{ \ov B\} \cup \Pi = \p \Pi$.

    Потом отрежем один бесконечный край ленты, развернем и приклеим к началу со сдвигом на один. Получим ленту с двумя бесконечными в одну сторону дорожками.

    При этом вычислительную мощность мы не потеряли.

\end{proof}
\begin{definition}
    Многоленточная Машина Тьюринга

    Пусть у нас есть $k$ лент и на каждой из них своя головка. То есть
    $\delta: Q\time \Pi^k \to Q \times \Pi^k \times \{\leftarrow, \rightarrow, \cdot\}^k$.
\end{definition}


\begin{statement}
    М.Л.М.Т. $\approx$ М.Т.
\end{statement}
\begin{proof}
    Докажем эквивалентность М.Л.М.Т. $2k$--дорожечной МТ.
    На нечетных дорожках будут данные, на четных дорожках будем хранить положение головки в $ i  / 2$--й ленте.

    Переход будем делать по стадиям пока не подвиним все головки.

    $\p Q = Q \times \Pi ^k \times 2^k \cup$ (специальные состояния для обратного прохода + служебные).
\end{proof}

\begin{note}
    Пусть М.Л.М.Т сделала $t$ шагов для симуляции 1 шага. Требуется $\mathcal{O} (t)$ шагов 1 л М.Т.

    А для всех $t$ шагов потребуется $\mathcal{O}(t^2)$.
\end{note}

\begin{theorem}[Тьюринга-Черча]
    $L$~--- полуразрешимый $\iff$ $\exists$ М.Т. $L(m) = L$.

    Усилинная версия. Замедление происходит не более, чем в полином раз.
\end{theorem}

Теперь давай в обратную сторону.

Пусть у нас есть недетерменированная М.Т.
$\delta :Q \times \Pi \to P_{< \infty} \left( Q \times \Pi \times \{ \leftarrow, \rightarrow, \cdot \} \right)$.

\begin{definition}
    $k$--стековая машина
    $\delta:Q\times(\Sigma \cup \{ \epsilon\})\times \Pi^k\to P_{<+\infty} \left( Q \times (\Pi^*)^k \right) $
\end{definition}

\begin{theorem}
    $L$~--- перечислим $\iff$ разрешим на 2--стековой машине.
\end{theorem}

\begin{definition}
    $k$--счетчиковая машина:\\
    $\delta: Q\times (\Sigma \cup \{ \epsilon\}) \times \{=0, > 0\}^k \to P_{<+\infty} \left( Q \times \{ +1, -1, +0\}^k \right)$.
\end{definition}

\begin{theorem}
    $k$--стековая машина $\implies$ $k + 1$--счетчиковая машина
\end{theorem}
\begin{theorem}
    $k$--счетчиковая машина $\implies$ 2--счетчиковая машина.
\end{theorem}

Другие вычислители, эквивалентные Машине Тьюринга.

Вообще, приимущество М.Т. в ее локальности. То есть $p\vdash q$, растояние Левенштейна между $p$ и $q$ равно $\mathcal{O} (1)$.

\begin{definition}
    Нормальные алгорифмы Мароква (Машина Мароква).

    У нас есть строковый регистр (строка данная на вход).
    Программа для Машины Маркова: переходы $s_1 \to t_1, s_2 \to t_2, \dots, s_k \to \mathbb{STOP}, \dots$.
    Просматриваем список правил, берем первое встретившееся правило, где $s_i = s$ и заменяем $s$ на $t_i$.
\end{definition}

\begin{theorem} [Маркова -- Тьюринга]
    М.М. $\equiv$ М.Т.
\end{theorem}
\begin{proof}
    В одну сторону $\pm$ понятно. А как на Машине Маркова сделать М.Т.?

    Добавим последнее правило $\epsilon \to \#_s$.

    $\delta (p, c) = (q, d , \rightarrow)$ переделаем в $\#_p c \to d\#_q$.

    $\beta (p, c) = (q, d, \leftarrow)$ переделаем в набор правил для всех $a$ $a\#_p c \to \#_q a d$.

    $\#_Y \to Y$, $\#_N \to N$
    $\#_{\mathbb{STOP}} \to \mathbb{STOP}$.

\end{proof}
