\subsubsection{Вычислимые свойства множеств}
Свойства языков:

по теореме Успенского--Райска, если $A$~--- нетривиальное свойство, то язык $A$~--- не разрешим

Какой свойство программ является перечислимым?

\begin{itemize}
    \item Например, множество программ, таких, что они принимают пустой вход.
\[ A = \{  L | \varepsilon \in  L\},~~~~ L(A) = \{ p\mid L(p) \in A \} = \{ p \mid \varepsilon \in L(p) \} = \{ p \mid p(\varepsilon) = 1\} \]
\end{itemize}


Какие еще свойства будут перечислимыми?
\begin{itemize}
    \item $A = \{ L \mid x_1 \in L \vee x_2 \in L \vee \dots \vee x_k \in L \}$,
    \item $A = \{ L \mid 0^n \in L $ для какого-то $n\}$
    \item $A = \{ L \bigvee_{j}\bigwedge_{i=1}^{k_j} X_{ji} \in L  \}$
\end{itemize}


\begin{lemma}
    $A$~--- перечислимое свойство, $L \in A$, $L \subset M$, то $M \in A$.
\end{lemma}

\begin{proof}
    Пусть $A$~--- перечислимое свойство, $L \subset M, L \in A, M \neq A\}$.
    $K$~--- перечислимое, но не разрешимое множество.

    Рассмотрим такую функцию:
    $V(n, x) = \begin{cases}
         1, &x \in M, n \in K,\\
         1, &x \in L, n \not\in K,\\
         \bot, &\text{иначе}
    \end{cases}$

    Утверждается, что $V(n, x)$~--- вычислима.

    Запустим в трех тредах:
    \begin{enumerate}
        \item $n \in K$?
        \item $x \in L$?~~~~~~ $1 \to $ return $1$, значит и остальные треды вернут $1$.
        \item $x \in M$?
    \end{enumerate}

    Получим полуразрешитель $\overline{K}$.

\begin{lstlisting}[mathescape=true]
def check$\overline{K}$(n):
    p = p(x):
        return V(n, x)
    return p
\end{lstlisting}

$n \in K \implies p$ возвращает 1, если $x \in M \implies L(p) \not\in A$.

$n \not\in K \implies p$ распознает $L \implies L(p ) \in A$

Получили два полуразешителя $\implies$ есть разрешитель, что противоречит предположению.
\end{proof}

\begin{lemma}
    $A$~--- преечислимое свойство, $L in A \implies \exists M \subset L $, $M$~--- конечный, $M \in A$.
\end{lemma}

\begin{proof}
    Пусть $A$~--- перечислимое свойство языка, $L \in A, \forall$  конченого $M \subset L, M\neq A$.

    Пусть $K$~--- перечслимое не разрешимое.

    Определим $V(n, x) = \begin{cases}
        1, &x \in L, \text{ за } x \text{ шагов перечисения } K  \text{ число } n  \text{ не появилось }\\
        \bot,& \text{ иначе}.
    \end{cases}$.


\textit{Очевидно}, $V$ рашрешима.

Плучим полуразрешитель для $\overline{K}$.

Аналогично доказательству леммы 1 приходим к противоречию.

\end{proof}

\begin{theorem}[Райса--Шапира]
    $A$~--- перечислим $\iff \, \exists$ перечислимое множество образцов, то есть $\forall L \in A \iff \, \exists$ образец, который уодовлетворяет $L$.
\end{theorem}

\begin{proof}[Доказательство теоремы]
    $\Leftarrow $~--- Очевидно

    $\Rightarrow $~--- A перечислимо, тогда рассмотрим конечные языки в $A$, напишем разрешитель для некоторого конечного вектора $[X_1 \dots X_k]$.
\end{proof}

Литература для самостоятельног изучения:
\begin{itemize}
    \item Если больше интересны абстрактные вычествители:\\
    Хопкрофт, Джон Э., Мотвани Раджив --- Введение в теорию автоматов, языков и вычислений.
    \item Если больше интересна математическая составляющая:\\
     Н. К. Верещагин, А.Шень --- Вычислимые функции
     \url{https://www.mccme.ru/free-books/shen/shen-logic-part3-2.pdf}
\end{itemize}
