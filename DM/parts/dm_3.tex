\begin{theorem}
    Для производящих функций, задающих рекурентные соотношения эквивалентны следующие высказывания
    
    \begin{itemize}
        \item $a_n = c_1 a_{n-1} + c_2 a_{n-2} + \dots + $
        \item $A(t) = \dfrac{P(t)}{Q(t)}$
        \item $a_n = \sum_{i = 1}^{b} p_i(n) \cdot r^i$, где $r_i \in \mathbb{C}$ 
    \end{itemize}
\end{theorem}

$Q(t) = 1 - c_1 t - c_2 t^2 - \dots - c_k t^k$

$P(t)$ определяет то, как надо подправить первые члены, чтобы получились те, которые нужны.

$A(t) Q(t)- P(t) = 0$. 

Как посчитать $r$?

Пусь $Q(t) = 1 - rt$,\\
$a_n = r \cdot a_{n - 1}$\\
$a_m = r \cdot a_{m-1}$\\
$a_{m + 1} = r \cdot a_{m}$\\
\ldots\\
$a_n = r^n \cdot \dfrac{a_{m-1}}{r^{m-1}}$

Пусть
$Q(t) = (1 - r_1 t) (1 - r_2 t)$, ~$r_1 \neq r_2$. 

\begin{lemma}
$Q( t ) = \prod_{i=1}^n (1 - r_i t)$, где $r_i \neq r_j$
$\dfrac{P(t)}{Q(t)} = \sum_{i= 1}^n \dfrac{P_i(t)}{1-r_it}$ 
\end{lemma}

$Q(t) = \sum_{i=1}^n d_i r_i^n$.

$r_i$ --- числа, обратные корням многочлена $Q$.
Если степень $Q$ равно $k$, то $Q$ имеет ровно $k$ корней (с учетом кратности).

Таким образом, $Q(t) = q_k \prod_{i=1}^k (t - t_i)  = 
(-1)^k q_k \prod_{i=1}^k \left(1 - \dfrac{t}{t_i} \right) \cdot t_i =$ \\ 
$= \left[ (-1)^k q_k \cdot \prod_{i=1}^k t_k \right] \prod_{i=1}^k (1- r_i t) =\alpha \prod_{i=1}^k (1- R_i t)$.

Почему нет корня 0? Потому что $Q(t)$ имеет вид $Q(t) = 1 - c_1 t - \dots$.

\begin{example}
    Рассмотрим числа Фибоначчии. $F(t) = \dfrac{1}{1 - t -t^2}$.
    

    Корни $t_{1,2} = \dfrac{-1\pm\sqrt{1 + 4}}{2}$, обратные корини $r_{1, 2} = \dfrac{1\mp \sqrt{5}}{2}$.
    Обратные корни разные~--- нам очень приятно.

    \[Q(t) = \left( 1 - \dfrac{1-\sqrt{5}}{2}\right) \left( 1 - \dfrac{1+\sqrt{5}}{2}\right).
    \]\[
        \dfrac{1}{(1 - r_1 t ) (1 - r_2 t)} = \dfrac{c_1}{1 - r_1 t} + \dfrac{c_2}{1 - r_2 t},~~~
        c_1(1 - r_2 t) + c_2(1 - r_1 t) = 1 \implies
    \]\[
        \begin{cases}
            c_1 + c_2 = 1\\
            c_1(-r_2) + c_2(-r_1) = 0
        \end{cases} \implies
        c_2 = \dfrac{-r_2}{r_1 - r_2} = \dfrac{-1-\sqrt{5}}{2 \cdot(-\sqrt{5})} 
        = \dfrac{5 + \sqrt{5}}{10},~~~
        c_1 = \dfrac{5 - \sqrt{5}}{10}.
    \]\[
        a_n = c_1 r_1^n = \dfrac{5-\sqrt{5}}{10} \cdot \left( \dfrac{1-\sqrt{5}}{2}\right)^n\\
        b_n = c_2 r_2^n = \dfrac{5+\sqrt{5}}{10} \cdot \left( \dfrac{1+\sqrt{5}}{2}\right)^n \implies
    \]\[
       f_n = \dfrac{5 - \sqrt{5}}{10} \cdot \left(\dfrac{1-\sqrt{5}}{2}\right)^n +
       \dfrac{5 + \sqrt{5}}{10} \cdot \left(\dfrac{1+\sqrt{5}}{2}\right)^n =
       \Theta \left( \left( \dfrac{1 + \sqrt{5}}{2}\right)^n \right).
    \]
\end{example}

\begin{remark}
    Если $\lambda$~--- единсвенный минимальный по модулю комплексный корень $Q(t)$,
    $A(t) = \dfrac{P(t)}{Q(t)}$, то $a_n = \Theta \left( \dfrac{1}{\lambda^n} \right)$.
\end{remark}

$\dfrac{1}{(1 - rt)^2} = \dfrac{1}{1 - 2rt +r^2 t^2}$. $a_0 = 1, ~a_1 = 2r$,~
$a_2 = 3 r^2$, $a_3 = 4r^3$, ~ \ldots, $a_n = (n + 1) r^n$.

$\dfrac{1}{r} \left( r^n t^n \right)^\prime = 
\dfrac{1}{r} \sum n r^n t^{n - 1} = \sum n r^{n-1} t^{n-1} = \sum (n + 1) r^n t^n$.

\begin{lemma}
    $\frac{1}{1-rt}^s = \sum_{n=0}^\infty p_s(n)r^nt^n$
\end{lemma}
\begin{proof}
    Докажем по индукции. 
    \begin{enumerate}
        \item База. $s = 0$ --- просто
        \item Переход. Далее много формул.
        $\left( \frac{1}{(1-rt)^s} \right)^\prime = \frac{-r(-s)}{(1-rt)^{s+1}}$.
        $\left( \sum_{n=0}^\infty p_s(n)r^nt^n \right)^\prime = \sum_{n=1}^\infty np_s(n)r^nt^{n-1} = 
        \sum_{n=0}^\infty (n+1)p_s(n+1)r^{n+1}t^n$.
        $\frac{1}{(1-rt)^{s+1}} = \sum_{n=0}^\infty \frac{n+1}{s}p_s(n+1)r^nt^n$,
        $p_{s+1}(n) = p_s(n+1)\frac{n+1}{s} = \sum_{i=0}^{s-1} p_{s,i} (n+1)^i \frac{n+1}{s}$.
        \[ p_{s,i} = \frac{a_{s,i}}{b},\quad b=s!,\ a_{s,i}\in \mathbb{Z} \] 
    \end{enumerate}
\end{proof}

\begin{theorem}
    Пусть $A(t) = \dfrac{P(t)}{Q(t)}$, ~$r_i$~--- обратный корень кратности $s_i$ $Q_i$, ~количество различных корней $b$.
    Тогда начиная с некоторого места (но точно, начиная с $k$) $a_n = \sum_{i = 1}^b p_i(n) r_i^n$, ~~~$\deg p_i = s_i - 1$,~~~
    $\sum_{i=1}^b s_i = k$.  
\end{theorem}

\begin{proof}
    \[ Q(t) = \prod_{i=1}^b \left( 1 - r_i t \right) ^{s_i},~~~
    \dfrac{P(t)}{Q(t)} = \sum_{i = 1} ^b \dfrac{P_i(t)}{(1 - r _it  ) ^{s_i}}.
    \]
\end{proof}

Если $\lambda_i$~--- единственный минимальный комплексный корень $Q(t)$ кратности $s_i$.
Тогда $a_n = \Theta \left( \dfrac{n^{s_i - 1}}{\lambda_i^n} \right)$.

\begin{example}
    $a_n = n^3$,~~ $a_n = 4 \cdot a_{n-1} - 6 \cdot a_{n -2} +
     4 \cdot a_{n - 3} - a_{n-4}$.
     
     Подберем поправку первых членов: $P(n) = t + 4t^2 + t^3$.
\end{example}

\begin{statement}
Асимптотическое поведение рекуррентности не зависит от начальных значений, оно зависит только от коэффициентов соотношений.
\end{statement}
   
\begin{statement}
    Пусть $\lambda_1, \lambda_2, \dots, \lambda_z$~--- минимальные корни максимальной кратности.

    $\lambda_j = \dfrac{e^{i \varphi_j}}{r}$. $\varphi_j = \dfrac{p_j}{q_j} \cdot 2\pi$.

    Пусть $\overline{q} = LCM(q_j)$. Тогда последовательность $a_i$ имеет асимптотическое поведение при $i \% \overline{q} = const$.
\end{statement}


\endinput