\section{Комбинаторика и производящие функции.}

Воспоминание о доминошках.

Комбинаторные обьекты это конструкции, которые состоят из атомов и разных связей атомов между собой. 
Под атомом мы понимаем некоторую неделимую часть комб. обьекта.
Давайте все наши комбинаторные обьекты сложим.

В этой сумме заменим каждый атом на $t^{\omega}$, где $\omega$ --- вес данного атома.
Потом $t^\omega$ атомов одного обьекта перемножим. 

Вес обьекта~--- сумма весов его атомов. 

\subsection*{Базовые объекты}
    Все, заданные перечиеслением. 
    \begin{enumerate}
        \item $U = \{u\}$, $w(u) = 1$. 
        Тогда $U(t) = t$.
        \item $B = \{ a,b\}$, $w(a) = w(b) = 1$.
        Тогда $B(t) = 2t$.
        \item $E = \{\epsilon\}$, $w(\epsilon) = 0$. 
        Тогда $E(t) = 1$.
        \item $E_k = \{ \epsilon_i\}_{i=1}^k$. $E_k(t) = k$.
        \item $D= \{a,A\}$, $w(a) = 1$, $w(A) = 2$. $D(t) = t + t^2$. 
    \end{enumerate}

\subsection*{Дюзъюнктное объединение}
$A, B, A\cap B = \emptyset$.
Пусть у комб. обеьктов $A$ производящая функция $A(t)$, у $B$ --- $B(t)$.
Тогда производящая функция их дизъюнктного объединения это $A(t)+B(t)$.

\subsection*{Упорядоченная пара, прямое произведение}
У прямого произведения производящая функция --- это произведение производящих функций.

Вспомним, что вес пары мы определяем как сумму весов компонент.

\subsection*{Конечные последовательности}
Рассмотрим комбинаторные объекты $A$. 
$Seq A = \{[], [A_1], [A_2], \ldots, [A_1, A_1], [A_1, A_2], \ldots, [A_1,A_2,A_3]\ldots  \}$.

\[ Seq A = [] + A_1([] + [A_1] + \ldots) + A_2([]) + [A_1] + \ldots = 1 + A \times Seq A \]
Заметим, что при $B = Seq A$ у нас получается $B(t) = 1 + A(t) B(t)$, откуда $B(t) = \frac{1}{1-{A(t)}}$.
Определение $B$ очень похоже на определение списка в функциональном языке.

Можно подойти с другой стороны, заметив здесь геометрическую прогрессию.
последовательность длины $k$ можно понимать как упорядоченную $k-$арку. 
\[ B(t) = 1 + A(t) + A^2(t) + \ldots A^k(t) + \ldots\] 
Отсюда сразу получится $B(t) = \frac{1}{1-{A(t)}}$.

% \subsection*{Примеры}
На примерах видно, что всё сходится. ДОПОЛНИТЬ.
Отметим, что $Seq A$ определен только в том случае, когда $a_0 = 0$, ведь иначе число обьектов веса 0 можно составить бесконечно много.

$Seq D = \frac{1}{1-t-t^2}$ --- числа Фиббоначи, привет.

\subsection*{Множества}
$Set, PSet$ --- комбинаторный объект, семейство всех множеств.
Обозначим $Set A$ за $C$.
Рассмотрим дополнительные комбинаторные обьекты, которые позволят закодировать, взяли ли мы в множество данный элемент из $A$ или нет.
\[B_a = \epsilon + a \]
Тогда $Set A = \prod_{i\in A}B_i$.
Значит $C(t) = \prod_{a\in A}(1 + t^{w(a)}) = \prod_{n=0}^\infty (1+t^n)^{a_n}$.

Примеры. 
$U = \{ u \}$. 
$Set U = C$, $C(t) = (1+t)^1 = 1+t$.

$B = \{ a, A \}$.
$C = Set B$, $C(t) = \prod (1+t^n)^{b_n} = 1 + t + t^2 + t^3$.

Заметим, что $\prod_{n=0}^\infty (1+t^n)^{a_n} = 2^{a_0} \prod (1+t^n)^{a_n}$.
Это обьясняется тем, что обьекты нулевого веса могут быть свободно включены в множество, не поменяв его веса.

\subsection*{Мультимножество}
$MSet$. 
Каждый обьект может включаться любое неотрицательное целое число раз.
\[a\in A \quad \epsilon + a + aa + \ldots  = Seq {a}\]
\[a_1, a_2\in A \quad Seq{a} \times Seq{a_2} = MSet \{ a_1 , a_2 \} \]
Итого $MSet A = \prod_{a\in A}Seq{a}$ и  $C(t) = \prod_{a\in A}\frac{1}{1-t^{w(a)}} = \prod_{n=1}^\infty(\frac{1}{1-t^n})^{a_n} = \prod (1-t^n)^{-a_n}.$

\subsection*{Циклы}

\endinput