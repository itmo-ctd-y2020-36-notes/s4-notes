\documentclass[10pt]{article}
% \DeclareMathOperator{\Sym}{Sym}
% \DeclareMathOperator{\Inn}{Inn}
% \DeclareMathOperator{\Conj}{Conj}
% \DeclareMathOperator{\Aut}{Aut}
% \DeclareMathOperator{\Transv}{Transv}
\usepackage{preference}
% \DeclareMathOperator{\Re}{Re}

\newcommand{\p}[1]{#1^{\prime}}
\newcommand{\pp}[1]{#1^{\prime\prime}}
\newcommand\N{\ensuremath{\mathbb{N}}}
\newcommand\R{\ensuremath{\mathbb{R}}}
\newcommand\Z{\ensuremath{\mathbb{Z}}}
\renewcommand\O{\ensuremath{\emptyset}}
\newcommand\Q{\ensuremath{\mathbb{Q}}}
\renewcommand\C{\ensuremath{\mathbb{C}}}
\newcommand{\tl}[1]{\widetilde{#1}}
\newcommand{\tll}[1]{\widetilde{\widetilde{#1}}}
\newcommand{\ov}[1]{\overline{#1}}
\let\svlim\lim\def\lim{\svlim\limits}
\let\svsum\sum\def\sum{\svsum\limits}
\usepackage{mdframed}
\mdfsetup{skipabove=1em,skipbelow=0em}
\theoremstyle{definition}
% \newmdtheoremenv[nobreak=true]{definition}{Определение}
% \newmdtheoremenv[nobreak=true]{theorem}{Теорема}
% \newmdtheoremenv[nobreak=true]{lemma}{Лемма}
% \newmdtheoremenv[nobreak=true]{problem}{Задача}
% \newmdtheoremenv[nobreak=true]{property}{Свойство}
% \newmdtheoremenv[nobreak=true]{statement}{Утверждение}
% \newmdtheoremenv[nobreak=true]{corollary}{Следствие}
\newtheorem*{note}{Замечание}
% \newtheorem*{example}{Пример}
\renewcommand\qedsymbol{$\blacksquare$}
\newcommand{\incfig}[1]{%
    \def\svgwidth{\columnwidth}
    \import{./figures/}{#1.pdf_tex}
}
\usepackage{tikz}
\newcommand{\declare}[1]{
    \expandafter\DeclareMathOperator\csname #1\endcsname{#1}
}
\declare{Int}
\declare{rang}
\declare{Cl}
\declare{Tq}
\declare{Tp}
\declare{Lp}
\declare{diam}
\declare{diag}
\undef\Re
\declare{Re}
\undef\Im
\declare{Im}

\begin{document}

\def\chap#1#2{\ \\ {\large\bf#1 \ | \ \tt\scshape#2} \par}

\ \vspace{-1cm}

{\bf
\ \\
\Large\centerline{\scshape Матан, лекции}
}\normalsize

\input{parts/lection1.tex}
\begin{theorem}
    [Критерий плотности]

    $\sqsupset (X, \mathscr{A})$ -- измеримое пространство, $\mu, \nu$ -- опр. (?) $\mathscr{A}$

    $h\in S_+(X)$. Тогда следующие утверждения равносильны:
    \begin{enumerate}
        \item $h$ -- плотность меры $\nu$ относительно $\mu$ ($\forall E\in \mathscr A\quad \nu(E) = \int_E hd\mu$)
        \item $\forall E \in\mathscr{A}$
        \[\inf_E h \cdot \mu(E) \leqslant d(E) \leqslant \sup_E h \cdot \mu(E) \]
    \end{enumerate}

    Если $(X, \mathscr{A}, \mu) = (\R^n, \mathcal{A}, \lambda_n)$, тогда $1 \iff 3$:
    \begin{enumerate}
        \item [3] \[\forall P\in \mathcal P_n\quad \inf_P h\cdot \mu(P) \leqslant \nu(P) \leqslant \sup_P h\cdot \mu(P)\]
    \end{enumerate}
\end{theorem}
\begin{proof}
    План: $1 \implies 2\implies 3$

    \begin{itemize}
        \item [$2\implies 1$?] $\sphericalangle E\in \mathscr{A} \quad \nu(E) \overset ? = \int_E hd\mu$

        \[E = E\left\{ h = 0 \right\} \coprod E\left\{ h = +\infty \right\} \coprod E \left\{ 0<h<+\infty  \right\} \]

        \begin{align*}
            \nu(E) &= \nu(E\left\{ h = 0 \right\}) + \nu(E\left\{ h = + \infty  \right\} ) + \nu(E\left\{ 0<h<+\infty  \right\} )\\
            \nu(E\left\{ h = 0 \right\}) &\leqslant  \sup_{E\left\{ h = 0 \right\} } = 0 = \int_{E\{h = 0\}} hd\mu\\
            \nu(E\left\{ h = + \infty  \right\} ) &\leqslant h\cdot \mu(E) + \infty \cdot \mu(E) = \int_{E\{h = +\infty \}hd\mu} 
        .\end{align*}
    \end{itemize}

    $\sphericalangle \dfrac{1}{q} \in (0, 1), ~ q >1~~~ (0, +\infty) = \bigvee\limits{k\in \Z} [q^k, q ^{k+1})$

    $E\{ h \in (0, +\infty)\} = \bigvee E \{ q^k \leqslant h < q^{k+1} \}$

    $q^k \mu(E_k) \leqslant \nu(E_k)\leqslant q^{k+1} \cdot \mu(E_k)$

    $q^k \mu(E_k) \leqslant \int h d\mu \leqslant q^{k+1} \cdot \mu(E_k)$

    $\dfrac{\nu(E_k)}{q} \leqslant q^k \cdot \mu(E_k) \leqslant \int_{E_k} h d\mu = 
    q \cdot q^{k} \mu(E_k) \leqslant q \cdot \nu(E_k)$

    Просуммируем это по всем $k$.
    
    $\dfrac{1}{q} \nu(E) = \int_E h d\mu \leqslant q \cdot \nu(E)$,~~$q \to 1 \implies$
    $\nu(E) \leqslant \int_E h d\mu \leqslant\nu(E) \implies \nu(E) = \int_E hd\mu$
    
    $\sphericalangle \tl \nu$ -- стандартное продолжение <...> (нужно дополнить)
\end{proof}

\begin{theorem}
    $\sqsupset \Phi$ -- диффеоморфизм множеств $G, O\subseteq \R^n\quad G \underset \Phi \rightarrow O$

    Тогда $\forall E\in \mathscr A_n\quad E\subseteq O$
    
    \[\lambda_n (E) = \int_{\Phi^{-1}(E)} \bigg| \det \p \Phi \bigg| d \lambda_n
    \]

    \[\lambda_n(O) = \int_G \left|\det \p \Phi  \right|d\lambda_n \]
    
    Если $O\sim \tl O\quad G\sim \tl G\quad \left( \lambda_n(O \setminus \tl O) = \O \ldots \right) $, то
    \[\lambda_n(\tl O) = \int_{\tl G} \left|\det \p \Phi  \right|d\lambda_n \]
\end{theorem}
\begin{note} % я добавлю добавил :thumbup:
    \begin{align*}
        \nu(P) \leqslant \sup_P hd\mu(P)\text{ -- от противного}\\
        \implies \exists \text{ ячейки } P_0:\quad \nu(P) > M\cdot \mu(P) = \sup_{P_0} h \cdot \mu(P)\\
        \Phi(x) = \Phi(x_0) + d_{x_0}\Phi(x - x_0) + o(x - x_0)\\
        x \approx x_0\qquad \Phi(x) \approx \Phi(x_0) + d_{x_0}\Phi(x - x_0)\\
    .\end{align*}

    Если $Q$ -- малая ячейка, то \[\lambda_n(\Phi(Q)) \approx \lambda_n d_{x_0}\Phi(Q) = \left| \det \p \Phi_{x_0} \right| \lambda_n(Q)\]
\end{note}

\begin{corollary}
    Если $\Phi: G \to O$ -- диффеоморфизм, $G, O \subseteq \R^n\quad \tl G \sim G, \tl O\sim O\quad f\in S(O)$, то 
    \[\int_{\tl O} f(x)d\lambda_n(x) = \int_{\tl G} f(\Phi(u)) \left| \det \p \Phi(u) d\lambda_n(u) \right| \]
\end{corollary}

\begin{example}
    Полярные координаты.
    
    $x = r \cos \varphi,~ y = r \sin \varphi$.
    
    $\Phi: (r, \varphi) \to (x, y)$,\\
    $([0, +\infty) \times [-\pi, \pi])) \to \R^n$,\\
    $(0, +\infty] \times (-\pi, \pi))) \to \R^n \setminus(-\infty, 0])$.

    $\det \p \Phi =r; ~~~E = \R^2: $
    \[
        \iint_E f(x, y) dx dy = \iint_{\Phi^{-1}} f(r\cos \varphi , r \sin\varphi) r dr d\varphi 
    \] 
 \end{example}

\begin{example}
    [интеграл Эйлера-Пуассона]

    \begin{align*}
        I &= \int_0^{+\infty }e^{-x^2}dx\\
        I\cdot I &= \int_0^{+\infty }e^{-x ^2}dx \cdot \int_0^{+\infty }e^{-ys} &= \iint _{\{x \geqslant 0, y\geqslant 0\}} e^{-x^2 + y^2}dxdy\\
        &= \iint_{\left\{ 0\leqslant \varphi \leqslant \frac{\pi}{2}\quad r >=0 \right\} } e^{-r^2}rdrd\varphi\\ 
        &=\int_0^{\frac{\pi}{2}}d\varphi \int_0^{+\infty }re^{-r^2}dr  \\% тут нет лишенего амперсанта? & ->  ^
        &= \frac{\pi}{2} \cdot \frac{e^{-r^2}}{-2} \mid_0^{+\infty } = \frac{\pi}{4}\\
        I = \int_0^{+\infty } r^{-x^2}dx = \frac{\sqrt \pi}{2}\\
    .\end{align*}
\end{example}

\begin{example}
    Цилиницрические координаты

    \begin{align*}
        r\cos \varphi = x\\
        r\sin \varphi = y\\
        h = z\\
    .\end{align*}

    $\Phi: (r, \varphi, h) \to (x, y, z)\quad \Phi: (0, +\infty )\times (-pi, pi)\times \R \to \R^3 \setminus \{(x, 0, z)|\mid x \leqslant 0\}\}$

    $\left| \det \p \Phi \right| = r  $

    $\iiint _E f(x, y, z)dxdydz = \iiint _{\Phi^{_1}(E)} f(r\cos\varphi, r\sin\varphi,h)\cdot rdrd\varphi dh$
\end{example}

\begin{example}
    Сферические координаты

    $r = \sqrt{x^2 + y ^2 + z^2}$
    \begin{align*}
        r\cos \varphi \cos \psi = x \\
        r\sin \varphi \cos \psi = y \\
        r\sin \psi \varphi \sin \psi = y
    \end{align*}

    $\det \p \Phi = r^2\cos\varphi$    
    
    Можно обобщить на $\R^n$

    \begin{align*}
        r = \|x\|\\
        x_1 = r\cos\varphi_{n-1}\cos\varphi_{n-2} \ldots \cos\varphi_{1}\\
        \ldots\\
        x_{n-2} = r\cos\varphi_{n-1}\cos\varphi_{n-2}\sin\varphi_{n-3}\\
        x_{n-1} = r\cos\varphi_{n-1}\sin \varphi_{n-2}\\
        x_n = r\sin\varphi_{n-1}\\
    .\end{align*}

\end{example}

\begin{example}
        \[\iiint\limits_{\substack{x^2+y^2+z^2 \leqslant \R^2\\ x^2 + y \leqslant z^2\\ z\geqslant 0}} f(x, y, z ) \,dx \,dy \,dz\]

        Преобразовать используя:
        \begin{itemize}
            \item Цилиндрические координаты

            Перепишем множество интегрирования в новых координатах:
            $\begin{cases}
                r^2+h^2 \leqslant R^2\\
                r^2\leqslant h^2 \implies r\leqslant h\\
                h\geqslant 0, r\geqslant 0
            \end{cases}$

            \begin{align*}
                I &= \iiint\limits_{\substack{r^2 + h^2 \leqslant R^2\\ r\leqslant h\\ h\geqslant 0, r\geqslant 0}} f\left( r\cos\varphi, r\sin\varphi, h \right) rdrd\varphi dh\\ 
                &= \iint\limits_{\substack{\pi \leqslant \varphi \leqslant  \pi\\ 0\leqslant r\leqslant \frac{R}{\sqrt 2}}} r \int_{r}^{\sqrt{R^2 - r^2} } f\left( r\cos\varphi, r\sin\varphi, h \right) dr \\
                &= \int_{-\pi}^{\pi}d\varphi \int_0^{\frac{R}{\sqrt 2}} rdr \int_r^{\sqrt{R^2 - r^2}} f\left( r\cos\varphi, r\sin\varphi, h \right) dh \\
            .\end{align*}

            \item Цилиндрические координаты (второй вариант)

             \begin{align*}
                 \int_0^{\frac{R}{\sqrt 2}} dh \int_{-\pi}^{\pi}d\varphi \int_0^h r fdr + \int_{\frac{R}{\sqrt 2}} dh \int_{-\pi}^{\pi}d\varphi \int_0^{\sqrt{R^2 - h^2}} rfdr\\
             .\end{align*}
            \item Сферические координаты

            $\begin{cases}
                x = r\cos\varphi\sin\psi\\
                y = r\sin\varphi\cos\psi\\
                z = r\sin\psi\\
                \tg^2 \psi \geqslant 0\\
                \sin\psi \geqslant 0\\
            \end{cases}$

            \begin{align*}
               0\leqslant r \leqslant R\\ 
                r^2\cos^2\psi \leqslant r^2\sin^2 \psi\\
                r\sin\psi \geqslant 0\\
            .\end{align*}

            \begin{align*}
                I &= \iiint_E f\left(r\cos\varphi\cos\psi, r\sin\varphi\cos\psi, r\sin\varphi  \right) r^2\cos\psi dr d\varphi d\psi \\
                &= \int_{-\pi}^{\pi}d\varphi \int_{\frac{\pi}{4}}^{\frac{\pi}{2}}d\psi \int_0^R f(\ldots)r^2\cos\psi dr \\
            .\end{align*}
        \end{itemize}
\end{example}

\begin{example}
    \[\iiiint_E z dxdydz\]

    $E:$
    \begin{align*}
       t^2(x^2 + y^2) \leqslant z^2\\
       0\leqslant z\leqslant t\leqslant 3\\ 
    .\end{align*}

    \begin{align*}
        \iiiint_E z dxdydz &= \iint_{\{0\leqslant z\leqslant t\leqslant 3\}} dzdt \iint_{\{x^2 + y^2 \leqslant \frac{4z^2}{t^2} \}} zdxdy \\
        &= \iint_{\{0\leqslant z\leqslant t\leqslant 3\}} dzdt z \pi \cdot \frac{4z^2}{z^2}\\   
        &= 4\pi \iint_{\{ 0\leqslant z \leqslant t \leqslant 3\}} \frac{z^3}{t^2}dzdt \\
        &= 4\pi \int_0^3 \frac{1}{t^2} dt \int_0^t z^3dz = \frac{4\pi}{4}\left( \int_0^3 t^2dt \right) = \pi\cdot 9  \\
    .\end{align*}


\end{example}

\section{Мера Лебега--Стилтьеса}

$\sqsupset g(x)\uparrow$ на $\R$ и непрерывна слева $\left(\lim_{x \to x_0-0} g(x) \equiv g(x_0)\right)$

\begin{problem}
    Если $h(x)$ -- произвольная возрастающая функция, то её можно превратить в непрерывную слева исправлением нбчс количества точек.

    $\exists \uparrow$ и непрерывная слева $g(x) = h(x)$ всюду кроме точек разрыва $h(x)$
    
    $g(x_0) = \lim_{x \to x_0-0}h(x) $
\end{problem}

Определим $\mu_g ([a,b]) = g(b) - g(a) \geqslant 0 $. Так же верно, что $\mu_g$ обладает счетной аддитивностью на $\mathcal P_1$ (доказывается так же, как в случае с мерой Лебега) 
$\implies \mu_g $ -- мера на $\mathcal P_1$

Стандартное продолжение $\mu_g$, которое также обозначается $\mu_g$ называется мерой Лебега-Стилтьеса, порождённой функцией $g$

\begin{align*}
    \mu_g\left( \left\{ c \right\}  \right) &= \mu_g\left( \bigcap\limits_{j=1}^{\infty }[c, c + \frac{1}{j}]  \right)  \\
    &= \lim_{j \to \infty} \mu_g\left( [c, c + \frac{1}{j}] \right)  \\
    &= \underbrace{\lim_{j \to \infty} g(c + \frac{1}{j})}_{ = g(c + 0)} - g(c) = g(c+0) - g(c)  \\
.\end{align*}

$\implies $ Если $c$ -- точка непрерывности, то $\mu_g(\{c\}) = 0$

$\mu_g ([a, b]) = \mu_g([a, b]) + \mu_g(\{b \}) = g(b) - g(a) +g(b + 0) - g(b) = \left( g(b + 0) - g(a - 0)\right)$
    
$\mu_g\left( \left( a, b \right)  \right) = \mu_g\left( [a,b)] \right)  - \mu(\{a\}) = g(b) - g(a) - (g(a + 0) - g(a)) = g(b) - g(a+0) $

$\mu_g\left( \left( a, b \right] \right) = g(b + 0 ) - g(a + 0)$

\begin{definition}
    Пусть $\mu = \sum\limits_{k=1}^\infty h_k \delta_{a_k}$,~~~
    $h_k \geqslant 0$, ~~~ $\delta_a(E) = \begin{cases}
        1,~~~ a \in E\\
        0,~~~ a \not\in E
    \end{cases}$,~~~ тогда $\mu$ --- дискретная мера.

\end{definition}

$E, E_j\in 2^{\R}\quad E = \bigvee\limits_{j=1}^{\infty }E_j \implies \delta_{a_k}(E) = \sum_{j=1}^{\infty } \delta_{a_k}(E_j)$

\begin{align*}
    \mu(E)  &= \mu(\bigvee_{j=1}^{\infty }E_j) = \sum_k\sum_j h_k\delta_{a_k}(E_j)\\
    &= \sum_j \mu(E_j)\\
.\end{align*}

Последний переход в равенстве по теореме Тонелли.

\begin{note}
    $\sqsupset \{a_k\}_{k=1}^{\infty }\subseteq \R$

    $\forall [a,b]\qquad \sum_{k: a_k\in [a,b]}h_k < + \infty $
\end{note}

\begin{example}
    Если $\{a_k\}$ -- дискретно (без точек сгущения на $\R$), то условие автоматически выполняется, т.к. перечесечения $a_k$-ых с промежутком будет конечно,
    а значит и сама сумма будет конечна

    $A = \Q\quad h_k = \dfrac{1}{2^k}$
\end{example}

\begin{definition}
    [функция Хэвисайда]
    \[\Theta(x) = \begin{cases}
        0&, x\leqslant 0\\
        1&, x>0
    \end{cases}\]

    $\sqsupset x_0\in \R\quad \forall C\in \R$

    \[g(x) = \sum_{k=1}^{\infty } h_k\cdot \left(\Theta(x - a_k) - \Theta(x_0 - a_k)\right) + C\]

    \begin{enumerate}
        \item $g(x)$ возрастает
        \item $x\in [a,b]\quad \sum_k h_k(\Theta(x - a_k) - \Theta(x_0 - a_k)) \leqslant \sum_{a_k:I_{x, x_0}} h_k$

        Разность Тет ненулевая, если $a_k$ находится между $x$ и $x_0$ -- $I_{x, x_0}$
    \end{enumerate}
\end{definition}

\begin{statement}
    $A = \{a_k\}_k$
    \begin{enumerate}
        \item $g \in C\left( \R \setminus A \right) $
        \item Непрерывность слева на $A$
    \end{enumerate}
\end{statement}
\begin{proof}
\begin{enumerate}
    \item $\sqsupset x\in \R \setminus A\quad \sqsupset (a,b)\ni x$

    $\sphericalangle \forall \varepsilon >0 \quad \sum_{k: a_k\in [a,b] } h_k < +\infty  \implies \exists K: \sum _{\substack{a_k\in [a, b]\\ k \geqslant K }} \leqslant \dfrac{\varepsilon}{2}$.
    
    $g_k (x) = h_k \left(  \Theta (x- a_k) - \Theta(x_0 - a_k)\right)$ --- локально постоянны в точке $x$ ($\exists V_{\delta}(x)~:~ g_k\mid_{V_{\delta}(x)}\equiv const$ для $k = 1, \ldots, k$)

    Не умаляя общности $[a,b] \supseteq V_{\delta}(x) $

    \begin{align*}
    g(\tl x) - g(x) &= \sum_{k=1}^{\infty } h_k\left( \Theta(x - a_k) - \Theta(x - a_k) \right) - \sum_{k=1}^{\infty }h_k \left( \Theta(\tl x - a_k) - \Theta(x_0 - a_k) \right)\\ 
    &= \sum_{k=1}^{\infty } h_k \left( \Theta(x - a_k) - \Theta(\tl x - a_k) \right)  \\
    &= \underbrace{\sum_{k=1}^{K} h_k \left( \Theta(x - a_k) - \Theta(\tl x - a_k) \right)}_{ = 0} + \underbrace{\sum_{k=K+1}^{\infty } h_k \left( \Theta(x - a_k) - \Theta(\tl x - a_k) \right)}_{ = \dfrac{\varepsilon}{2}}  \\
    .\end{align*}

    $\implies $ Непрерываность

    Если $x = a_k\quad g(x) = g_{k_0}(x) + \underbrace{\sum _{k \neq k_0}g_k}_{\text{непрерывна как в пред. случае}}$
\end{enumerate}
\end{proof}

\begin{align*}
    \mu_g\left([a,b)]  \right) &= g(b) - g(a)\\ 
    &= \sum_{k=1}^{\infty} h_k\left( \Theta(b - a_k) - \Theta(a - a_k) \right)\qquad a \leqslant  a_k \leqslant b\\
    &= \sum_{k: a\leqslant a_k < b} h_k = \mu([a,b)])\\
.\end{align*}

$\mu$ и $\mu_g$ совпадают на совокупности всевозможных промежутков.


\begin{definition}
    Пусть $f: \R \to \R$.

    Функция $f$ называется локально суммируемой  на $\R$
    $\iff \forall \, [a,b]\quad\quad f\bigg|_{[a,b]} \in \mathcal{L} (\lambda_1)$. 
\end{definition}

\begin{definition}
    $f: \R \to \R$.
    
    Функция $f$ называется абсолютно непрерывной, если существует локально суммируемая функция $h(x)$ и точка $x_0\in \R$: \[g(x) = \int_{x_0}^x h(x)d\lambda\] 
    (интеграл Лебега. Если $x < x_0$, то $\int_{x_0}^x h\, d\lambda = - \int_{[x, x_0]}h\, d\lambda$)
    
    Если $h$ непрерывна в точке $x$, то $g(x)$ дифференцируема в точке $x$ и $\p g(x) = h(x)$. Доказательство  -- смотри теорему Барроу\ldots
    
    Если $h(x) \geqslant 0$, то $g(x)\nearrow $

    Функция $g(x)$ непрерывна на $\R$. Следует из абсолютной непрерывности интеграла.

\end{definition}
\endinput
\end{document}