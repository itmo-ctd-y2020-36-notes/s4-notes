\documentclass[10pt]{article}
% \DeclareMathOperator{\Sym}{Sym}
% \DeclareMathOperator{\Inn}{Inn}
% \DeclareMathOperator{\Conj}{Conj}
% \DeclareMathOperator{\Aut}{Aut}
% \DeclareMathOperator{\Transv}{Transv}
\usepackage{preference}
% \DeclareMathOperator{\Re}{Re}

\newcommand{\p}[1]{#1^{\prime}}
\newcommand{\pp}[1]{#1^{\prime\prime}}
\newcommand\N{\ensuremath{\mathbb{N}}}
\newcommand\R{\ensuremath{\mathbb{R}}}
\newcommand\Z{\ensuremath{\mathbb{Z}}}
\renewcommand\O{\ensuremath{\emptyset}}
\newcommand\Q{\ensuremath{\mathbb{Q}}}
\renewcommand\C{\ensuremath{\mathbb{C}}}
\newcommand{\tl}[1]{\widetilde{#1}}
\newcommand{\tll}[1]{\widetilde{\widetilde{#1}}}
\newcommand{\ov}[1]{\overline{#1}}
\let\svlim\lim\def\lim{\svlim\limits}
\let\svsum\sum\def\sum{\svsum\limits}
\usepackage{mdframed}
\mdfsetup{skipabove=1em,skipbelow=0em}
\theoremstyle{definition}
% \newmdtheoremenv[nobreak=true]{definition}{Определение}
% \newmdtheoremenv[nobreak=true]{theorem}{Теорема}
% \newmdtheoremenv[nobreak=true]{lemma}{Лемма}
% \newmdtheoremenv[nobreak=true]{problem}{Задача}
% \newmdtheoremenv[nobreak=true]{property}{Свойство}
\newmdtheoremenv[nobreak=true]{statement}{Утверждение}
% \newmdtheoremenv[nobreak=true]{corollary}{Следствие}
\newtheorem*{note}{Замечание}
% \newtheorem*{example}{Пример}
\renewcommand\qedsymbol{$\blacksquare$}
\usepackage{import}
\newcommand{\incfig}[1]{%
    \def\svgwidth{\columnwidth}
    \import{./figures/}{#1.pdf_tex}
}
\usepackage{tikz}
\newcommand{\declare}[1]{
    \expandafter\DeclareMathOperator\csname #1\endcsname{#1}
}
\declare{Int}
\declare{rang}
\declare{Cl}
\declare{Tq}
\declare{Tp}
\declare{Lp}
\declare{diam}
\declare{diag}
\undef\Re
\declare{Re}
\undef\Im
\declare{Im}

\begin{document}

\def\chap#1#2{\ \\ {\large\bf#1 \ | \ \tt\scshape#2} \par}

\ \vspace{-1cm}

{\bf
\ \\
\Large\centerline{\scshape Матан, лекции}
}\normalsize

\input{parts/lection1.tex}



\begin{theorem}
    [Фубини]

    \begin{align*}
        x = \left( x_1, \ldots, x_k \right) \\
        y = \left( y_1, \ldots, y_m \right) \\
        f(x, y) \in \mathscr L\left( E, \lambda_{k+m} \right) \\
        E\in \mathscr A_{k+m}\\
    .\end{align*}

    то:
    \begin{enumerate}
        \item Для почти всех $x\in \R^k \quad g\left( \cdot  \right)  = f\left( x, \cdot  \right) \in \mathscr L\left( E\left( x, \cdot  \right)  \right) $
        \item $I(x) = \int_{E\left( x, \cdot  \right) } f\left( x, y \right) d\lambda_m(y) \in \mathscr L\left( \R^k \right)   $\\
        \item 
            \begin{align*}
                \int_E f\left( x, y \right) d\lambda_{k+m}\left( x, y \right)  = \int_{\R^k} \left( \int_{E\left( x, \cdot  \right) } f\left( x, y \right) d\lambda_m(y) \right) d\lambda_k(x)\\
            .\end{align*}
    \end{enumerate}         
\end{theorem}

\begin{example}
    $E = A \times \{0\} \subseteq \R^{k + m} \quad_0\in \R^n$

    $A$ -- неизмеримое в  $\R^k$

    $E$ -- измеримо в  $\R^{k+m}$

    $Pr_x(E) = A$ -- неизмеримое

    Если  $Pr_x(E)$ измеримо, то вместо интеграла по $\R^k$ можно написать интеграл по проекции
\end{example}

\begin{figure}[!ht]
    \centering
    \incfig{proection}
    \caption{Переход в интегралу по проекции}
    \label{fig:proection}
\end{figure}

\begin{note}
    Если $E$ -- компактное или открытое, то  $Pr_x(E)$ измеримо.

    $Pr_x(E) = \Phi(E)$, где  $\Phi(x,y)\equiv x$ -- отображение проектирования

    Если  $E$ -- компактное, то  $\Phi(E)$ -- компактное. Если открытое, то открытое.
\end{note}

\begin{example}
    \begin{enumerate}
        \item 
            \begin{align*}
                \int_0^1dx\int_0^1 \frac{x^2-y^2}{\left( x^2+y^2 \right) ^2}dy = I_1\\
                \int_0^1dy\int_0^1 \frac{x^2-y^2}{\left( x^2+y^2 \right) ^2}dx = I_2\\
            .\end{align*}
            Если интегралы существуют, то они антиравны.

             \begin{align*}
                 I_1 = \int_0^1 \frac{y}{\left( x^2+y^2 \right) }_{y = 0}^{y = 1}dx = \int_0^1 \frac{1}{x^2+1} - 0dx = \arctg x |_0^1 = \frac{\pi}{4}
            .\end{align*}

            Вывод: функция $f(x,y) \not\in \mathscr L\left( [0,1]^2, \lambda_2 \right) $
        \item 
            \begin{align*}
                \int _{-1}^1 dx\int_{-1}^1 \frac{xy}{\left( x^2+y^2 \right) ^2}dy\\
                \int _{-1}^1 dy\int_{-1}^1 \frac{xy}{\left( x^2+y^2 \right) ^2}dx\\
            \end{align*}
            \begin{align*}
                    f\in \mathscr {L}^2\left( [-1,1]^2 \right)  \iff |f|\in \mathscr{L} \left( [-1,1]^2 \right)  \implies |f|\in \mathscr{L} \left( [-1,1]^2 \right) 
            \end{align*}
            \begin{align*}
                    \iint _{[0,1]^2} f\left( x,y \right) dy = \int _0^1 dx \int _0^1 \frac{xy}{\left( x^2+y^2 \right) ^2}dx
            \end{align*}
    \end{enumerate}
\end{example}

<.....>

\begin{statement}
    Семейство называется суммируемым, если функция суммируема 
\end{statement}

\begin{statement}
    Если семейство $\left( a_x \right)_{x\in X} $ суммируемо, то $\{x: a_x \neq 0\}$ -- не более чем счётное.
\end{statement}
\begin{proof}
    Не умаляя общности $a_x \geqslant 0$

    $+\infty  > \int_X a_xdv = \int _{X_0}a_xdv > \int a_x dv \geqslant \frac{1}{j}\nu\left( x_j \right) \implies \nu(x_j) <+\infty  $ 

    $X_0 = \bigcup\limits_{j=1}^{\infty }X_j $ -- не более, чем счётное
\end{proof}

\begin{statement}
    $\sqsupset X$ -- н.б.ч.с, $Y$ -- числовое множество,  $\left( a_x \right) _{x\in X}\subseteq Y\quad \varphi: \N  \to X$

    Тогда $\left( a_x \right) $ суммируемы $\iff \sum_{k=1}^{\infty } a_{\varphi(k)}$ сходится абсолютно.
\end{statement}

\section{Замена переменной в интеграле по мере}

\subsection{``Пересадка'' меры}

$\Phi: X \to Y$ . $\sqsupset \left( X, \mathscr A, \mu \right) $ -- пространство с мерой.

$\mathscr D = \left\{ B\subseteq Y | \Phi^{-1}(B)\in \mathscr A \right\} $

$\Phi^{-1}\left( \bigcap\limits_{k=1}^{\infty }B_k  \right) = \bigcap\limits_{k=1}^{\infty }\Phi^{-1}\left( B_k \right) \in \mathscr A $ 

$\nu\left( B \right)  = \mu\left( \Phi^{-1}\left( B \right)  \right) $ 

\begin{example}
    $X = [0,2\pi )\quad \mathscr A = \mathscr A_1 \cap [0, 2\pi )$

    $\Phi(t\in X) = \left( \cos t, \sin t \right) $
\end{example}

\begin{theorem}
    [Общая схема замены переменных]

    $\sqsupset \left( X, \mathscr A, \mu \right) \quad \left( Y, \mathscr{D}, \nu \right) $

    $\Phi: X \to Y$ -- не портит измеримость.

    $\sqsupset h\in S_+(X): \forall B\in \mathscr D$

    \[\nu(B) = \int_{\Phi^{-1}(B)}hd\mu\]

    Тогда $\forall f\in f\in S\left( Y, \nu \right)$
    \[\int_Y fd\nu = \int_X f\left( \Phi(x) \right) h(x) d\mu(x) \]
\end{theorem}
\begin{proof}
    $f\circ \Phi$ -- измерима?

    $X \left\{ f\circ \Phi < a \right\}  = \Phi^{-1}\left( Y\left\{ f<a \right\}  \right) $. $Y\left\{ f<a \right\} \in \mathscr L$, т.к. $f$ измеримо. А тогда  $\Phi^{-1}\left( \ldots \right) \in\mathscr A$ 

    Совпадение интегралов:
    \begin{enumerate}
        \item $f$ -- ступенчатая,  $f = \sum_{k=1}^{K} C_k\chi_{D_k}\quad \left\{ D_k \right\} $ -- разбиение $X$

             \begin{align*}
                 \int_Y fd\nu = \sum_{k=1}^{K} C_k \nu\left( D_k \right)  = \sum_{k=1}^{K} C_k\int_{\Phi^{-1}\left( D_k \right)} h d\mu &=  \\
                 &= \int_X \left( \sum_{k=1}^{K} C_k\chi_{\Phi^{-1}\left( D_k \right)} <?>  \right)  \\
                 &= \int_X f\circ \Phi(x)h(x)d\mu(x) \\
                 f\circ \Phi(x) = C_k\quad x\in \Phi^{-1}(D_k)\\
                 \sum_{k=1}^{K} C_k\chi_{\Phi^{-1}(D_k)}(x) = C_k
             .\end{align*}
         \item $f\in S_+(Y)\quad \exists \left\{ g_{j} \right\} $ -- ступенчатая небобратимая $g_i\uparrow f$

              \begin{align*}
                  \int_Y fd\nu = \lim_{j \to \infty} \int_Yg_jd\nu = \lim_{j \to \infty} \int_X g_j\left( \Phi(x) \right) h(x)d\mu\\
                  &= \int_X f\left( \Phi(x) \right) h(x)dmu(x).
                \end{align*}

         \item Общий случай:

             $f = f_+ + f_-$

              \begin{align*}
                  \int_Y fd\nu &= \int_Y f_+ - \int_Y f_-d\mu = \int_X f_+\left( \Phi(x) \right) h(x)d\mu(x) - \int_Y f_-\left( \Phi(x) \right) h(x)d\mu(x)\\
                  &= \int f\left( \Phi(x) \right) h(x)d\mu(x)~~~
                  \left( f\left( \Phi \right) h \right) _+ = f_+\left( \Phi \right) h
             .\end{align*}
    \end{enumerate}
\end{proof}

\begin{corollary}
    $\sqsupset \left( X, \mathscr A, \mu \right) \quad \left( Y, \mathscr D, \nu \right) $

    $h\in S_+(X);\quad \Phi:X \to Y\quad \Phi^{-1}(\mathscr D)\subseteq \mathscr A$ 

    и выполняется условие теоремы общей замены переменной. Тогда $\forall E\subseteq \mathscr D\quad f\in S\left( E, \nu \right) $:
    \[\int_E f(y)d\nu(y) = \int_{\Phi^{-1}(E) f\left( \Phi(x) \right) h(x)d\mu(x)}\]

    Рассмотрим продолжение нулём $f$ с  $E$ на  $Y$

    \[\int_E fd\nu = \int_Y (y)\chi_E(y)d\nu(y) = \int_X f\left( \Phi(x) \right) \underbrace{\chi_E\left( \Phi(x) \right)}){\chi_{\Phi^{-1}(E)}} h(x)d\mu(x) = \int_{\Phi^{-1}(E)} f\left( \Phi(x)h(x)d\mu(x) \right). \]
\end{corollary}

\begin{corollary}
    [частный случай 1]

    Если $h \equiv 1$ в условии теоремы.

    ($\forall E|in \mathscr D\quad \nu(E) = \int_{\Phi^{-1}(E)}d\mu = \mu\left( \Phi^{-1}\left( E \right)  \right) $) 

    мера $\nu$ при этом называется образом меры  $\mu$

    \[\forall  f\in S(E)\quad \int_E f d\nu  = \int_{\Phi^{-1}(E)} f\circ \Phi(x)d\mu(x)\]
\end{corollary}

\begin{corollary}
    [Частный случай 2]

    $X = Y\quad \Phi = id\quad \nu(E) = \int_E h(x)d\mu(x)$
\end{corollary}

<..>

\begin{theorem}
    $\sqsupset \left( X, \mathscr A, \mu \right) $ -- пространство с мерой, $\Phi:X \to Y\quad h\in S_+(X)$ 

    Следующие утверждения равносильны: 
    \begin{enumerate}
        \item $h$ плотность  $\nu$ относительно  $\mu$
        \item $\forall E\in \mathscr A$ \[\inf_E h \mu E \leqslant \nu(E) \leqslant  \sup_D h\mu(E)\]
    \end{enumerate}
\end{theorem}
\begin{proof}
    $I \iff \forall E\in \mathscr A\quad \nu(E) = \int_E h d\mu$

    Т.о. $I \implies II$
\end{proof}

\begin{theorem}
    [Критерий плотности]

    $\sqsupset (X, \mathscr{A})$ -- измеримое пространство, $\mu, \nu$ -- опр. (?) $\mathscr{A}$

    $h\in S_+(X)$. Тогда следующие утверждения равносильны:
    \begin{enumerate}
        \item $h$ -- плотность меры $\nu$ относительно $\mu$ ($\forall E\in \mathscr A\quad \nu(E) = \int_E hd\mu$)
        \item $\forall E \in\mathscr{A}$
        \[\inf_E h \cdot \mu(E) \leqslant d(E) \leqslant \sup_E h \cdot \mu(E) \]
    \end{enumerate}

    Если $(X, \mathscr{A}, \mu) = (\R^n, \mathcal{A}, \lambda_n)$, тогда $1 \iff 3$:
    \begin{enumerate}
        \item [3] \[\forall P\in \mathcal P_n\quad \inf_P h\cdot \mu(P) \leqslant \nu(P) \leqslant \sup_P h\cdot \mu(P)\]
    \end{enumerate}
\end{theorem}
\begin{proof}
    План: $1 \implies 2\implies 3$

    \begin{itemize}
        \item [$2\implies 1$?] $\sphericalangle E\in \mathscr{A} \quad \nu(E) \overset ? = \int_E hd\mu$

        \[E = E\left\{ h = 0 \right\} \coprod E\left\{ h = +\infty \right\} \coprod E \left\{ 0<h<+\infty  \right\} \]

        \begin{align*}
            \nu(E) &= \nu(E\left\{ h = 0 \right\}) + \nu(E\left\{ h = + \infty  \right\} ) + \nu(E\left\{ 0<h<+\infty  \right\} )\\
            \nu(E\left\{ h = 0 \right\}) &\leqslant  \sup_{E\left\{ h = 0 \right\} } = 0 = \int_{E\{h = 0\}} hd\mu\\
            \nu(E\left\{ h = + \infty  \right\} ) &\leqslant h\cdot \mu(E) + \infty \cdot \mu(E) = \int_{E\{h = +\infty \}hd\mu} 
        .\end{align*}
    \end{itemize}

    $\sphericalangle \dfrac{1}{q} \in (0, 1), ~ q >1~~~ (0, +\infty) = \bigvee\limits{k\in \Z} [q^k, q ^{k+1})$

    $E\{ h \in (0, +\infty)\} = \bigvee E \{ q^k \leqslant h < q^{k+1} \}$

    $q^k \mu(E_k) \leqslant \nu(E_k)\leqslant q^{k+1} \cdot \mu(E_k)$

    $q^k \mu(E_k) \leqslant \int h d\mu \leqslant q^{k+1} \cdot \mu(E_k)$

    $\dfrac{\nu(E_k)}{q} \leqslant q^k \cdot \mu(E_k) \leqslant \int_{E_k} h d\mu = 
    q \cdot q^{k} \mu(E_k) \leqslant q \cdot \nu(E_k)$

    Просуммируем это по всем $k$.
    
    $\dfrac{1}{q} \nu(E) = \int_E h d\mu \leqslant q \cdot \nu(E)$,~~$q \to 1 \implies$
    $\nu(E) \leqslant \int_E h d\mu \leqslant\nu(E) \implies \nu(E) = \int_E hd\mu$
    
    $\sphericalangle \tl \nu$ -- стандартное продолжение <...> (нужно дополнить)
\end{proof}

\begin{theorem}
    $\sqsupset \Phi$ -- диффеоморфизм множеств $G, O\subseteq \R^n\quad G \underset \Phi \rightarrow O$

    Тогда $\forall E\in \mathscr A_n\quad E\subseteq O$
    
    \[\lambda_n (E) = \int_{\Phi^{-1}(E)} \bigg| \det \p \Phi \bigg| d \lambda_n
    \]

    \[\lambda_n(O) = \int_G \left|\det \p \Phi  \right|d\lambda_n \]
    
    Если $O\sim \tl O\quad G\sim \tl G\quad \left( \lambda_n(O \setminus \tl O) = \O \ldots \right) $, то
    \[\lambda_n(\tl O) = \int_{\tl G} \left|\det \p \Phi  \right|d\lambda_n \]
\end{theorem}
\begin{note} % я добавлю добавил :thumbup:
    \begin{align*}
        \nu(P) \leqslant \sup_P hd\mu(P)\text{ -- от противного}\\
        \implies \exists \text{ ячейки } P_0:\quad \nu(P) > M\cdot \mu(P) = \sup_{P_0} h \cdot \mu(P)\\
        \Phi(x) = \Phi(x_0) + d_{x_0}\Phi(x - x_0) + o(x - x_0)\\
        x \approx x_0\qquad \Phi(x) \approx \Phi(x_0) + d_{x_0}\Phi(x - x_0)\\
    .\end{align*}

    Если $Q$ -- малая ячейка, то \[\lambda_n(\Phi(Q)) \approx \lambda_n d_{x_0}\Phi(Q) = \left| \det \p \Phi_{x_0} \right| \lambda_n(Q)\]
\end{note}

\begin{corollary}
    Если $\Phi: G \to O$ -- диффеоморфизм, $G, O \subseteq \R^n\quad \tl G \sim G, \tl O\sim O\quad f\in S(O)$, то 
    \[\int_{\tl O} f(x)d\lambda_n(x) = \int_{\tl G} f(\Phi(u)) \left| \det \p \Phi(u) d\lambda_n(u) \right| \]
\end{corollary}

\begin{example}
    Полярные координаты.
    
    $x = r \cos \varphi,~ y = r \sin \varphi$.
    
    $\Phi: (r, \varphi) \to (x, y)$,\\
    $([0, +\infty) \times [-\pi, \pi])) \to \R^n$,\\
    $(0, +\infty] \times (-\pi, \pi))) \to \R^n \setminus(-\infty, 0])$.

    $\det \p \Phi =r; ~~~E = \R^2: $
    \[
        \iint_E f(x, y) dx dy = \iint_{\Phi^{-1}} f(r\cos \varphi , r \sin\varphi) r dr d\varphi 
    \] 
 \end{example}

\begin{example}
    [интеграл Эйлера-Пуассона]

    \begin{align*}
        I &= \int_0^{+\infty }e^{-x^2}dx\\
        I\cdot I &= \int_0^{+\infty }e^{-x ^2}dx \cdot \int_0^{+\infty }e^{-ys} &= \iint _{\{x \geqslant 0, y\geqslant 0\}} e^{-x^2 + y^2}dxdy\\
        &= \iint_{\left\{ 0\leqslant \varphi \leqslant \frac{\pi}{2}\quad r >=0 \right\} } e^{-r^2}rdrd\varphi\\ 
        &=\int_0^{\frac{\pi}{2}}d\varphi \int_0^{+\infty }re^{-r^2}dr  \\% тут нет лишенего амперсанта? & ->  ^
        &= \frac{\pi}{2} \cdot \frac{e^{-r^2}}{-2} \mid_0^{+\infty } = \frac{\pi}{4}\\
        I = \int_0^{+\infty } r^{-x^2}dx = \frac{\sqrt \pi}{2}\\
    .\end{align*}
\end{example}

\begin{example}
    Цилиницрические координаты

    \begin{align*}
        r\cos \varphi = x\\
        r\sin \varphi = y\\
        h = z\\
    .\end{align*}

    $\Phi: (r, \varphi, h) \to (x, y, z)\quad \Phi: (0, +\infty )\times (-pi, pi)\times \R \to \R^3 \setminus \{(x, 0, z)|\mid x \leqslant 0\}\}$

    $\left| \det \p \Phi \right| = r  $

    $\iiint _E f(x, y, z)dxdydz = \iiint _{\Phi^{_1}(E)} f(r\cos\varphi, r\sin\varphi,h)\cdot rdrd\varphi dh$
\end{example}

\begin{example}
    Сферические координаты

    $r = \sqrt{x^2 + y ^2 + z^2}$
    \begin{align*}
        r\cos \varphi \cos \psi = x \\
        r\sin \varphi \cos \psi = y \\
        r\sin \psi \varphi \sin \psi = y
    \end{align*}

    $\det \p \Phi = r^2\cos\varphi$    
    
    Можно обобщить на $\R^n$

    \begin{align*}
        r = \|x\|\\
        x_1 = r\cos\varphi_{n-1}\cos\varphi_{n-2} \ldots \cos\varphi_{1}\\
        \ldots\\
        x_{n-2} = r\cos\varphi_{n-1}\cos\varphi_{n-2}\sin\varphi_{n-3}\\
        x_{n-1} = r\cos\varphi_{n-1}\sin \varphi_{n-2}\\
        x_n = r\sin\varphi_{n-1}\\
    .\end{align*}

\end{example}

\begin{example}
        \[\iiint\limits_{\substack{x^2+y^2+z^2 \leqslant \R^2\\ x^2 + y \leqslant z^2\\ z\geqslant 0}} f(x, y, z ) \,dx \,dy \,dz\]

        Преобразовать используя:
        \begin{itemize}
            \item Цилиндрические координаты

            Перепишем множество интегрирования в новых координатах:
            $\begin{cases}
                r^2+h^2 \leqslant R^2\\
                r^2\leqslant h^2 \implies r\leqslant h\\
                h\geqslant 0, r\geqslant 0
            \end{cases}$

            \begin{align*}
                I &= \iiint\limits_{\substack{r^2 + h^2 \leqslant R^2\\ r\leqslant h\\ h\geqslant 0, r\geqslant 0}} f\left( r\cos\varphi, r\sin\varphi, h \right) rdrd\varphi dh\\ 
                &= \iint\limits_{\substack{\pi \leqslant \varphi \leqslant  \pi\\ 0\leqslant r\leqslant \frac{R}{\sqrt 2}}} r \int_{r}^{\sqrt{R^2 - r^2} } f\left( r\cos\varphi, r\sin\varphi, h \right) dr \\
                &= \int_{-\pi}^{\pi}d\varphi \int_0^{\frac{R}{\sqrt 2}} rdr \int_r^{\sqrt{R^2 - r^2}} f\left( r\cos\varphi, r\sin\varphi, h \right) dh \\
            .\end{align*}

            \item Цилиндрические координаты (второй вариант)

             \begin{align*}
                 \int_0^{\frac{R}{\sqrt 2}} dh \int_{-\pi}^{\pi}d\varphi \int_0^h r fdr + \int_{\frac{R}{\sqrt 2}} dh \int_{-\pi}^{\pi}d\varphi \int_0^{\sqrt{R^2 - h^2}} rfdr\\
             .\end{align*}
            \item Сферические координаты

            $\begin{cases}
                x = r\cos\varphi\sin\psi\\
                y = r\sin\varphi\cos\psi\\
                z = r\sin\psi\\
                \tg^2 \psi \geqslant 0\\
                \sin\psi \geqslant 0\\
            \end{cases}$

            \begin{align*}
               0\leqslant r \leqslant R\\ 
                r^2\cos^2\psi \leqslant r^2\sin^2 \psi\\
                r\sin\psi \geqslant 0\\
            .\end{align*}

            \begin{align*}
                I &= \iiint_E f\left(r\cos\varphi\cos\psi, r\sin\varphi\cos\psi, r\sin\varphi  \right) r^2\cos\psi dr d\varphi d\psi \\
                &= \int_{-\pi}^{\pi}d\varphi \int_{\frac{\pi}{4}}^{\frac{\pi}{2}}d\psi \int_0^R f(\ldots)r^2\cos\psi dr \\
            .\end{align*}
        \end{itemize}
\end{example}

\begin{example}
    \[\iiiint_E z dxdydz\]

    $E:$
    \begin{align*}
       t^2(x^2 + y^2) \leqslant z^2\\
       0\leqslant z\leqslant t\leqslant 3\\ 
    .\end{align*}

    \begin{align*}
        \iiiint_E z dxdydz &= \iint_{\{0\leqslant z\leqslant t\leqslant 3\}} dzdt \iint_{\{x^2 + y^2 \leqslant \frac{4z^2}{t^2} \}} zdxdy \\
        &= \iint_{\{0\leqslant z\leqslant t\leqslant 3\}} dzdt z \pi \cdot \frac{4z^2}{z^2}\\   
        &= 4\pi \iint_{\{ 0\leqslant z \leqslant t \leqslant 3\}} \frac{z^3}{t^2}dzdt \\
        &= 4\pi \int_0^3 \frac{1}{t^2} dt \int_0^t z^3dz = \frac{4\pi}{4}\left( \int_0^3 t^2dt \right) = \pi\cdot 9  \\
    .\end{align*}


\end{example}

\section{Мера Лебега--Стилтьеса}

$\sqsupset g(x)\uparrow$ на $\R$ и непрерывна слева $\left(\lim_{x \to x_0-0} g(x) \equiv g(x_0)\right)$

\begin{problem}
    Если $h(x)$ -- произвольная возрастающая функция, то её можно превратить в непрерывную слева исправлением нбчс количества точек.

    $\exists \uparrow$ и непрерывная слева $g(x) = h(x)$ всюду кроме точек разрыва $h(x)$
    
    $g(x_0) = \lim_{x \to x_0-0}h(x) $
\end{problem}

Определим $\mu_g ([a,b]) = g(b) - g(a) \geqslant 0 $. Так же верно, что $\mu_g$ обладает счетной аддитивностью на $\mathcal P_1$ (доказывается так же, как в случае с мерой Лебега) 
$\implies \mu_g $ -- мера на $\mathcal P_1$

Стандартное продолжение $\mu_g$, которое также обозначается $\mu_g$ называется мерой Лебега-Стилтьеса, порождённой функцией $g$

\begin{align*}
    \mu_g\left( \left\{ c \right\}  \right) &= \mu_g\left( \bigcap\limits_{j=1}^{\infty }[c, c + \frac{1}{j}]  \right)  \\
    &= \lim_{j \to \infty} \mu_g\left( [c, c + \frac{1}{j}] \right)  \\
    &= \underbrace{\lim_{j \to \infty} g(c + \frac{1}{j})}_{ = g(c + 0)} - g(c) = g(c+0) - g(c)  \\
.\end{align*}

$\implies $ Если $c$ -- точка непрерывности, то $\mu_g(\{c\}) = 0$

$\mu_g ([a, b]) = \mu_g([a, b]) + \mu_g(\{b \}) = g(b) - g(a) +g(b + 0) - g(b) = \left( g(b + 0) - g(a - 0)\right)$
    
$\mu_g\left( \left( a, b \right)  \right) = \mu_g\left( [a,b)] \right)  - \mu(\{a\}) = g(b) - g(a) - (g(a + 0) - g(a)) = g(b) - g(a+0) $

$\mu_g\left( \left( a, b \right] \right) = g(b + 0 ) - g(a + 0)$

\begin{definition}
    Пусть $\mu = \sum\limits_{k=1}^\infty h_k \delta_{a_k}$,~~~
    $h_k \geqslant 0$, ~~~ $\delta_a(E) = \begin{cases}
        1,~~~ a \in E\\
        0,~~~ a \not\in E
    \end{cases}$,~~~ тогда $\mu$ --- дискретная мера.

\end{definition}

$E, E_j\in 2^{\R}\quad E = \bigvee\limits_{j=1}^{\infty }E_j \implies \delta_{a_k}(E) = \sum_{j=1}^{\infty } \delta_{a_k}(E_j)$

\begin{align*}
    \mu(E)  &= \mu(\bigvee_{j=1}^{\infty }E_j) = \sum_k\sum_j h_k\delta_{a_k}(E_j)\\
    &= \sum_j \mu(E_j)\\
.\end{align*}

Последний переход в равенстве по теореме Тонелли.

\begin{note}
    $\sqsupset \{a_k\}_{k=1}^{\infty }\subseteq \R$

    $\forall [a,b]\qquad \sum_{k: a_k\in [a,b]}h_k < + \infty $
\end{note}

\begin{example}
    Если $\{a_k\}$ -- дискретно (без точек сгущения на $\R$), то условие автоматически выполняется, т.к. перечесечения $a_k$-ых с промежутком будет конечно,
    а значит и сама сумма будет конечна

    $A = \Q\quad h_k = \dfrac{1}{2^k}$
\end{example}

\begin{definition}
    [функция Хэвисайда]
    \[\Theta(x) = \begin{cases}
        0&, x\leqslant 0\\
        1&, x>0
    \end{cases}\]

    $\sqsupset x_0\in \R\quad \forall C\in \R$

    \[g(x) = \sum_{k=1}^{\infty } h_k\cdot \left(\Theta(x - a_k) - \Theta(x_0 - a_k)\right) + C\]

    \begin{enumerate}
        \item $g(x)$ возрастает
        \item $x\in [a,b]\quad \sum_k h_k(\Theta(x - a_k) - \Theta(x_0 - a_k)) \leqslant \sum_{a_k:I_{x, x_0}} h_k$

        Разность Тет ненулевая, если $a_k$ находится между $x$ и $x_0$ -- $I_{x, x_0}$
    \end{enumerate}
\end{definition}

\begin{statement}
    $A = \{a_k\}_k$
    \begin{enumerate}
        \item $g \in C\left( \R \setminus A \right) $
        \item Непрерывность слева на $A$
    \end{enumerate}
\end{statement}
\begin{proof}
\begin{enumerate}
    \item $\sqsupset x\in \R \setminus A\quad \sqsupset (a,b)\ni x$

    $\sphericalangle \forall \varepsilon >0 \quad \sum_{k: a_k\in [a,b] } h_k < +\infty  \implies \exists K: \sum _{\substack{a_k\in [a, b]\\ k \geqslant K }} \leqslant \dfrac{\varepsilon}{2}$.
    
    $g_k (x) = h_k \left(  \Theta (x- a_k) - \Theta(x_0 - a_k)\right)$ --- локально постоянны в точке $x$ ($\exists V_{\delta}(x)~:~ g_k\mid_{V_{\delta}(x)}\equiv const$ для $k = 1, \ldots, k$)

    Не умаляя общности $[a,b] \supseteq V_{\delta}(x) $

    \begin{align*}
    g(\tl x) - g(x) &= \sum_{k=1}^{\infty } h_k\left( \Theta(x - a_k) - \Theta(x - a_k) \right) - \sum_{k=1}^{\infty }h_k \left( \Theta(\tl x - a_k) - \Theta(x_0 - a_k) \right)\\ 
    &= \sum_{k=1}^{\infty } h_k \left( \Theta(x - a_k) - \Theta(\tl x - a_k) \right)  \\
    &= \underbrace{\sum_{k=1}^{K} h_k \left( \Theta(x - a_k) - \Theta(\tl x - a_k) \right)}_{ = 0} + \underbrace{\sum_{k=K+1}^{\infty } h_k \left( \Theta(x - a_k) - \Theta(\tl x - a_k) \right)}_{ = \dfrac{\varepsilon}{2}}  \\
    .\end{align*}

    $\implies $ Непрерываность

    Если $x = a_k\quad g(x) = g_{k_0}(x) + \underbrace{\sum _{k \neq k_0}g_k}_{\text{непрерывна как в пред. случае}}$
\end{enumerate}
\end{proof}

\begin{align*}
    \mu_g\left([a,b)]  \right) &= g(b) - g(a)\\ 
    &= \sum_{k=1}^{\infty} h_k\left( \Theta(b - a_k) - \Theta(a - a_k) \right)\qquad a \leqslant  a_k \leqslant b\\
    &= \sum_{k: a\leqslant a_k < b} h_k = \mu([a,b)])\\
.\end{align*}

$\mu$ и $\mu_g$ совпадают на совокупности всевозможных промежутков.


\begin{definition}
    Пусть $f: \R \to \R$.

    Функция $f$ называется локально суммируемой  на $\R$
    $\iff \forall \, [a,b]\quad\quad f\bigg|_{[a,b]} \in \mathcal{L} (\lambda_1)$. 
\end{definition}

\begin{definition}
    $f: \R \to \R$.
    
    Функция $f$ называется абсолютно непрерывной, если существует локально суммируемая функция $h(x)$ и точка $x_0\in \R$: \[g(x) = \int_{x_0}^x h(x)d\lambda\] 
    (интеграл Лебега. Если $x < x_0$, то $\int_{x_0}^x h\, d\lambda = - \int_{[x, x_0]}h\, d\lambda$)
    
    Если $h$ непрерывна в точке $x$, то $g(x)$ дифференцируема в точке $x$ и $\p g(x) = h(x)$. Доказательство  -- смотри теорему Барроу\ldots
    
    Если $h(x) \geqslant 0$, то $g(x)\nearrow $

    Функция $g(x)$ непрерывна на $\R$. Следует из абсолютной непрерывности интеграла.

\end{definition}
\endinput

\begin{theorem}
    [воспоминание]

    \[\mu(E) = \int_{\Phi^{-1}} h d\mu \iff \forall E\in \mathscr{A} \quad \inf_E h \mu(E) \leqslant \nu(E) \leqslant \sup_E h \mu(E)\]
\end{theorem}

\begin{note}
    \[g(x) = \sum_{k=1}^{\infty } h_k \left( \Theta(x - a_k) - \Theta(x_0 - a_k) \right) \]

    Для этой меру нужно было фиксировать открытый интервал $\Delta$, что \[ \forall [a,b] \subseteq \Delta\quad \sum_{k: a_k\in [a,b]}h_k < +\infty  \]

    \begin{align*}
       g(a_k + 0) - g(a_k - 0) &= h_k\left( \Theta(a_k - a_k + 0) - \Theta(x_0 - a_k + 0) - \Theta(a_k-a_k - 0) + \Theta(x_0 -a_k - 0) \right)  \\ 
       &= h_k \\ %?
    .\end{align*}
\end{note}

\begin{statement}    
Если $\nu = \sum\limits_k h_k \delta_{a_k}$, то $\nu$ совпадает с $\mu_g$ на $\mathscr{A}_{\mu_g}$ при условии (*).
\end{statement}
\begin{proof}

    Если хочется скорее сослаться на теорему об единственности, то можно сделать так:    
    Рассмотрим $[a, b)$. $\nu([a, b)) = \sum\limits_{k \,:\, a_k \in [a, b)} h_k.$

    \[ \mu_g([a, b)) = g(b) - g(a) = \sum_{k\in\N} h_k \left(\Theta(b - a_k) - \Theta(a - a_k)\right) = \sum_{k \,:\, a_k \in [a, b)} h_k.\]

    Если $\{ a_k\}_k$ --- конечное множество, то вопросов с суммируемостью не веознкает.

    \[g(x) = \sum_k h_k \cdot \Theta(x - a_k) + C   \]
\end{proof}

\begin{note}
    Локально суммируемая функция -- это такая, что она будет на любом шаре суммируемой по Лебегу
\end{note}

\begin{theorem}
    $g(x)$ -- абслолютно непрерывная $\iff \exists h\in \mathscr L_{loc}\left( \R, \lambda \right) \exists x_0\in \R, c\in \R$ 
    \[(x) = \int_{x_0}^x h(x) d\lambda + C\]
\end{theorem}  

По теореме Барроу $g(x)$:
\begin{itemize}
    \item $g(x)\in C(\R)$,
    \item $g(x)$ дифференцируема в точках ... функции $h(x)$. 
\end{itemize}

\begin{proof}
\begin{itemize}
    \item  Если $x_1\in \R$ \[g(x) - g(x_1) = \int_{x_1}^x h(x)dx\]

$\exists \delta_0 >0, x\in V_{\delta_0}(x_1), \quad h\in \mathscr L\left( V_{\delta_0} \right) $

$\forall \varepsilon >0 \exists \delta (\leqslant \delta_0 ) >0 : \int_E h(x)d\lambda  < \varepsilon \forall E\subseteq V_{\delta_0}(x_1):\, \lambda_1(E) <\delta$

$\implies $ Если  $\left| x_1 - x \right| <\delta\quad \left| \int_{x_1}^x h(x)dx \right| \leqslant \varepsilon$

\item Пусть $x_1$ --- точка непрерывности для $h(x)$. $h(x) = h(x_1) + \underbrace{ \alpha (x - x_1)}_{o(1) \text{ при } x\to x_1 }$

\[ \dfrac{g(x) - g(x_1)}{x - x_1} = \frac{1}{x - x_1} \int_{x_1}^x h(x_1) + \alpha(x-x_1)dx = h(x_1) + \frac{1}{x - x_1} \int_{x_1}^x \alpha(x - x_1)dx \leqslant \varepsilon(x - x_1)\]

Если ``$x$ остаточно близок к $x_1$''
\end{itemize}
\end{proof}

\begin{note}
    В частности, если $h(x) \in C(\R) \implies g\in C^1(\R)$ и $\p g(x)\equiv h(x)$    
\end{note}

\begin{note}
    \[\int_E fd\nu = \sum_{k\,:\,a_k\in E} h_k f(a_k) = \sum_{k\,:\,a_k\in E} f(a_k) \cdot \text{ скачок } g(a_k)\]  
\end{note}

\begin{statement}
    $\sqsupset g(x) = \int_{x_0}^x h(x)d\lambda_1(x) +C\quad h(x) \geqslant 0\quad h\in \mathscr{L}_{loc}\left(  \R, \lambda\right) $     абсолютно непрерывная возрастающая функция.
  
    Тогда $\int_E f d \mu_g = \int_E f(x) h(x) d \lambda (x)$.

    В частности, $\forall$~возрастающей $g(x) \in C^{1} (\R)$.
    \[\int_E f d\mu_g = \int_E f \cdot \p g (x) d \lambda(x) \left( = \int_E f \cdot dg \right). \]
\end{statement}

\begin{proof}
    $\sphericalangle \nu (E) = \int_E h d \lambda_1$. 
    \[\mu_g (\langle a, b \rangle )= \mu_g ([a, b)) = g(b) - g(a) = \int_a^b h(x) d \lambda_1 = \nu([a, b)) = \nu(\langle a, b \rangle ). \]

    $\mu_g$ и $\nu$ совпадают на открытых. Если $K$ -- компакт, $K = B \setminus \left( B \setminus K  \right) $

    $\nu(K) + + \nu(B \setminus K) = \nu(B)\qquad \mu_g(K) = \nu(K) = \nu(B) - \nu(B \setminus K)$

    $\sqsupset E$ -- $\lambda_1$-мера $O$

    $ \implies \exists \delta >0 \exists $ открытое $G: E\subseteq G$ и $:\lambda_1(G) <\delta$
    
    
    $\implies \int\mid_{G_0}$ -- абсолютно непрерывное $\implies \forall \varepsilon>0 \exists \delta>0: \lambda_1(\tl E)<\delta\quad \tl E \subseteq G$
    
    $\int_{\tl E}h < \varepsilon\quad \tl E = G \implies \nu(G) <\varepsilon \implies \mu_g(G) <\varepsilon\,\varepsilon\text{ --- } \forall  \implies \nu(E) = \mu_g(E) = 0$   

    Если $E$~--- неограничено $\lambda_1$--меры 0 $\implies \exists~$ ограниченное $E_j\,:\,E=\bigcup E_j$.
    $\forall i \in \N ~ \lambda_1 (E_j) = 0 \implies \nu(E_j) =\mu_g (E_j) = 0$ $\implies \nu(E) = \mu_g (E).$

    Дальше можно применить теорему о плотности меры. Применяю общую мхему замены переменной все доказывается. 
\end{proof}
\begin{problem}
    \begin{enumerate}
        \item $g(x) = \arctg x$. Найти:
        \begin{enumerate}
            \item $\sup \Bigg\{ \mu_g(I)~:~ I = \langle a, b \rangle,~ \lambda_1 (I) \leqslant \delta \Bigg\},~ \delta > 0$.
            \item $\sup \Bigg\{ \lambda_1(I)~:~ I = \langle a, b \rangle,~ \mu_g (I) \leqslant \delta \Bigg\},~ \delta > 0$.
        \end{enumerate}
        \item $g(x) = \arctg x + \Theta(x - 1)$
        \begin{enumerate}
            \item Для $\delta = 1$
        \end{enumerate}  
    \end{enumerate}
\end{problem}
\begin{proof}
    [Решение]

    $\mu_g(I) = g(b) -g(a) = \int_I \p g(t)dt = \int_{[a,b]}\frac{dt}{1 + t^2}$

    \begin{enumerate}
        \item 
        \begin{enumerate}
            \item \begin{align*}    
                \sup \{\mu_g(I)\} &= 2 \int_0^{\frac{\pi}{2}} \frac{dt}{1 + t ^2}\\
            .\end{align*}
        \end{enumerate}
    \end{enumerate}
\end{proof}

\begin{example}
    Пример меры Лебега--Стилтьеса 
    не евклидовой, не дискретной, не абсолютно непрерывной:

    \begin{align*}
        C_0 &= \left[0,1\right]\\
        C_1 &= \left[0, \frac{1}{3}\right] \cup \left[\frac{2}{3}, 1\right] \\
        C_2 &= \left[0, \frac{1}{9}\right] \cup \left[\frac{2}{9}, \frac{1}{3}\right] \cup \left[\frac{2}{3}, \frac{7}{9}\right] \cup \left[\frac{8}{9}, 1\right]  \\
        C_{k+1} &\subseteq C_k\quad C_k \text{ -- компакт}\\
        C &= \bigcap\limits_{k=1}^{\infty}C_k \text{ -- компакт}  \\
        \lambda_1(C) &= \lambda_1([0,1]) - \frac{1}{3} - \frac{2}{9}  - \ldots - \frac{2^{k-1}}{3^k} = 0\\
    .\end{align*}  

    $\psi(x) = \frac{1}{3}x\quad \Theta(x) = 1-x$
    
    \[\Phi = \left\{ [0,1] \cap C, \psi(C), \Theta\psi(C), \psi \psi(C), \psi\Theta(C), \Theta\psi\psi(C) , \Theta\psi\Theta\psi(C), \ldots \right\} \] 
    -- полукольцо

    $\mu(C) = 1\quad \mu(P) = \frac{1}{2^k}$ -- если $P$ есть результат применения $k$ штук $\psi$ и $\Theta$

    $\sphericalangle \mu$ -- стандартное продолжение
\end{example}
\end{document}