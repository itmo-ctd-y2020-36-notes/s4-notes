
\begin{theorem}
    [воспоминание]

    \[\mu(E) = \int_{\Phi^{-1}} h d\mu \iff \forall E\in \mathscr{A} \quad \inf_E h \mu(E) \leqslant \nu(E) \leqslant \sup_E h \mu(E)\]
\end{theorem}

\begin{note}
    \[g(x) = \sum_{k=1}^{\infty } h_k \left( \Theta(x - a_k) - \Theta(x_0 - a_k) \right) \]

    Для этой меру нужно было фиксировать открытый интервал $\Delta$, что \[ \forall [a,b] \subseteq \Delta\quad \sum_{k: a_k\in [a,b]}h_k < +\infty  \]

    \begin{align*}
       g(a_k + 0) - g(a_k - 0) &= h_k\left( \Theta(a_k - a_k + 0) - \Theta(x_0 - a_k + 0) - \Theta(a_k-a_k - 0) + \Theta(x_0 -a_k - 0) \right)  \\ 
       &= h_k \\ %?
    .\end{align*}
\end{note}

\begin{statement}    
Если $\nu = \sum\limits_k h_k \delta_{a_k}$, то $\nu$ совпадает с $\mu_g$ на $\mathscr{A}_{\mu_g}$ при условии (*).
\end{statement}
\begin{proof}

    Если хочется скорее сослаться на теорему об единственности, то можно сделать так:    
    Рассмотрим $[a, b)$. $\nu([a, b)) = \sum\limits_{k \,:\, a_k \in [a, b)} h_k.$

    \[ \mu_g([a, b)) = g(b) - g(a) = \sum_{k\in\N} h_k \left(\Theta(b - a_k) - \Theta(a - a_k)\right) = \sum_{k \,:\, a_k \in [a, b)} h_k.\]

    Если $\{ a_k\}_k$ --- конечное множество, то вопросов с суммируемостью не веознкает.

    \[g(x) = \sum_k h_k \cdot \Theta(x - a_k) + C   \]
\end{proof}

\begin{note}
    Локально суммируемая функция -- это такая, что она будет на любом шаре суммируемой по Лебегу
\end{note}

\begin{theorem}
    $g(x)$ -- абслолютно непрерывная $\iff \exists h\in \mathscr L_{loc}\left( \R, \lambda \right) \exists x_0\in \R, c\in \R$ 
    \[(x) = \int_{x_0}^x h(x) d\lambda + C\]
\end{theorem}  

По теореме Барроу $g(x)$:
\begin{itemize}
    \item $g(x)\in C(\R)$,
    \item $g(x)$ дифференцируема в точках ... функции $h(x)$. 
\end{itemize}

\begin{proof}
\begin{itemize}
    \item  Если $x_1\in \R$ \[g(x) - g(x_1) = \int_{x_1}^x h(x)dx\]

$\exists \delta_0 >0, x\in V_{\delta_0}(x_1), \quad h\in \mathscr L\left( V_{\delta_0} \right) $

$\forall \varepsilon >0 \exists \delta (\leqslant \delta_0 ) >0 : \int_E h(x)d\lambda  < \varepsilon \forall E\subseteq V_{\delta_0}(x_1):\, \lambda_1(E) <\delta$

$\implies $ Если  $\left| x_1 - x \right| <\delta\quad \left| \int_{x_1}^x h(x)dx \right| \leqslant \varepsilon$

\item Пусть $x_1$ --- точка непрерывности для $h(x)$. $h(x) = h(x_1) + \underbrace{ \alpha (x - x_1)}_{o(1) \text{ при } x\to x_1 }$

\[ \dfrac{g(x) - g(x_1)}{x - x_1} = \frac{1}{x - x_1} \int_{x_1}^x h(x_1) + \alpha(x-x_1)dx = h(x_1) + \frac{1}{x - x_1} \int_{x_1}^x \alpha(x - x_1)dx \leqslant \varepsilon(x - x_1)\]

Если ``$x$ остаточно близок к $x_1$''
\end{itemize}
\end{proof}

\begin{note}
    В частности, если $h(x) \in C(\R) \implies g\in C^1(\R)$ и $\p g(x)\equiv h(x)$    
\end{note}

\begin{note}
    \[\int_E fd\nu = \sum_{k\,:\,a_k\in E} h_k f(a_k) = \sum_{k\,:\,a_k\in E} f(a_k) \cdot \text{ скачок } g(a_k)\]  
\end{note}

\begin{statement}
    $\sqsupset g(x) = \int_{x_0}^x h(x)d\lambda_1(x) +C\quad h(x) \geqslant 0\quad h\in \mathscr{L}_{loc}\left(  \R, \lambda\right) $     абсолютно непрерывная возрастающая функция.
  
    Тогда $\int_E f d \mu_g = \int_E f(x) h(x) d \lambda (x)$.

    В частности, $\forall$~возрастающей $g(x) \in C^{1} (\R)$.
    \[\int_E f d\mu_g = \int_E f \cdot \p g (x) d \lambda(x) \left( = \int_E f \cdot dg \right). \]
\end{statement}

\begin{proof}
    $\sphericalangle \nu (E) = \int_E h d \lambda_1$. 
    \[\mu_g (\langle a, b \rangle )= \mu_g ([a, b)) = g(b) - g(a) = \int_a^b h(x) d \lambda_1 = \nu([a, b)) = \nu(\langle a, b \rangle ). \]

    $\mu_g$ и $\nu$ совпадают на открытых. Если $K$ -- компакт, $K = B \setminus \left( B \setminus K  \right) $

    $\nu(K) + + \nu(B \setminus K) = \nu(B)\qquad \mu_g(K) = \nu(K) = \nu(B) - \nu(B \setminus K)$

    $\sqsupset E$ -- $\lambda_1$-мера $O$

    $ \implies \exists \delta >0 \exists $ открытое $G: E\subseteq G$ и $:\lambda_1(G) <\delta$
    
    
    $\implies \int\mid_{G_0}$ -- абсолютно непрерывное $\implies \forall \varepsilon>0 \exists \delta>0: \lambda_1(\tl E)<\delta\quad \tl E \subseteq G$
    
    $\int_{\tl E}h < \varepsilon\quad \tl E = G \implies \nu(G) <\varepsilon \implies \mu_g(G) <\varepsilon\,\varepsilon\text{ --- } \forall  \implies \nu(E) = \mu_g(E) = 0$   

    Если $E$~--- неограничено $\lambda_1$--меры 0 $\implies \exists~$ ограниченное $E_j\,:\,E=\bigcup E_j$.
    $\forall i \in \N ~ \lambda_1 (E_j) = 0 \implies \nu(E_j) =\mu_g (E_j) = 0$ $\implies \nu(E) = \mu_g (E).$

    Дальше можно применить теорему о плотности меры. Применяю общую мхему замены переменной все доказывается. 
\end{proof}
\begin{problem}
    \begin{enumerate}
        \item $g(x) = \arctg x$. Найти:
        \begin{enumerate}
            \item $\sup \Bigg\{ \mu_g(I)~:~ I = \langle a, b \rangle,~ \lambda_1 (I) \leqslant \delta \Bigg\},~ \delta > 0$.
            \item $\sup \Bigg\{ \lambda_1(I)~:~ I = \langle a, b \rangle,~ \mu_g (I) \leqslant \delta \Bigg\},~ \delta > 0$.
        \end{enumerate}
        \item $g(x) = \arctg x + \Theta(x - 1)$
        \begin{enumerate}
            \item Для $\delta = 1$
        \end{enumerate}  
    \end{enumerate}
\end{problem}
\begin{proof}
    [Решение]

    $\mu_g(I) = g(b) -g(a) = \int_I \p g(t)dt = \int_{[a,b]}\frac{dt}{1 + t^2}$

    \begin{enumerate}
        \item 
        \begin{enumerate}
            \item \begin{align*}    
                \sup \{\mu_g(I)\} &= 2 \int_0^{\frac{\pi}{2}} \frac{dt}{1 + t ^2}\\
            .\end{align*}
        \end{enumerate}
    \end{enumerate}
\end{proof}

\begin{example}
    Пример меры Лебега--Стилтьеса 
    не евклидовой, не дискретной, не абсолютно непрерывной:

    \begin{align*}
        C_0 &= \left[0,1\right]\\
        C_1 &= \left[0, \frac{1}{3}\right] \cup \left[\frac{2}{3}, 1\right] \\
        C_2 &= \left[0, \frac{1}{9}\right] \cup \left[\frac{2}{9}, \frac{1}{3}\right] \cup \left[\frac{2}{3}, \frac{7}{9}\right] \cup \left[\frac{8}{9}, 1\right]  \\
        C_{k+1} &\subseteq C_k\quad C_k \text{ -- компакт}\\
        C &= \bigcap\limits_{k=1}^{\infty}C_k \text{ -- компакт}  \\
        \lambda_1(C) &= \lambda_1([0,1]) - \frac{1}{3} - \frac{2}{9}  - \ldots - \frac{2^{k-1}}{3^k} = 0\\
    .\end{align*}  

    $\psi(x) = \frac{1}{3}x\quad \Theta(x) = 1-x$
    
    \[\Phi = \left\{ [0,1] \cap C, \psi(C), \Theta\psi(C), \psi \psi(C), \psi\Theta(C), \Theta\psi\psi(C) , \Theta\psi\Theta\psi(C), \ldots \right\} \] 
    -- полукольцо

    $\mu(C) = 1\quad \mu(P) = \frac{1}{2^k}$ -- если $P$ есть результат применения $k$ штук $\psi$ и $\Theta$

    $\sphericalangle \mu$ -- стандартное продолжение
\end{example}

\section{Интегралы, зависящие от параметра}
\begin{example}
    \[\Gamma(p) = \int_0^{+\infty} x^{p-1} e ^{-x} dx, ~~~
    p > 0, p \in \R;~~~
    \int_a^b f(x, y) dx,~~~
    \int_\alpha(y)^\beta(y) f(x, y) dx. \]
\end{example}

Пока что мы будем рассматривать интегралы, зависящие от параметра $y$ 
по фиксированному промежутку: $I(y) = \int_X f(x, y) d\mu (x)$.

Пусть у нас есть пространство с мерой $(X, \mathscr{A}, \mu)$, $f(\dot, \mu) \in \mathcal{L}(X, \mu)$.
$Y \subseteq \overline{Y}$.

Для чего это нужно? Бывает, что просто сформулированные задачи имеют ответ в виде интеграла с параметром. 
Бывает, что введение параметра упрощает вычисление интеграла.

\begin{statement}
    $f$ уовлетворяет условию Лебега локально относительно $y_0$, $y_0$~--- параметр,
    если $\exists \,$ открытое $V(y_0)$ в $\overline{Y}$ и $\Phi_{(x)} \in \mathcal{L}(X, \mu)$
    $\forall y \in V(y_0)$ для почти всех $x \in X$. 
\end{statement}

\begin{statement}
    Пусть у нас есть пространство с мерой $(X, \mathscr{A}, \mu)$, 
    $\overline{Y}$~--- метрическое пространство, $Y \subseteq \overline{Y}$, 
    $ y_0$~--- предельная точка для $Y$.
    Почти везде $f(x,y ) \to g(x)$ при $y \to y_0$, и $f(x, y) $ удовлетворяет локаольно условию Лебега относительно $y_0$. 

    Тогда $g(x) \in \mathscr{L}(X, \mu)$ и
    \[\lim_{y \to y_0} \int_X f(x, y) d\mu(x) = \int_X g(x) d\mu(x)\]  
    % \[\lim \to \infty}   \]
\end{statement}
\begin{proof}
    Так как $y_0$ --- предельная, $\exists \{y_k\}\subseteq Y \to y_0$.
    $f_k(x) = f(x, y_k)$, $y_k \in V(y_0) \implies |f_k(x) | \leqslant \Phi(x)$
    $\implies$ по теореме Лебега о мажорируемой сходимости,
    $g(x) = \lim_{k\to\infty} f(x, y_k) \in \mathcal{L}(X, \mu)$ и
    \[\int_X g(x) d\mu =\int_X \lim_{k\to\infty} f(x, y_k) d\mu = \lim_{k\to\infty} \int_X f(x, y_k) d\mu . \]

    \[I(y) = \int_X f(x, y) d\mu(x);~~ \lim_{k\to\infty} I(y_k) ~\forall ~ \text{последовательности } y_k \to y_o \implies \exists \lim_{y\to y_0} Y(y).\]
\end{proof}
% todo example % ~done
\begin{example}
    $\sqsupset p_0 >0\quad \sqsupset $

    $\forall p\in V_{\delta}(p)$
    
    $x\in \left( 0, 1 \right]\quad x^{p-1}e^{-x} \leqslant x^{p_0 - \delta}e^{-x}$

    $x > 1\quad x^{p-1}e^{-x} \leqslant x^{p_0 + \delta}e^{-x}$ 

    $\Phi(x) = \begin{cases}
        x^{p_0 - \delta}e^{-x}&, x\in (0,1]\\
        x^{p_0 + \delta}e^{-x}&, x > 1\\
    \end{cases}\quad \int_0^{+\infty }x^qe^{-x}dx$ -- сходится для любого 
\end{example}

\begin{note}
    Если в условиях предыдущего утверждения $f(x, y)$~--- непрерывна по $y$ в точке $y_0$, 
    то наш интеграл $I(y)$ тоже будет непрерывен в точке $y_0$. 
\end{note}

\begin{definition}
    Пусть имеется пространство с мерой $(X, \mathscr{A}, \mu)$, $y_0$~--- предельная точка для $Y\subseteq \overline{Y}$
    $f(x,y ) \rightrightarrows g(x)$ на $X$ при $y\to y_0$ если 
    $\forall \varepsilon > 0 \, \exists $ окрестность $V(y_0)$\,:
    \[ \forall x \in X~~~ \forall y \in V(y_0) ~~~ |f(x,y) -g (x) | < \varepsilon \iff \sup_{x\in X} |f(x, y) - g(x) | \underset{y\to y_0}{\to} 0. \]
\end{definition}

\begin{example}
    \begin{enumerate}
        \item (хороший) $f(x, y) = \frac{\sin(x^2 + y^2)}{1+ x^2 + y^2}\quad y \to +\infty $
        
        \[|f(x, y) | \leqslant \dfrac{1}{1 + y^n}\implies y\to \infty \sup |f(x, y) | = \dfrac{1}{1 + y^n} \underset{y\to\infty} \to 0. \]
        Сходимость есть и равномерная сходимость тоже есть.
        
        \item (плохой) $x y e^{-xy} \underset{y\to 0}{\to} 0$.
        Сходимость к нулю есть, а
        \[ \sup{x> 0} x y e^{-xy} \geqslant f(\dfrac{1}{y}, y) = \dfrac{1}{e} \not \to 0 \implies \text{равномерно не сходится}. \]
    \end{enumerate}
\end{example}

\begin{statement}
    Пусть $(X, \mathscr{A}, \mu), ~~ \mu(X) < +\infty$.
    \[f(x,y ) \underset{y\to y_0} {\rightrightarrows} g(x),~~~ f(x, y) \in \mathcal{L}(X, \mu). \]

    Тогда $g(x)\in \mathscr{L}(X, \mu)$ И 
    \[\lim_{y \to y_0}\int_X f(x, y)d\mu(x)  = \int_X g(x)d\mu(x)\]    
\end{statement}

\begin{proof}
    Для $\varepsilon  = 1 ~~~ \exists $ окрестность $V(y_0)\,:~ $
    $\forall x \in X, y \in V(y_0)~~ | f(x, y ) - g(x) | \leqslant 1 :$
    
    $|g(x)| \leqslant |f(x,y)| + |g(x) - f(x,y)| \leqslant  |f(x,y)| + 1 \implies g\in \mathscr{L}(X, \mu)$
\end{proof}

\begin{statement}
    $(X, \mathscr{A}, \mu)$~--- пространство с метрой
    $y \subseteq \R (\C), ~~~ y_0$~--- предельная точка для $Y$.
    Пусть $f(x, y),~~~ \p f_y $~--- удовлетворяет условию Липшица локально, $f: X \times Y \to \R (\C)$.

    Тогда $I(y)  = \int_X f(x, y) d\mu (x)$ дифференцируема в точке $y_0$ и
    \[ \p I (y_0) = \int_X \p f_y (x, y) d\mu (x).\]
\end{statement}

\begin{proof}
    \begin{align*}
        \p I (y_0) &= \lim_{y\to y_0} \dfrac{I(y) - I(y_0)}{y - u_0}\\ 
        &= \lim_{y \to y_0} \dfrac{1}{y - y_0 } \int _ X \underbrace{\left( f(x, y) - f(x, y_0) \right)}_{\p f _y (x, y_0 + \Theta (y - y_0)), ~ \Theta \in (0, 1)} d\mu(x)\\ 
        &= \lim \int_X \p f_y(x, \overset{C(y)}{y_0 + \Theta(y - y_0)}) d\mu(x)\\
        &\underset{\text{Утв}}= \int_X \lim(\ldots)d\mu(x) = \int_X \p f_y(x, y_0)d\mu(x).
    \end{align*}

    $y\in V_{\delta}(y_0)$ -- из условия Липшица для $\p f_y \implies C(y)\in V_{\delta}(y_0) $
    
    $\implies \left| \underbrace{\p f_y(x, C(y))}_{\p f_y(x, y_0)}  \right| \leqslant \Phi(x) $ 
\end{proof}

\begin{example}
    $\Gamma(p) \int_0^\infty x^{p - 1} e^{-x} d\mu$.

    \[\p f_p (x, p) = (p - 1) x^{p - 2} e^{-x}, ~p - 2 > -1 \implies  p > 1.\]

    При $p> 1$
    \[\p \Gamma (p) = (p - 1) \int_{0}^{+\infty} x^{p-2} e^{-x} = (p-1) \cdot \Gamma(p - 1) \implies 
    \p \Gamma (p) = (p - 1) \cdot \Gamma(p-1).\]

    \[ \Gamma(p) = \int_0^{+\infty} \dfrac{\p (x^p)}{p} = \dfrac{1}{p} \left( x^p e^{-x} \bigg|_o^{\infty} - \int_{o}^{+\infty} x^p \p(e^{-x}) d x \right) = \dfrac{1}{p} \cdot (p + 1). \]

    \begin{align*}
        \Gamma(1) &= \int_0^{+\infty }e^{-x} = 1\\
        \Gamma(2) &= 1\\
        \Gamma(3) &= 2\\ 
        \Gamma(n) = (n-1)!\quad n \in \N 
    .\end{align*}
\end{example}
% enought? 


