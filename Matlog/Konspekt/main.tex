\documentclass[10pt]{article}
\usepackage{preference}

\usepackage{mdframed}
\usepackage{stmaryrd}
\mdfsetup{skipabove=1em,skipbelow=0em}
\theoremstyle{definition}
\newmdtheoremenv[nobreak=true]{statement}{Утверждение}
\newtheorem*{note}{Замечание}
\newcommand\true{\text{\textit{И}}}
\newcommand\false{\text{\textit{Л}}}

%опа, нашёл как жить

%бтв, Костя, можешь вставить вот это в преамбулу того, где нужно писать код (скорее всего только алгосы)

\usepackage{listings}

\definecolor{codegreen}{rgb}{0,0.6,0}
\definecolor{codegray}{rgb}{0.5,0.5,0.5}
\definecolor{codepurple}{rgb}{0.58,0,0.82}
\definecolor{backcolour}{rgb}{0.95,0.95,0.92}

\lstdefinestyle{myStyle}{
    backgroundcolor=\color{backcolour},   
    commentstyle=\color{codegreen},
    keywordstyle=\color{magenta},
    numberstyle=\tiny\color{codegray},
    stringstyle=\color{codepurple},
    basicstyle=\ttfamily\footnotesize,
    breakatwhitespace=false,         
    breaklines=true,                 
    captionpos=b,                    
    keepspaces=true,                 
    numbers=left,                    
    numbersep=5pt,                  
    showspaces=false,                
    showstringspaces=false,
    showtabs=false,                  
    tabsize=2,
    extendedchars=\true
}

\lstset{style=myStyle}

\lstset{language=Python}
\begin{document}

\def\chap#1#2{\ \\ {\large\bf#1 \ | \ \tt\scshape#2} \par}

\ \vspace{-1cm}

{\bf
\ \\
\Large\centerline{\scshape Матлог, лекции}
}\normalsize

% % \section{Введение}

% Логика -- довольно старая наука, но наш предмет довольно молодой
% В какой-то момент логики как дисциплины, которая учит просто правильно рассуждать, стало не хватать.
% Появилась теория множеств.
% Общего здравого смысла не хватает, нужен строгий математический язык. 
% Это рубеж 19-20 веков.

% У нас теория множеств не будет фокусом, как это могло бы быть на мат. факультете.

% Теория множеств, когда она была впервые сформулирована, была противоречива (как матан, сформулированный Ньютоном).
% Чтобы уверенно и эффективно заниматься матаном, нужно суметь его формализовать. 

% <Парадокс Рассела / парадокс брадобрея>
% Мы приписываем элементу-человеку свойство, которое невыполнимо.
% Объекта, выходит, не существует.
% Мы смогли очень быстро определить противоречие в этом определении.
% Но, может быть, мы не смогли его определить в других наших определениях? 
% (конструкциях вещественной прямой, и т.д и т.д)

% Программа Гильберта.
% \begin{enumerate}
% \item Формализуем математику!
% Сформулируем теорию на языке (не на русском или английском), который не будет допускать парадоксов, 
% \item $\ldots$ и на котором можно будет доказать непротиворечивость. 
% \end{enumerate}

% В 1930 году становится понятно, что сколько-нибудь сильная (= в ней можно построить формальную арифметику) теория не может быть доказана непротиворечивой.

% Возможно, сама наша логика неправильная? 
% Эта идея будет нам полезна, и к ней мы ещё вернемся.

% Возможно, что это просто свойство мира, и мы хотим невозможного.

% Из этих рассуждений выросло большое множество хороших идей, которые оказались полезны в других местах.
% Матлогика служит широкому кругу нужд.

% Мы можем доказывать, что программа работает корректно. 
% Именно доказывать, а не проверять тестами!

% Мы можем изучать свойства самих языков.
% Изоморфизм Карри-Говарда--- доказательство это программа, утверждения это тип.
% Можно изучать языки программирования и можно развернуть изоморфизм: изучать математику как язык программирования. 

% Функциональные языки: окамль + хаскель. 
% Ознакомление с этими языками представляет собой способ ознакомиться с предметом немного с другой стороны.

\section{Исчисление высказываний}
Мы говорим на двух языках: на предметном языке и метаязыке.
Предметный язык~--- это то, \textbf{что} изучается, а метаязык~--- это язык, \textbf{на} котором это изучается.

На уроках английского предметным является сам английский, а метаязыком может быть русский.
Метаязык~--- это язык исследователя, а предметный язык~--- это язык исследуемого.
(Что вообще такое язык? Это отдельный вопрос.)

Высказывание~--- это одно из двух: 
\begin{enumerate}
\item Большая латинская буква начала алфавита, возможно с индексами и штрихами~--- это пропозициональные переменные.
\item Выражение вида $(\alpha \land \beta)$, $(\alpha \lor \beta)$, $(\alpha \to \beta)$, $(\neg \alpha)$. 
\end{enumerate}

В определении выше альфа и бета это метапеременные~--- места, куда можно подставить высказывание.
Они являются частью языка исследователя.
\begin{enumerate}
\item $\alpha, \beta, \gamma$ --- метапеременные для всех высказываний.
\item $X, Y, Z$ --- метапеременные для пропозициональных переменных.
\end{enumerate}

В формализации мы останавливаемся до места, в котором мы можем быть уверены, что сможем написать программу, которая всё проверяет.

Сокращение записи, приоритет операций: сначала $\neg$, потом $\&$, потом $\vee$, потом $\to$.
Если скобки опущены, мы восстанавливаем их по приоритетам.
Выражение без скобок является частью метаязыка, и становится частью предметного, когда мы восстанавливаем их.
Скобки последовательных импликаций расставляются по правилу правой ассоциативности~--- справа налево.
\subsection{Теория моделей}
У нас есть истинные значения $\{T, F\}$ в классической логике. 
И есть оценка высказываний $\llbracket \alpha\rrbracket$. 
Например $\llbracket A \lor \neg A\rrbracket$ истинно.
Всё, что касается истинности высказываний, касается теории моделей. 
\begin{definition}
    Оценка~--- это функция, сопоставляющая высказыванию его истинное (истинностное) значение.
\end{definition}
\subsection{Теория доказательств}
\begin{definition}
    Аксиомы~--- это список высказываний.
    Схема аксиомы~--- высказывание вместе с метапеременными; при любой подстановке высказываний вместо метапеременной получим аксиому. 
\end{definition}

\begin{definition}
    Доказательство (вывод)~--- последовательность высказываний $\gamma_1, \gamma_2\ldots$ где $\gamma_i$--- любая аксиома, 
    либо существуют $j,k < i$ такие что $\gamma_j$ имеет вид $(\gamma_k \to \gamma_i)$.
    Это правило ''перехода по следствию'' или Modus ponens.
\end{definition}

Определим следующие 10 схем аксиом для того исчисления высказываний, которое мы рассматриваем. 
\begin{enumerate}
    \item $\alpha \to \beta \to \alpha$~--- добавляет импликацию
    \item $(\alpha \to \beta) \to (\alpha \to \beta \to \gamma) \to (\alpha \to \gamma)$~--- удаляет импликацию
    \item $\alpha \land \beta \to \alpha$~--- удаление конъюнкции
    \item $\alpha \land \beta \to \beta$~--- удаление конъюнкции
    \item $\alpha \to \beta \to \alpha \land \beta$~--- внесение конъюнкции
    \item $\alpha \to \alpha \lor \beta$~--- внесение дизъюнкции
    \item $\beta \to \alpha \lor \beta$~--- внесение дизъюнкции
    \item $(\alpha \to \gamma) \to (\beta \to \gamma) \to (\alpha \lor \beta \to \gamma)$
    \item $(\alpha \to \beta) \to (\alpha \to \neg \beta) \to (\neg \alpha)$
    \item $\neg \neg \alpha \to \alpha$ --- очень спорная штука.
\end{enumerate}

\begin{example} Доказательство
    $\vdash A \to A $.
    \begin{enumerate}
        \item $A \to (A \to A) \to A$ (схема 1)
        \item $A \to A \to A$  (схема 1)
        \item $(\underbrace{A}_\alpha \to \underbrace{A \to A}_\beta) \to (\underbrace{A}_\alpha \to \underbrace{(A \to A)}_\beta \to \underbrace{A}_\gamma) \to (\underbrace{A}_\alpha \to \underbrace{A}_\gamma)$  (схема 2)
        \item $(A \to (A \to A) \to A) \to (A \to A)$ (m.p 2, 3)
        \item $A \to A$ (m.p 1, 4)
    \end{enumerate}
\end{example}
% \section{Теорема о дедукции}

\begin{definition}
    (Метаметаопределение).
    Будем большими греческими буквами $\Gamma, \Delta, \Sigma\ldots$ --- списки формул, неупорядоченные. 
\end{definition}

\begin{definition}
    Вывод из гипотез: $\Gamma \vdash \alpha$ (см. лекцию 1) 
\end{definition}

\begin{theorem}
    $ \Gamma, \alpha \vdash \beta$ тогда и только тогда, когда $\Gamma \vdash \alpha \to \gamma$. 
\end{theorem}

\begin{proof}
    \begin{itemize}
        \item [$\Leftarrow $] 
        Пусть $\delta_1, \delta_2\ldots \delta_n \equiv \alpha \to \beta$ выводит $\alpha \to \beta$. 
        Дополним этот вывод двумя новыми высказываниями: $\delta_{n+1} \equiv \alpha$ (дано нам в гипотезе), $\gamma_{n+2} \equiv \beta$ (MP шагов $n$, $n+1$) --- это и требовалось.
        \item Напишем программу, которая трансформирует один вывод в другой.
        Инвариант, который мы будем поддерживать: всё до $\alpha \to \delta_i $ --- док-во.
        Доказательство индукцией по $n$.
        \begin{enumerate}
            \item База: $n=1$ --- без комментариев.
            \item Если $\delta_1,\ldots, \gamma_n$ можно перестроить в доказательство $\alpha \to \gamma_n$, то $\gamma_1 \ldots \gamma_{n+1}$ тоже можно перестроить.
            Разберём случаи:
            \begin{enumerate}
                \item $\delta_i$ --- аксиома или гипотиза из $\Gamma$. \\
                ($i - 0.6$) $\delta_{i}$ \\
                ($i - 0.3$) $\delta_i \to \alpha \to \delta_i$\\
                ($i$) $\alpha \to \delta_i$
                \item $\alpha_i \to \delta$\\
                ($i - 0.8, i-0.6, i-0.4, i-0.2$) (доказательство $\alpha \to \alpha$) \\
                $(i)$ $\alpha\to\alpha$
                \item $\delta_i$ получено из $\delta_j$ и $\delta_k$\\
                ($j$) $\alpha \to $
                \dots
            \end{enumerate}
            ДОПОЛНИТЬ
        \end{enumerate}
    \end{itemize}
\end{proof}

\section{Теория моделей}
Мы можем докаывать модели или оценивать их.
''Мы можем доказать, что мост не развалится или можем выйти и попрыгать на нём.''

\begin{definition}
    $\mathbb{V}$~--- истинностное множество.

    $F$~--- множество высказываний нашего исчисления высказываний.

    $P$~--- множество пропозициональных переменных.
    \[ \llbracket\cdot\rrbracket: F \to \mathbb{V} \text{~--- оценка}\]
\end{definition}

\begin{definition}
    Для задания оценки необходимо задать оценку пропозициональных переменных.
    \[\llbracket \cdot \rrbracket : P\to \mathbb{V} \quad f_P\]
    Тогда: 
    \[ \llbracket x \rrbracket = f_p(x)\]
\end{definition}

\begin{remark}
    Обозначение: значения пропозициональных переменных будем определять в верхнем индексе: $\llbracket \alpha \rrbracket ^{A = T, B = F \ldots}$
\end{remark}


\begin{definition}
    $\alpha$ --- общезначна (истинна), если $\llbracket \alpha \rrbracket = T$ при любой оценке $P$.

    $\alpha$ --- невыполнима (ложна), если $\llbracket \alpha \rrbracket = F$ при любой оценке $P$.
    
    $\alpha$ --- выполнима, если $\llbracket \alpha \rrbracket = T$ при некоторой $f_P$.

    $\alpha$ --- опровержима, если $\llbracket \alpha \rrbracket = F$ при некоторой $f_P$.
\end{definition}

\begin{definition}
    Теория корректна, если доказуемость влечёт общезначимость.

    Теория полна, если общезначимость влечёт доказуемость.
\end{definition}

\begin{definition}
    $\Gamma \vDash \alpha$ означает, что $\alpha$ следует из $\Gamma = \{ \gamma_1, \ldots, \gamma_n\}$, если $\llbracket \alpha \rrbracket = T$ всегда при $\llbracket \gamma_i \rrbracket = T$ при любых $i$.  
\end{definition}

\begin{theorem}
    Исчисление высказываний корректно
    \[ \vdash \alpha\ \text{влечёт}\ \vDash \alpha. \]
\end{theorem}
\begin{proof}
    Индукция по длине доказательства. 
    
    Разбор случаев:
    \begin{enumerate}
        \item $\gamma_n$ аксиома $\implies$ построить таблицу истинности
        \item $\gamma_n$ --- м.п. $\gamma_j$, $\gamma_k$
    \end{enumerate}
\end{proof}

Мы даём доказательство на метаязыке, не пускаясь в отчаянный формализм.
Такая строгость нас устраевает.

В матлогике бесмысленно формализовывать русский язык.
Она нужна, чтобы дать ответы на сложные вопросы в математике, где здравого смысла недостаточно и нужна формализация.

\section{Полнота исчисления высказываний}
\begin{theorem}
    Исчисление высказываний полно.
\end{theorem}

\begin{definition}
    $_{[\beta]}\alpha = 
    \begin{cases}
        \alpha, & \llbracket \beta \rrbracket = T\\
        \neg \alpha,&  \llbracket \beta \rrbracket = F\\
    \end{cases}$
\end{definition}

\begin{lemma}
    $_{[\alpha]}\alpha,\ _{[\beta]}\beta \vdash _{[\alpha \star \beta]} \alpha \star \beta$ 

    $_{[\alpha]} \alpha \vdash _{[\neg \alpha]}\neg\alpha $
\end{lemma}

\begin{lemma}
Если $\Gamma \vdash \alpha$, то $\Gamma, \Delta \vdash \alpha$.
\end{lemma}

\begin{lemma} 
Пусть дана $\alpha$, $X_1,\ldots, X_n$ --- её переменные.   

\[ _{[X_1]}X_1,\ldots _{[X_n]}X_n\ \vdash _{[\alpha]}\alpha  \]
\end{lemma}
\begin{proof}
    Индукция по структуре. ДОПОЛНИТЬ
\end{proof}

Сократим запись и вместо этой кучи $X$ будем писать $X^\prime$.
\begin{lemma}
    Если $\vDash \alpha$, то $X^\prime \vdash \alpha$.
\end{lemma}

\begin{lemma}
    \[ \Gamma, Y \vdash \alpha, \quad \Gamma, \neg Y \vdash, \text{ то } \Gamma \vdash \alpha \]
\end{lemma}

\begin{theorem}
    Если $\vDash \alpha$, то $\vdash \alpha$.
\end{theorem}

\section{Интуиционистская логика}
Мы не хотим дурацких коснтрукций вроде парадокса брадобрея.
Мы не хотим странных, но логически верных утверждений вроде $A\to B \lor B \to A$.
Интуиционисткая логика предлагает свою математику, в которой своя интерпретация логических связок.
ВНК-интерпретация (Брауер-Гейтинг-Колмогоров).

\begin{itemize}\itemsep=-1mm
    \item $\alpha, \beta, \gamma \ldots$ --- это конструкции.
    \item $\alpha \land \beta$ если мы умеем строить и $\alpha$, и $\beta$.
    \item $\alpha \lor \beta$, если мы умеем строить $\alpha, \beta$ и знаем, что именно.
    \item $\alpha \to \beta$, если мы умеем перестроить $\alpha$ в $\beta$.
    \item $\bot $ --- не имеет построения
    \item $\neg \alpha \equiv \alpha \to \bot$   
\end{itemize}

''Теория доказательств''.
Рассмотрим классическое исчисление высказываний и заменим схему аксиом 10 на следующую 
\[ \alpha \to \neg \alpha \to \beta \] 

В этой формализации мы следуем не сути интуиционисткой логики, а традиции.
В интуиционисткой логике формализм это не источник логики.

Примеры моделей.
\begin{enumerate}
    \item Модели КИВ подходят: корректны, но не полны.
    \item Пусть $X$ топологическое пространство.
\end{enumerate}


\endinput
% \section{Общая топология}

Раньше были телевизоры с \textit{бесконченым} количеством пикселей (это зависит от химических свойств вещества кинескоп).

\begin{center}
    \includegraphics[scale=0.8]{parts/topology_tv_example}
\end{center}

Возьмем множество $X$. Определим на нем топологию как подмножество множества всех подмножеств
$\Omega \subseteq \mathcal{P}(X)$. $\Omega$ --- топология, если это множество открытых множеств и выполнены следующие условия:
\begin{enumerate}
    \item $\varnothing, X \in \Omega$;
    \item $\bigcup\limits_i \in \Omega$, если все $A_i \in \Omega$;
    \item $\bigcap\limits_{i = 1} ^ n A_i \in \Omega$, если $A_1, \dots, A_n \in \Omega$.
\end{enumerate}

То есть топологическое пространство~--- пара $\langle X, \Omega \rangle$ и про $\Omega$ верны приведенные выше три утверждения.

\begin{definition}
[Замкнутое мноежство] Множество $B$ такое, что $X \setminus B \in \Omega$ называется замкнутым.
\end{definition}

\begin{definition}
    [Связное топологическое пространство] $\langle X, \Omega \rangle$ связно, если нет $A, B \in \Omega ~:~ A \cup B = X$ и $A \cap B = \varnothing$
\end{definition}
    
\begin{definition}[Подпространство]
    $\langle X_1, \Omega_1 \rangle $ --- подпространство $\langle X, \Omega \rangle$, если
    $X_1 \subseteq X$ и $\Omega_1 = \{ a \cap X_1 ~|~ a \in \Omega$ \}  
\end{definition}

\begin{definition}
    [Связное множество]

    Множество, являющееся связным подпространством.
\end{definition}

\begin{center}
    \includegraphics[scale=0.6]{parts/topology_connectivy}
\end{center}

\subsection{Примеры топологических пространств}
Возьмем дерево (граф). Множество $X$~--- множество вержин. $\Omega$~--- множество всех вершин, что
$B \in \Omega$, \underline{если} $a \in B$, $x \leqslant a$ влечет $x \in B$. 
То есть $\Omega$ --- семейство множеств вершин, которые входят вместе с поддеревом.  

\begin{theorem}
    Граф без цикла свяен тогда и только тогда, когда оно своязно как топологическое пространство.
    \begin{proof}[Доказательство будет в дз]
    \end{proof}
\end{theorem}

\begin{definition}[Решетки]
    $X$~--- частично упорядоченное множество отношением $\leqslant$.\\
    Множество верхних граней $a, b$ --- множество $\{ x \in X ~|~ a \leqslant x, b \leqslant x\}$.\\
    Множество нижних граней $a, b$ --- $a \sqcup b$ --- множество $\{ x \in X ~|~ a \geqslant x, b \geqslant x\}$.\\
    $A$, наименьший элемент $A$ --- такой $a \in A$, что нет $b \in A$, $b \leqslant a$.\\
    $a+b$ = наименьший элемент множества верхних граний.\\
    $a*b$ = наибольший элемент множества нижних граний.

    Решетка --- частично упорядоченное множество, где для каждых двух элементов существуют $a+b$ и $a*b$.
\end{definition}

\begin{example}
    Дерево --- не решетка (в общем случае), так как $a + b$ есть, а $a*b$ может не быть.
\end{example}

\begin{theorem}
    Пусть $\langle X, \Omega \rangle$ топологическое пространство, $A, B \in \Omega$. $A \leqslant B$, если $A \subseteq B$.

    Тогда $\langle \Omega, \leqslant\rangle$ --- решетка. $A \cdot B = A \cap B, ~ A + B = A \cup B$.
\end{theorem}

\begin{definition}
    Дистрибутивная решетка --- это такая решетка, что $a,b,c \in \Omega$, ~$a + (b \cdot c) = (a + b) \cdot (a + c)$.
\end{definition}

\begin{lemma}
    Для дистрибутивной решетки так же верно, что $a \cdot (b + c) = (a \cdot b) + (a \cdot c)$.
\end{lemma}

\begin{definition}
    Псевдодополнение $a \to b = \text{наименьшее} \{ c ~|~ a \cdot c \leqslant b\} = b$.
\end{definition}

\begin{definition}
    Диамант --- такая решетка, что там нет для кого-то псевдодопллнения.

    \begin{center}
        \includegraphics[scale=0.8]{parts/topology_diomant}
    \end{center}
\end{definition}

\begin{definition}
    Решетка с псевдодополнением для всех элементов называется импликативной.
\end{definition}

\begin{definition}[0 и 1] .
    \begin{itemize}
        \item $0$ --- элемент, что $0 \leqslant x $ при всех $x$.
        \item $1$ --- элемент, что $x \leqslant 1 $ при всех $x$.
    \end{itemize}
\end{definition}

\begin{theorem}
    [В импликативной решетке 1 есть всегда] $X, \leqslant$ --- импликативная решетка.
    
    Рассмотрим $a \to a= \text{наиб} \{ c ~|~ q \cdot c \leqslant a\} = \text{наиб} X = 1$.
\end{theorem}

\begin{theorem}
    Рассмотрим $\langle X, \Omega \rangle$ --- импликативная решетка с $0$. Рассмотрим И.И.В.


    Определим оценки $\mathbb{V}  = X$:
    \begin{itemize}
        \item     $\llbracket \alpha \& \beta \rrbracket = \llbracket\alpha \rrbracket \cdot \llbracket\beta \rrbracket$.
        \item     $\llbracket \alpha \vee \beta \rrbracket = \llbracket\alpha \rrbracket + \llbracket\beta \rrbracket$.
        \item     $\llbracket \alpha \to \beta \rrbracket = \llbracket\alpha \rrbracket \to \llbracket\beta \rrbracket$.
        \item     $\llbracket \neg \alpha \rrbracket = \llbracket\alpha \rrbracket \to 0$.
    \end{itemize}

    $\alpha$ истинно, если $\llbracket \alpha \rrbracket = 1$.

    Полученная модель --- корректная модель И.И.В.
\end{theorem}

У нас будет натуральный вывод, интуиция и все такое.

$\overline{\Gamma, \varphi \vdash \varphi}$ (аксиома).

Вывод утверждения в доказательстве $\Gamma \vdash \varphi$.

\[
    \dfrac{\Gamma, \varphi \vdash \psi}{\Gamma \vdash \varphi \to \psi},~~~
    \dfrac{\Gamma, \varphi \vdash \psi~~~ \Gamma \vdash \varphi}{\Gamma \vdash \psi},~~~
    \dfrac{\Gamma, \varphi ~~~ \Gamma \vdash \psi}{\Gamma \vdash \varphi \& \psi},~~~
    \dfrac{\Gamma, \vdash \varphi \& \psi}{\Gamma \vdash \varphi},~~~
    \dfrac{\Gamma, \vdash \varphi \& \psi}{\Gamma \vdash \psi},
\]\[
    \dfrac{\Gamma \vdash \varphi}{\Gamma\vdash\varphi \vee \psi},~~~
    \dfrac{\Gamma \vdash \psi}{\Gamma\vdash\varphi \vee \psi},~~~
    \dfrac{\Gamma, \varphi \vdash \rho~~~ \Gamma, \psi \vdash \rho~~~ \Gamma \vdash \varphi \vee \psi}{\Gamma\vdash\rho},~~~
    \dfrac{\Gamma \vdash \bot }{\Gamma\vdash\varphi}.
\]

В теореме выше нужно добавить, что $\llbracket\bot \rrbracket = 0$.

$\neg \alpha \equiv \alpha \to \bot$.

\endinput
\begin{definition}
    Алгебра Гейтинга (псевдобулева алгебра)~--- импликативная решетка с 0.
\end{definition}

\begin{definition}
    Введем операцию $\sim a \equiv a \to 0$~--- дополнение до 0.
\end{definition}
\begin{definition}
    Булева алгебра~--- Алгебра Гейтинга, где $a + \sim a = 1$.
\end{definition}

\begin{example}
    Булева Алгебра

    \includegraphics[scale=0.8]{img/bool_algebra}

    \begin{itemize}
        \item $\cdot$ соответствует $\&$,
        \item $+$ соответствует $\vee$,
        \item $\to$ соответствует $\to$,
        \item $\sim$ соответствует $\neg$.
    \end{itemize}
\end{example}

Далее $\alpha, \beta$~--- высказывания в ИИВ.

\begin{definition}
    $\alpha \leqslant \beta$, если $\alpha \vdash \beta$
\end{definition}

\begin{definition}[Равносильность высказываний]
    $\alpha \approx \beta$, если $\alpha \leqslant \beta$ и $\beta \leqslant \alpha$
\end{definition}

\begin{definition}[Алгебра Линденбаума]
    Пусть $\xi$~--- множество всех высказываний ИИВ.

    $[ \xi ]$ (множество классов эквивалентности высказываний по отношению $\approx$) называется алгеброй Линденбаума $\mathcal{L}$.
\end{definition}

\begin{theorem}
    $\mathcal{L}$ (алгебра Линденбаума) --- Алгебра Гейтинга.
\end{theorem}

Введем оценку высказывания в алгебре Линденбаума. Отобразим $\llbracket\alpha\rrbracket = [\alpha]$ (то есть оценка $\alpha$ есть ее класс эквивалентности).

\begin{lemma}
    $\mathds{1} = [A \to A]$
\end{lemma}

\begin{proof}
    $\alpha \vdash A \to A$, верно (очевидно), то есть $[\alpha] \leqslant [A \to A]$, то есть $[A \to A] = \mathds{1}$.
\end{proof}

\begin{theorem}
    $\mathcal{L}$~--- корректная модель ИИВ.
\end{theorem}

\begin{proof}
    Действительно, каждая формула имеет оценку, оценки представляются в виде решетки по отношению доказуемости. Если $[\alpha] = [A \to A]$, то и $\llbracket \alpha \rrbracket = [A \to A]= \mathds{1}$.
\end{proof}

\begin{theorem}
    $\mathcal{L}$~--- полная модель ИИВ.

\end{theorem}

\begin{proof}
    $\vDash \alpha$, то есть $[\alpha] = \mathds{1}$.

    $\llbracket\alpha\rrbracket = \mathds{1}$, значит $\llbracket\alpha\rrbracket = [A \to A]$, это есть что $\alpha$ эффективна $\mathds{1}$, то есть $\beta \leqslant [\alpha]$ при всех $\beta$.

    Возьмем $\beta = A \to A$. Докажем, что $\vdash \alpha$.
    \begin{itemize}
        \item[1-5)] $A \to A$
        \item[6)] $A \to A \to \alpha$ (теорема о дедукции)
        \item[7)] $\alpha$
    \end{itemize}
\end{proof}

\begin{theorem}
    Алгебра Гейтинга (все возможные ее модели)~--- полная и корректная модель ИИВ.
\end{theorem}

\begin{definition}
    Исчисление дизъюнктно, если для любых $\alpha, \beta\quad \vdash \alpha \lor \beta$ влечёт $\vdash \alpha$ или $\vdash \beta$.
\end{definition}

\begin{theorem}
    ИИВ дизъюнктно.
\end{theorem}

\begin{definition}
    Пусть существует $f: A \to B, \quad A, B$ -- алгебры Гейтинга.

    $f$ -- гомоморфизм, если $f(0_A) = 0_B\quad f(1_A) = 1_B$ и $f(\alpha \star_A \beta) = f(\alpha) \star_B f(\beta)$
\end{definition}

\begin{definition}[Геделева Алгебра]
    Это такая алгебра, где $a + b = 1$ влечет $a = 1$ или $b = 1$.
\end{definition}

\begin{definition}
    [$\Gamma (A)$]
    Пусть $A$~--- алгебра Гейтинга.

    Определим $\gamma: A \to \Gamma(A)$ так:
    $\gamma(x) = \begin{cases}
        \omega, &x = 1_A\\
        x, & x < 1_A\\
    \end{cases}$
    и добавим $1_{\Gamma(A)}$: $t \leqslant 1_\Gamma(A)$, если $t \in \Gamma(A)$.

    \begin{center}
        \includegraphics[scale=0.5]{img/gedelerisation.png}
    \end{center}
\end{definition}

\begin{remark}
    $\Gamma (A)$ неофициально называется Геделеризацией.
\end{remark}

\begin{theorem}
  $\Gamma(A)$ -- Гёделева алгебра.
\end{theorem}
\begin{proof}
    Пусть $a+b = 1_{\Gamma(A)}$, посмотрим на картинку.
\end{proof}

\begin{definition}
    Каноническое отображение $g(x): \Gamma(\mathcal L) \to \mathcal L$\\
    $g(x) = \begin{cases}
        1,& x = 1 \text{ или } \omega\\
        x,& \text{ иначе}\\
    \end{cases} $.
\end{definition}
\begin{statement}
        $g(x)$ -- гомоморфизм
\end{statement}

\begin{statement}
    $\Gamma(\mathcal L)$~--- Гёделева алгебра.
\end{statement}

\begin{theorem}
    Рассмотрим ИИВ и алгебры Гейтинга $\mathcal L, \Gamma (\mathcal{L})$
\end{theorem}

\begin{statement}
    Если $g: A \to B$ и $\llbracket\alpha\rrbracket_A = 1_A$, то
    $\llbracket \alpha \rrbracket_B = g(1_A)$.
\end{statement}


\begin{proof}[Доказательство теоремы]

    Рассмотрим $\vdash \alpha \vee \beta$.

    $\Gamma (\mathcal L)$~--- Гёделева алгебра, то есть алгебра Гейтинга.

    $\llbracket \alpha \lor \beta \rrbracket_{\Gamma(\mathcal L)} = 1_{\Gamma(\mathcal L)}$,
    т.е. либо $\llbracket \alpha \rrbracket =1_\gamma{\mathcal L}$ либо $\llbracket \beta \rrbracket _{\Gamma(\mathcal L)} = 1_{\Gamma(\mathcal L)}$

    Рассмотрим $g : \Gamma(\mathcal{L}) \to \mathcal{L}$

    $\llbracket \alpha \rrbracket _{\Gamma(\mathcal L )} = 1_{\Gamma(\mathcal L)}$, тогда $\llbracket \alpha \rrbracket _{\mathcal L} = g(1_{\Gamma(\mathcal L)}) = 1_{\mathcal L}$

    т.е. $\vdash \alpha$.
\end{proof}

\begin{definition}
    Модель ИИВ называется табличной, если\begin{itemize}
        \item $\mathbb{V}$~--- множество истинностных значений;
        \item $\llbracket \alpha \star \beta \rrbracket = f_\star \left( \llbracket \alpha \rrbracket, \llbracket \alpha, \beta \rrbracket \right)$,
        \item  Существует $\true \in \mathcal{S}$ -- выделенная истина $\llbracket \alpha \rrbracket = \true$ тогда и только тогда, когда $\vDash \alpha$.
    \end{itemize}

\end{definition}

\begin{definition}
    Конечная табличная модель~--- модель, где $\mathbb V$~--- конечная.
\end{definition}

\begin{note}
    Оценка работает так:
    $\llbracket P_i \rrbracket = f_\rho (P_i)$,
    $\llbracket \alpha\star\beta\rrbracket=f_\star\left(\llbracket \alpha, \rrbracket, \llbracket \beta\rrbracket\right)$,
    $\llbracket \neg\alpha\rrbracket=f_{\neg}\left(\llbracket \alpha, \rrbracket\right)$.
    Если $\vdash\alpha$, то $\llbracket\alpha\rrbracket = \true$ при любой оценке.
\end{note}


Представим не один мир, а много миров, в которых высказывания могут принимать разные значения.

\includegraphics[scale=0.6]{img/kripke_model_greate_ferma_theorem}

\begin{definition}[Модель Крипки]
    Рассмотрим $W_i$ множество миров, имеющие частичный порядок ($\leqslant$).

    Зададим отношение вынужденности $W_i \Vdash P_i$ ($\Vdash \subseteq W_i \times P_i$).

    При этом если $W_j\Vdash P_i$ и $W_j \leqslant W_k$, то $W_k \Vdash P_i$.
\end{definition}

\begin{definition}
Доопределим связки  $\Vdash$ на все выражения:
\begin{enumerate}
    \item $W_i \Vdash \alpha\& \beta$, если $W_i \Vdash \alpha$ и $W_i \Vdash \beta$
    \item $W_i \Vdash \alpha\lor \beta$, если $W_i \Vdash v$ или $W_i \Vdash \beta$
    \item $W_i \Vdash \neg \alpha$, если нет $W_j \geqslant W_i$, что $W_j \Vdash \alpha$
    \item Пусть во всех $W_j \geqslant W_i$ всегда, когда $W_j \Vdash \alpha$, имеет место $W_j \Vdash \beta$, тогда в мире $W_i$ вынуждена импликация.
\end{enumerate}
\end{definition}

\includegraphics[scale=0.7]{img/forced_variable_worlds}

\begin{definition}
    Если $W_i \Vdash \alpha$ при всех $W_i \in W$, то $\vDash \alpha$ ($\alpha$ общезначима).
\end{definition}


\begin{theorem}
    Модель Крипке~--- корректная модель ИИВ.
\end{theorem}
\begin{proof}
    Пусть $\langle W, \Omega\rangle$~--- топология, $\Omega = \{\tl W \subseteq W \mid$ если  $ W_i\in \tl W, W_i \leqslant W_j$, то $W_j \in \tl W \}$ (множество всех подлесов, где каждый узел присутствует вместе со своим поддеревом).\\
    Пусть $\{W_k \mid W_k \Vdash P_j\}$~--- открытое множество.\\
    Примем $\llbracket P_j \rrbracket = \{ W_k \mid W_k \Vdash P_j\}$, тогда $\llbracket \alpha \rrbracket = \{ W_k \mid W_k \Vdash \alpha\}$.

    Поскольку любая топологическая модель~--- корректная модель ИИВ, то и модель Крипке~--- корректная.
\end{proof}

\begin{definition}
    $\vDash \alpha$ если $W \vdash \alpha$.
\end{definition}

    \begin{theorem}
        У ИИВ нет полной конечной табличной модели.
    \end{theorem}
    \begin{proof}
        $\varphi(n) = \bigvee\limits_{i = 1, j = 1, i \neq j}^{n, n} A_i \to A_{j}$ .

        Пусть $T$~--- модель, $|\mathbb{V}| = n$.

        Рассмотрим $\varphi(n+1)$.  По принципу Дирихле. Есть $A_j$ и $A_i$: $\llbracket A_j \rrbracket = \llbracket A_i \rrbracket$.

        Несложно показать $\llbracket A_i \to A_j\rrbracket = \true \implies$ $\llbracket \varphi(n + 1) \rrbracket = \true$.

        Рассмотрим модель, где $\varphi(n)$ не доказуемо ни при каком $n$.

        \includegraphics[scale=0.5]{img/iiv_table_model.PNG}

        $\llbracket A_3 \to A_k \rrbracket = \false$.
    \end{proof}

\subsection{Изоморфизм Кари--Ховарда}

\begin{statement}
    $\tau, \sigma$~-- типы.

    $\tau \to \sigma$
    \begin{lstlisting}[mathescape=true]
    f(x : $\tau$): $\sigma$ {
        return g(x);
    }\end{lstlisting}

    $\tau \& \sigma$
    \begin{lstlisting}[mathescape=true]
    f(x: $\tau$, y: $\sigma$)\end{lstlisting}

    $\tau \vee \sigma$
    \begin{lstlisting}[mathescape=true]
    f(x: std:variant<$\tau$, $\sigma$>)\end{lstlisting}

\end{statement}

\begin{definition}
    [Изоморфизм Кари--Ховарда]
    Программа соответствует доказательству.
    Тип соответствует утверждению. ...
    (всё в интуиционистской логике)
\end{definition}

\begin{note}
    $f: \neg\neg \alpha \to \alpha $ -- потом подумаем как это интерпретировать.
\end{note}

\endinput


\end{document}