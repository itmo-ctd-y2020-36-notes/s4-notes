\section{Введение}

Логика -- довольно старая наука, но наш предмет довольно молодой
В какой-то момент логики как дисциплиниы, которая учит просто правильно рассуждать, стало нехватать.
Появилась теория множеств.
Общего здравого смысла не хватает, нужен строгий математичесий язык. 
Это рубеж 19-20 веков.

У нас теория множеств не будет фокусом, как это могло бы быть на мат. факультете.

Теория множеств, когда она была впервые сформулирована, была противоречива (как матан, сформулированный Ньютоном).
Чтобы уверенно и эффективно заниматься матаном, нужно суметь его формализовать. 

<Парадокс Рассела / парадокс брадобрея>
Мы приписываем элементу-человеку свойство, которое невыполнимо.
Обьекта, выходит, не существует.
Мы смогли очень быстро определить противоречие в этом определении.
Но, может быть, мы не смогли его определить в других наших определениях? 
(конструкциях вещественной прямой, и т.д и т.д)

Программа Гильберта.
\begin{enumerate}
\item Формализуем математику!
Сформулируем теорию на языке (не на русском или английском), который не будет допускать парадоксов, 
\item $\ldots$ и на котором можно будует доказать непротиворечивость. 
\end{enumerate}

В 1930 году становится понятно, что сколько-нибудь сильная (= в ней можно построить формальную арифметику) теория не может быть доказана непротиворечивой.

Возможно, сама наша логика неправильная? 
Эта идея будет нам полезна, и к ней мы ещё вернемся.

Возможно, что это просто свойство мира, и мы хотим невозможного.

Из этих рассуждений выросло большое множество хороших идей, которые оказались полезны в других местах.
Матлогика служит широкому кругу нужд.

Мы можем доказывать, что программа работае корректно. 
Именно доказывать, а не проверять тестами!

Мы можем изучать свойства самих языков.
Изоморфизм Карри-Говарда--- доказательство это программа, утверждения это тип.
Можно изучать языки программирования и можно развернуть изоморфизм: изучать математкиу как язык программирования. 

Функциональные языки: окамль + хаскель. 
Ознакомление с этими языками преставляет собой способ ознакомиться с предметом немного с другой стороны.

\section{Исчисление высказываний}
Мы говоирм на двух языках: на предметном языке и метаязыке.
Предметный язык -- это то, что изучается, а метаязык -- это язык, НА котором это изучается.

На уроках английского предметным является сам английский, а метаязыком может быть русский.
Метаязык -- это язык исследователя, а предметный язык -- это язык исследоваемого.
Что такое язык вообще? Хороший вопрос.

Высказывание --- это одно из двух: 
\begin{enumerate}
\item Большая латниская буква начала алфавита, возможно с индексами и штрихами --- это пропозициональные переменные.
\item Выражение вида $(\alpha \land \beta)$, $(\alpha \lor \beta)$, $(\alpha \to \beta)$, $(\neg \alpha)$. 
\end{enumerate}

В определении выше альфа и бета это метапеременные--- места, куда можно подставить высказывание.
\begin{enumerate}
\item $\alpha, \beta, \gamma$ --- метапеременные для всех высказываний.
\item $X, Y, Z$ --- метапеременные для пропозициональных переменных.
\end{enumerate}

Метапеременные являются частью языка исследователя.

В формализации мы останавливаемся до места, в котором мы можем быть уверены, что сможем написать программу, которая всё проверяет.

Сокращение записи, приоритет операций: сначала $\neg$, потом $\&$, потом $\vee$, потом $\to$.
Если скобки опущены, мы восстанавливаем их по приоритетам.
Выражение без скобок является частью метаязыка, и становится частью предметного, когда мы восстанавливаем их.
Скобки последовательных импликаций расставляются по правилу правой ассоциативности --- справа налево.
\subsection{Теория моделей}
У нас есть истинные значения $\{T, F\}$ в классической логике. 
И есть оценка высказываний $\llbracket \alpha\rrbracket$. 
Например $\llbracket A \lor \neg A\rrbracket$ истинно.
Всё, что касается истинности высказываний, касается теории моделей. 
\begin{definition}
    Оценка --- это функция, сопоставляющая высказыванию его истинное (истинностное) значение.
\end{definition}
\subsection{Теория доказательств}
\begin{definition}
    Аксиомы --- это список высказываний.
    Схема аксиомы --- высказывание вместе с метопеременными; при любой подстановке высказываний вместо метапеременной получим аксиому. 
\end{definition}

\begin{definition}
    Доказательство (вывод) --- последовательность высказываний $\gamma_1, \gamma_2\ldots$ где $\gamma_i$--- любая аксиома, 
    либо существуют $j,k < i$ такие что $\gamma_j \equiv (\gamma_k \to \gamma_i)$.
    (знак $\equiv$ здесь сокращение для "имеет вид"). Это правило ''перехода по следствию'' или Modus ponens.
\end{definition}

Определим следующие 10 схем аксиом для того исчисления высказываний, которое мы рассматриваем. 
\begin{enumerate}
    \item $\alpha \to \beta \to \alpha$ --- добавляет импликацию
    \item $(\alpha \to \beta) \to (\alpha \to \beta \to \gamma) \to (\alpha \to \gamma)$ --- удаляет импликацию
    \item $\alpha \land \beta \to \alpha$
    \item $\alpha \land \beta \to \beta$
    \item $\alpha \to \beta \to \alpha \land \beta$
    \item $\alpha \to \alpha \lor \beta$
    \item $\beta \to \alpha \lor \beta$
    \item $(\alpha \to \gamma) \to (\beta \to \gamma) \to (\alpha \lor \beta \to \gamma)$
    \item $(\alpha \to \beta) \to (\alpha \to \neg \beta) \to (\neg \alpha)$
    \item $\neg \neg \alpha \to \alpha$ --- очень спорная штука.
\end{enumerate}

<вывод $A\to A$>

