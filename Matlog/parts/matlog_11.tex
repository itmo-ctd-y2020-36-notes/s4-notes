\subsection{Координальные числа}
\textit{Ординальные числа --- порядковые, а координальные~--- количественные.}
\begin{definition}
    $\langle D$ (обл. определения, domain)$, C$ (обл. значения, codomain)$, G$ (график)$\rangle$~--- это функция.
\end{definition}

\begin{definition}
    Биективная функция~--- сюрективная и инъективная функция.
\end{definition}

\begin{definition}
    $|X| = |Y|$ (множество $X$ равномощно множеству $Y$), если существует биективная функция $f: X \to Y$.
\end{definition}

\begin{definition}
    $|X| \leqslant |Y|$ ($X$ имеет мощность не больше $Y$), если существует инъектиная функция $f: X\to Y$.
\end{definition}

\begin{definition}
    $|X| < |Y|$ ($X$ имеет мощность строго меньше $Y$), если $|X| \leqslant |Y|$ и $|X| \neq |Y|$.
\end{definition}

\begin{definition}
    Координальное число~--- ординальное число $a$, у которого нет меньшего его, равномощного ему, для которого не существует числа $b$ меньшего, но равномощного ему.

    То есть, нет $b$: $|a| = |b|$, но $b \in a$.
\end{definition}

\begin{example}
    Примеры: $0, 1, 2, 3, 4, \dots, 178$.
\end{example}

\begin{statement}
    $\omega$~--- это $0, 1, 2, 3, \dots$.
\end{statement}
\begin{proof}
    Действительно, если $b \in \omega$, то $b$~--- конечен (по определеню $\omega$).
    Очевидно, $b$ не равномощен $\omega$.
\end{proof}

\begin{definition}
    На самом деле $|\omega| = \aleph_0$.
\end{definition}

Что такое $\aleph_1$?
\begin{definition}
    $\aleph_1$~--- такой $c$, что $\aleph_0 < |c|$.
\end{definition}

\begin{note}
    Координальные числа так же имеют нумерацию (потому что они ординальные числа и имеют полный порядок).

    \[ 0 < 1 < 2 < \dots < \aleph_0 < \aleph_1. \]
\end{note}

\begin{definition}
    [Альтернативное определение]

    Координальные числа~--- множество всех равномощных ориднальных чисел.
\end{definition}

Что такое $\aleph_1$? Какие есть мощности, большие $\aleph_0$ (больше счетной мощности).

\begin{example}
    \begin{enumerate}
        \item $|\omega + 1| = \aleph_0$.

        Определим $f(x) = \begin{cases}
            \omega,& x = 0\\
            x - 1, & x > 0
        \end{cases}$. $f$ очевидно биекция, следовательно имеет место быть равномощность.
        \item $|\omega \cdot \omega| = \aleph_0$.

        $\omega \cdot $~--- множество упорядоченных пар $\langle k, l\rangle$, их можно пересчитать как рациональные числа.
        \item $|\omega^\omega| = \aleph_0$.

        Доказательство в домашнем задании.
    \end{enumerate}
\end{example}

\subsubsection{Теорема Кантора}
\begin{theorem}
    Если $X$~--- некоторое множество, то $|X| < \left|\mathcal{P} (X) \right|$.
\end{theorem}
\begin{note}
    $\mathcal{P}(X)$ здесь также, как и раньше означает булеан, то есть семейство всех подмножеств $X$.
\end{note}
\begin{proof}
    Используем диагональный метод.

    Пусть не так. Пусть существует $f: X \to \mathcal{P}(X)$ и $f$~--- биекиця.

    Пусть, например  $X$~--- $\omega$. $f(3) = \left\{3, 1, 2 \right\}$, $f(2) = \left\{ 1, 4\right\}$, $f(0) = \{ \omega\}$.
    $X = \{ x_0, x_1, x_2, \dots \}$.

    Будем писать табличку.
    \begin{tabular}{|l|c|c|c|c|c|}
    \hline
    & $x_0$ & $x_1$ & $x_2$ & $x_3$ & $x_4$\\
    \hline
    $x_0$ & $\times$ & $\times$ & $\times$ & $\times$ & $\times$\\
    \hline
    $x_1$ & & & & & \\
    \hline
    $x_2$ & & $\times$ & & & $\times$\\
    \hline
    $x_3$ &  & $\times$ & $\times$ & $\times$ & \\
    \hline
    \end{tabular}
    \vspace{0.2cm}

    Биективная ли эта функция? В инъективность мы готовы поверить, а вот что с сюрьективностью?

    Пусть она сюрьективна, построим $T$, не имеющий прообраз.

    Рассмотрим $f(x_0)$. Верно ли, что $x_0 \in f(x_0)$? Если да, то $T\not\ni x_0$. Если нет, то $x_0 \in T$.

    То есть $T = \left\{ x \in \mathcal{P}(X) \mid x \notin f(x) \right\}$ (просто аксиома выделения).

    Каков $t$, что $f(t) = T$? Ни $x_0$, ни $x_1$, ни \ldots не может быть $t$.

    Значит $f$ не сюрьективна, то есть $|X| \neq \left| \mathcal{P}(X)\right|$.

    Однако, $|X| \leqslant \left|\mathcal{P}(X)\right|$. Просто потому что можно рассмотреть функцию $f(x) = \{ x\}$. $f$~--- инъективна.

    Значит $|X| < \left| \mathcal P (X)\right|$.
\end{proof}

\subsubsection{Можности, большие алеф 0, континум гипотеза}
Как тогда получить $\aleph_1$?

Действительно, можно взять $\mathcal{P} (\omega)$. Но равно ли $\left| \mathcal P (\omega) \right| = \aleph_1$?

На самом деле не известно.

\begin{definition}
    $\left| \mathcal{P}(\omega) \right|$~--- мощность континум, то есть мощность вещественных чисел.
\end{definition}

\begin{statement}
    Континум гипотеза: существуют ли промежуточные между $\aleph_0$ И $\left| \mathcal{P}(\omega) \right|$?
\end{statement}

\begin{theorem}
    [Коэн, 1962--1963]
    Континум--гипотеза не зависит от аксиоматики Цермело--Френкеля.

    Мы можем согласиться с ней, а можем отвергнуть.
\end{theorem}

\subsubsection{Теорема Кантора--Бернштейна}
\begin{theorem}
    Если $|X| \leqslant |Y|$ и $|Y| \leqslant |X|$, то $|X| = |Y|$.
\end{theorem}
\begin{proof}
    У нас есть функции $f: X \to Y$ и $g:Y \to X$, они обе инъективны.

    \begin{center}
        \tikz{
	\node (X) at (0,5.5) {$X$};
	\node (Y) at (7,5.5) {$Y$};
    \draw[dashed] (0,1) -- (0,5);
    \draw[dashed] (7,1) -- (7,5);
    \draw[->] (0,5) -- (7,4.5);
	\draw[->] (0,1) -- (7,1.5);
    \draw[-] (7.5,4.5) -- (7.5,1.5);

    \draw[->] (7,5) -- (0,3.75);
	\draw[->] (7,1) -- (0,2.25);
    \draw[-] (-0.5,3.75) -- (-0.5,2.25);
    \node (M1) at (-1,3) {$g(Y)$};
    \node (M2) at (8,3) {$f(X)$};
    \draw[->] (7,4.5) -- (0,3.25);
	\draw[->] (7,1.5) -- (0,2.75);
    \draw[-] (1,3.25) -- (1,2.75);
    \node (M3) at (2,3) {$g(f(X))$};
    }
    \end{center}

    $|X| = |f(X)|$, $|Y| = |g(Y)|$ (так как функции инъективны), значит $|g(f(X))|: X \to X$~--- инъектиная функция. Хотим показать, что ее образ равномощен исходному образу.

    Занумеруем множества.

    \begin{tabular}{r l}
$X = A_0$ & \\
$g(Y) = A_1$ & $A_1 \subseteq A_0$, так как $g: Y \to X$\\
$g(f(X)) = A_2$ & $A_2\subseteq A_1$, так как $f(X) \subseteq Y$\\
$g(f(g(Y))) = A_3$ & $A_3 \subseteq A_2$\\
$g(f(g(f(X)))) = A_4$ & $A_4 \subseteq A_3$\\
$\dots = A_5$ & \\
$\dots$ & $A_{n+1}\subseteq A_n$
    \end{tabular}

Давайте пересечем все $A_i$: $A_0 \cap A_1 \cap A_2 \cap \dots = A$.

Опреедлим $C_n = A_n \setminus A_{n+1}$. Тогда $X = C_0 \cup C_1 \cup C_2 \cup \dots \cup A $. $g(f(X)) = C_1 \cup C_2 \cup \dots \cup A$. Действительно, в $g(f(X))$ входит все, что входит в $X$, кроме $C_0$.

Построим биекцию между $X$ и $g(Y)$ так:

\begin{center}
    \tikz{
\node (X) at (-1,3) {$X$};
\node (XC0) at (-0.5,3) {$=$};
\node (C0) at (0,3) {$C_0$};
\node (C01) at (0.5,3) {$\cup$};
\node (C1) at (1,3) {$C_1$};
\node (C12) at (1.5,3) {$\cup$};
\node (C2) at (2,3) {$C_2$};
\node (C23) at (2.5,3) {$\cup$};
\node (C3) at (3,3) {$C_3$};
\node (C34) at (3.5,3) {$\cup$};
\node (C4) at (4,3) {$C_4$};
\node (C4dots) at (4.5,3) {$\cup$};
\node (CAdots) at (7.5,3) {$\cup$};
\node (CA) at (8,3) {$A$};


\node (gY) at (-1,1) {$g(Y)\,\,\,$};
\node (YC0) at (-0.5,1) {$=$};
\node (YC1) at (1,1) {$C_1$};
\node (YC12) at (1.5,1) {$\cup$};
\node (YC2) at (2,1) {$C_2$};
\node (YC23) at (2.5,1) {$\cup$};
\node (YC3) at (3,1) {$C_3$};
\node (YC34) at (3.5,1) {$\cup$};
\node (YC4) at (4,1) {$C_4$};
\node (YC4dots) at (4.5,1) {$\cup$};
\node (YCAdots) at (7.5,1) {$\cup$};
\node (YCA) at (8,1) {$A$};


\draw[<->] (C0) -- (YC2);
\draw[<->] (C1) -- (YC1);
\draw[<->] (C2) -- (YC4);
\draw[<->] (C3) -- (YC3);
\draw[<->] (C4) -- (6,1);
\draw[<->] (CA) -- (YCA);

\node (2gY) at (-1,0) {$Y$};
\node (2YC0) at (-0.5,0) {$=$};
\node (2g) at (3,0) {под действием $g^{-1}$};
}
\end{center}

Обозначим построенное отображение $p: X \to g(Y)$~--- биекиця. Тогда $q(x) = g^{-1}(p(X)): X\to Y$~--- биекция. То есть $|X| = |Y|$.
\end{proof}

\subsubsection{Мощность вещественных чисел}
\begin{theorem}
    Мощность вещественных чисел~--- $\mathcal{P}(\omega)$.
\end{theorem}
\begin{proof}
    Рассмотрим все двоичные дроби: $10,101 = 1 \cdot 2^1 + 0 \cdot 2^0 + 1 \cdot 2^{-1} + 0 \cdot 2^{-2} + 1 \cdot 2^{-3}$.
    \begin{itemize}
        \item Для каждой двоичной дроби есть вещественное число
        \item Почти все записи двоичных дробей не равны (тех, которые равны~-- конечное число).
    \end{itemize}

    Все двоичные дроби~--- это множество всех подмножеств натуральных чисел.
\end{proof}