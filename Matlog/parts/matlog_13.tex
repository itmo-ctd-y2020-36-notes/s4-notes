\section{Теорема о корректности формальной арифметики}


\paragraph{Два вида индукции}
\begin{definition}[Принцип математической индукции]
Какое бы ни было $\varphi(x)$, если $\varphi(0)$ и при всех $x$ выполнено $\varphi(x)\rightarrow \varphi(x')$, то
при всех $x$ выполнено и само $\varphi(x)$.
\end{definition}

\begin{definition}[ Принцип полной математической индукции]
Какое бы ни было $\psi(x)$, если $\psi(0)$ и при всех $x$ выполнено $(\forall t.x < t \rightarrow \psi(x))\rightarrow \psi(x')$, то
при всех $x$ выполнено и само $\psi(x)$.
\end{definition}

\begin{theorem}
  Принципы математической индукции эквивалентны
\end{theorem}
\begin{proof}
$(\Rightarrow)$ взяв $\varphi := \psi$, имеем выполненность $\varphi(x)\rightarrow\varphi(x')$, значит, $\forall x.\psi(x)$.

$(\Leftarrow)$ возьмём $\psi(x) := \forall t.t\le x\rightarrow\varphi(t)$
\end{proof}


\paragraph{Трансфинитная индукция}
\begin{theorem}Принцип трансфинитной индукции. Если для $\varphi(x)$ --- некоторого утверждения
теории множеств --- выполнено:
\begin{enumerate}
\item $\varphi(\varnothing)$
\item Если $\forall u.u \in \upsilon \rightarrow \varphi(u)$, то $\varphi(\upsilon)$ (где $\upsilon$ - это ординал)
\end{enumerate}
то $\forall u.\varphi(u)$.
\end{theorem}


\paragraph{Индукция для натуральных чисел}
\begin{lemma}Свойство индукции выполнено для натуральных чисел:
если $\varphi(0)$ и $\forall x\in\mathbb{N}_0.f(x) \rightarrow f(x')$, то $\forall x\in\mathbb{N}_0.f(x)$.
\end{lemma}

\begin{proof}
Пусть $\varphi(\varnothing)$ и $\forall u.(u \in \omega) \rightarrow \varphi(u) \rightarrow \varphi(u')$.
Рассмотрим $\tau(n) = \forall u.u \in n \rightarrow \varphi(u)$. Очевидно,
что если $m \in n$, то $\tau(n) \rightarrow \tau(m)$. Значит, выполнены условия принципа
трансфинитной индукции для $\omega$, отсюда $\tau(\omega)$, отсюда $\forall u.(u \in \omega) \rightarrow \varphi(u)$.
\end{proof}


\paragraph{Исчисление $S_\infty$}
\begin{enumerate}
\item Язык: связки $\neg$, $\vee$, $\forall$; нелогические символы: $(+)$,$(\cdot)$,$(')$,$0$,$(=)$.
\item Аксиомы: все истинные формулы вида $\theta_1=\theta_2$; все истинные отрицания формул вида $\neg\theta_1=\theta_2$
($\theta_i$ --- термы без переменных).
\item Структурные (слабые) правила:
\[ \dfrac{\zeta\vee\alpha\vee\beta\vee\delta}{\zeta\vee\beta\vee\alpha\vee\delta}, ~~~~~~~~~~~
\dfrac{\alpha\vee\alpha\vee\delta}{\alpha\vee\delta}. \]

\item Сильные правила
\[\dfrac{\delta}{\alpha\vee\delta}~~~~~~
\dfrac{\neg\alpha\vee\delta\quad\neg\beta\vee\delta}{\neg(\alpha\vee\beta)\vee\delta}~~~~~~
\dfrac{\alpha\vee\delta}{\neg\neg\alpha\vee\delta}~~~~~~
\dfrac{\neg\alpha[x := \theta]\vee\delta}{(\neg\forall x.\alpha)\vee\delta}.\]

\item Бесконечная индукция
\[ \dfrac{\alpha[x:=\overline{0}]\vee\delta~~~
\alpha[x:=\overline{1}]\vee\delta~~~
\alpha[x:=\overline{2}]\vee\delta\quad\dots}{(\forall x.\alpha)\vee\delta}.\]

\item Сечение
\[ \dfrac{\zeta\vee\alpha  \quad\quad \neg\alpha\vee\delta}
{\zeta\vee\delta}. \]
Здесь: \\
$\alpha$ --- секущая формула \\
Число связок в $\neg\alpha$ --- степень сечения.
\end{enumerate}

{Дерево доказательства}
\begin{enumerate}
\item Доказательства образуют деревья.
\item Каждой формуле в дереве сопоставим порядковое число (ординал).
\item Порядковое число заключения любого неструктурного правила строго больше порядкового числа его посылок
(больше или равно в случае структурного правила).

\[ \dfrac{
   \dfrac{}{0 = 0}\quad
   \dfrac{\dfrac{\dfrac{}{0= 0}}{\dots}}{0'= 0'}\quad
   \dfrac{\dfrac{\dfrac{\dfrac{\dfrac{}{0 = 0}}{\dots}}{0'= 0'}}{\dots}}{0''= 0''}\quad\dots
}{(\forall a.a = a)_\omega}~~~~~~
\dfrac{
   \dfrac{}{0 = 0}\quad
   \dfrac{}{0'= 0'}\quad
   \dfrac{}{0''= 0''}\quad\dots
}{(\forall a.a = a)_1}. \]

\item Существует конечная максимальная степень сечения в дереве (назовём её степенью вывода).
\end{enumerate}


\begin{theorem}
  Если $\vdash_\text{фа}\alpha$, то $\vdash_\infty|\alpha|_\infty$.
\end{theorem}
\begin{theorem}
  Если Ф.А. противоречива, то противоречива и $S_\infty$.
\end{theorem}
\begin{example}
  Обратное неверно: \[ \dfrac
{\neg\omega(\overline{0},\overline{\ulcorner\sigma\urcorner})\quad\quad
 \neg\omega(\overline{1},\overline{\ulcorner\sigma\urcorner})\quad\quad
 \neg\omega(\overline{2},\overline{\ulcorner\sigma\urcorner})\quad\quad\dots}{\forall x.\neg\omega(x,\overline{\ulcorner\sigma\urcorner})}.\]
\end{example}


% {Обратимость правил}
\begin{theorem}
Если формула $\alpha$ доказана и имеет вид, похожий на заключение правил де Моргана,
отрицания и бесконечной индукции --- то посылки соответствующих правил могут быть получены из самой
формулы $\alpha$ доказательством, причём доказательством с не большей степенью и не большим порядком.
\end{theorem}
\begin{proof}
Например, формула вида $\neg\neg \alpha\vee\delta$.

Проследим историю $\alpha$; она получена:
\begin{enumerate}
\item ослаблением --- заменим $\neg\neg\alpha$ на $\alpha$ в этом узле и последующих.
\item отрицанием --- выбросим правило, заменим $\neg\neg\alpha$ на $\alpha$ в последующих.
\end{enumerate}
Изменённый вывод --- доказательство требуемого.
\end{proof}


{Устранение сечений}
\begin{theorem}Если $\alpha$ имеет вывод степени $m>0$ порядка $t$, то
можно найти вывод степени строго меньшей $m$ с порядком $2^t$.
\end{theorem}

\begin{proof}Трансфинитная индукция по порядку $t$.\begin{enumerate}
\item База. Если $t=0$, то неструктурных правил нет, отсюда $m = 0$.
\item Переход. Рассмотрим заключительное правило.
\begin{enumerate}
\item Не сечение.
\item Сечение, секущая формула --- элементарная.
\item Сечение, секущая формула --- $\neg\alpha$.
\item Сечение, секущая формула --- $\alpha\vee\beta$.
\item Сечение, секущая формула --- $\forall x.\alpha$.
\end{enumerate}
\end{enumerate}
\end{proof}


{Случай 1. Не сечение}
\[\dfrac{\pi_{t_0}\quad\pi_{t_1}\quad\pi_{t_2}\quad\dots}{\alpha}.\]
Заменим доказательства посылок $\pi_i$ по индукционному предположению.

\begin{enumerate}
\item Если $m'_i < m_i$, то $\max m'_i < \max m_i$
\item Если $t_i \le t$, то $2^{t_i} \le 2^t$.
\end{enumerate}


{Случай 2.4. Сечение с формулой вида $\forall x.\alpha$}
\[ \dfrac{\zeta\vee\forall x.\alpha\quad\quad\neg(\forall x.\alpha)\vee\delta}{\zeta\vee\delta}.\]
Причём степень и порядок выводов компонент, соответственно, $(m_1,t_1)$ и $(m_2,t_2)$.
\begin{enumerate}
\item По индукции, вывод $\zeta\vee\forall x.\alpha$ можно упростить до $(m_1',2^{t_1})$.
\item По обратимости, для постоянного $\theta$ можно построить вывод $\zeta\vee\alpha[x := \theta]$ за $(m_1',2^{t_1})$.
\item В формуле $(\neg \forall x. \alpha)\vee\delta$ формула $\neg\forall x.\alpha$ получена
либо ослаблением, либо квантификацией из $\neg\alpha[x := \theta_k]\vee\delta_k$.
\begin{enumerate}
\item Каждое правило квантификации заменим на:
\[ \dfrac{\zeta\vee\alpha[x := \theta_k]\quad\quad(\neg\alpha[x := \theta_k])\vee\delta_k}{\zeta\vee\delta_k}.\]
\item Остальные вхождения $\neg\forall x.\alpha$ заменим на $\zeta$ (в правилах ослабления).
\end{enumerate}
\item В получившемся дереве меньше степень --- так как в $\neg\alpha[x := \theta]$ меньше связок, чем в $\neg\forall x.\alpha$.
\item Нумерацию можно также перестроить.
\end{enumerate}


{Теорема об устранении сечений}
\begin{definition}Итерационная экспонента
\[ (a\uparrow)^m(t) =
  \left\{
    \begin{array}{ll}     t,&m=0\\
 a^{(a\uparrow)^{m-1}(t)},&m > 0
    \end{array}
  \right. .\]
\end{definition}
\begin{theorem}Если $\vdash_\infty\sigma$ степени $m$ порядка $t$, то найдётся доказательство без сечений
порядка $(2\uparrow)^m(t)$
\end{theorem}
\begin{proof}
В силу конечности $m$ воспользуемся индукцией по $m$ и теоремой об уменьшении степени.
\end{proof}


{Непротиворечивость формальной арифметики}
\begin{theorem}Система $S_\infty$ непротиворечива\end{theorem}
\begin{proof}
Рассмотрим формулу $\neg 0=0$.
Если эта формула выводима в $S_\infty$, то она выводима и в $S_\infty$ без сечений.
Тогда какое заключительное правило?
\begin{enumerate}
\item Правило Де-Моргана?  Нет отрицаний дизъюнкции ($\neg(\alpha\vee\beta)\vee\delta$).
\item Отрицание?  Нет двойного отрицания ($\neg\neg\alpha\vee\delta$).
\item Бесконечная индукция или квантификация?  Нет квантора.
\item Ослабление?  Нет дизъюнкции ($\alpha \vee \delta$).
\end{enumerate}

То есть, неизбежно, $\neg 0=0$ --- аксиома, что также неверно.
\end{proof}
