\subsection{Теорема о дедукции}

\begin{definition}
    (Метаметаопределение).
    Будем большими греческими буквами $\Gamma, \Delta, \Sigma\ldots$ --- списки формул, неупорядоченные.
\end{definition}

\begin{definition}
    Вывод из гипотез: $\Gamma \vdash \alpha$.

    То есть существует $\delta_1,\dots, \delta_n$,~$\delta_n \equiv \alpha$, где $delta_i$ или схема аксиом, или m.p. из $j$ и $k$ и $j, k < i$.
\end{definition}

\begin{theorem}
    $ \Gamma, \alpha \vdash \beta$ тогда и только тогда, когда $\Gamma \vdash \alpha \to \gamma$.
\end{theorem}

\begin{proof}
    \begin{itemize}
        \item [$\Leftarrow $]
        Пусть $\delta_1, \delta_2\ldots \delta_n \equiv \alpha \to \beta$ выводит $\alpha \to \beta$.
        Дополним этот вывод двумя до доказательства новыми высказываниями: $\delta_{n+1} \equiv \alpha$ (дано нам в гипотезе), $\gamma_{n+2} \equiv \beta$ (MP шагов $n$, $n+1$) --- это и требовалось.
        \item [$\Rightarrow $]  Пусть $\Gamma, \alpha \vdash \beta$. Напишем программу, которая построит $\Gamma \vdash \alpha \to \beta$.

        Инвариант, который мы будем поддерживать: всё до $\alpha \to \delta_i $ --- док-во.
        Доказательство индукцией по $n$.
        \begin{enumerate}
            \item База: $n=1$ --- без комментариев.
            \item Если $\delta_1,\ldots, \gamma_n$ можно перестроить в доказательство $\alpha \to \gamma_n$, то $\gamma_1 \ldots \gamma_{n+1}$ тоже можно перестроить.
            Разберём случаи:
            \begin{enumerate}
                \item $\delta_i$ --- аксиома или гипотиза из $\Gamma$. \\
                ($i - 0.6$) $\delta_{i}$ \\
                ($i - 0.3$) $\delta_i \to \alpha \to \delta_i$\\
                ($i$) $\alpha \to \delta_i$ (m.p из $i-0.6$ и $i - 0.3$)
                \item $\delta_i = \alpha$, то есть надо построить $\alpha \to \alpha$\\
                ($i - 0.8, i-0.6, i-0.4, i-0.2$) (доказательство $\alpha \to \alpha$) \\
                $(i)$ $\alpha\to\alpha$
                \item $\delta_i$ получено из $\delta_j$ и $\delta_k$ ($\delta_k \equiv \delta_j \to \delta_i$)\\
                по индукционному предположению, уже есть строчки вида $\alpha \to \delta_j, \alpha \to \delta_k$\\
                ($j$) $\alpha \to \delta_j$\\
                ($k$) $\alpha \to (\delta_j \to \delta_i)$\\
                ($i - 0.6$) $(\alpha \to \delta_j) \to (\alpha \to \delta_j \to \delta_i) \to (\alpha \to \delta_i)$ (схема 2)\\
                ($i - 0.3$) $(\alpha \to \delta_j \to \delta_i) \to (\alpha \to \delta_i)$ (m.p.)\\
                ($i$) $(\alpha \to \delta_i)$ (m.p.)
            \end{enumerate}
        \end{enumerate}
    \end{itemize}
\end{proof}

\section{Теория моделей}
Мы можем докаывать модели или оценивать их.
''Мы можем доказать, что мост не развалится или можем выйти и попрыгать на нём.''

\begin{definition}
    $\mathbb{V}$~--- истинностное множество.

    $F$~--- множество высказываний нашего исчисления высказываний.

    $P$~--- множество пропозициональных переменных.
    \[ \llbracket\cdot\rrbracket: F \to \mathbb{V} \text{~--- оценка}\]
\end{definition}

\begin{definition}
    Для задания оценки необходимо задать оценку пропозициональных переменных.
    \[\llbracket \cdot \rrbracket : P\to \mathbb{V} \quad f_P\]
    Тогда:
    \[ \llbracket x \rrbracket = f_p(x)\]
\end{definition}

\begin{remark}
    Обозначение: значения пропозициональных переменных будем определять в верхнем индексе: $\llbracket \alpha \rrbracket ^{A = T, B = F \ldots}$
\end{remark}


\begin{definition}
    $\alpha$ --- общезначна (истинна), если $\llbracket \alpha \rrbracket = T$ при любой оценке $P$.

    $\alpha$ --- невыполнима (ложна), если $\llbracket \alpha \rrbracket = F$ при любой оценке $P$.

    $\alpha$ --- выполнима, если $\llbracket \alpha \rrbracket = T$ при некоторой $f_P$.

    $\alpha$ --- опровержима, если $\llbracket \alpha \rrbracket = F$ при некоторой $f_P$.
\end{definition}

\begin{definition}
    Теория корректна, если доказуемость влечёт общезначимость.

    Теория полна, если общезначимость влечёт доказуемость.
\end{definition}

\begin{definition}
    $\Gamma \vDash \alpha$ означает, что $\alpha$ следует из $\Gamma = \{ \gamma_1, \ldots, \gamma_n\}$, если $\llbracket \alpha \rrbracket = T$ всегда при $\llbracket \gamma_i \rrbracket = T$ при всех $i$.
\end{definition}

\subsection{Корректность исчисления высказываний}
\begin{theorem}
    Исчисление высказываний корректно.
    $\vdash \alpha$ влечёт $\vDash \alpha$.
\end{theorem}
\begin{proof}
    Индукция по длине доказательства $\delta_1, \dots, \delta_n$.

    Разбор случаев:
    \begin{enumerate}
        \item $\delta_i$ аксиома $\implies$ построить таблицу истинности, проверить, что все верно.
        \item $\delta_i$ --- м.п. $\delta_j$, $\delta_k \equiv \delta_j \to \delta_i \implies$ также рассмотрим таблицу истинности.
    \end{enumerate}
\end{proof}

Мы даём доказательство на метаязыке, не пускаясь в отчаянный формализм.
Такая строгость нас устраевает.

В матлогике бесмысленно формализовывать русский язык.
Она нужна, чтобы дать ответы на сложные вопросы в математике, где здравого смысла недостаточно и нужна формализация.

\subsection{Полнота исчисления высказываний}
\begin{theorem}
    Исчисление высказываний полно.
\end{theorem}

\begin{definition}
    $_{[\beta]}\alpha =
    \begin{cases}
        \alpha, & \llbracket \beta \rrbracket = T\\
        \neg \alpha,&  \llbracket \beta \rrbracket = F\\
    \end{cases}$
\end{definition}

\begin{lemma}
    $_{[\alpha]}\alpha$, \\ $_{[\beta]}\beta \vdash _{[\alpha \star \beta]} \alpha \star \beta$, \\
    $_{[\alpha]} \alpha \vdash _{[\neg \alpha]}\neg\alpha $
\end{lemma}
\begin{example}
    $\llbracket \alpha \rrbracket = T, \llbracket \beta \rrbracket = F \implies  \alpha \wedge \neg \beta \vdash \neg (\alpha \wedge \beta) $.
\end{example}

\begin{lemma}
Если $\Gamma \vdash \alpha$, то $\Gamma, \Delta \vdash \alpha$.
\end{lemma}

\begin{lemma}
Пусть дана $\alpha$, $X_1,\ldots, X_n$ --- её переменные.

\[ _{[X_1]}X_1,\ldots _{[X_n]}X_n\ \vdash _{[\alpha]}\alpha  \]
\end{lemma}
\begin{proof}
    Пусть $\tl X = _{[X_1]} X_i \dots _{[X_n]} X_n$.

    Индукция по длинне формулы $\alpha$.\\
    База: $\alpha = X_i$.\\
    Переход: есть $\alpha, \beta$. По предположению $\tl X \vdash _{[\alpha]} \alpha ~~~ \tl X \vdash _{[\beta]} \beta$.

    По леме 1 тогда  $\tl X \vdash _{[\alpha \star \beta]} \alpha \star \beta$.
\end{proof}

\begin{lemma}
    Если $\vDash \alpha$, то $\tl X \vdash \alpha$.
    То есть при любых подстановках значнией $\alpha $ будет истинна.
\end{lemma}

\begin{lemma}
    \[ \Gamma, Y \vdash \alpha, \quad \Gamma, \neg Y \vdash, \text{ то } \Gamma \vdash \alpha \]
\end{lemma}
\begin{proof}[Доказательство было в дз]
\end{proof}

\begin{lemma}
    Если $\tl X \vdash \alpha$ при всех оценках $X_1, \dots, X_n$, то $\vdash \alpha$.
\end{lemma}
\begin{proof}
    [Доказательство индукцией по $n$]
\end{proof}
\begin{theorem}
    Если $\vDash \alpha$, то $\vdash \alpha$.
\end{theorem}
\begin{proof}
    По лемме 4 и лемме 6.
\end{proof}

\section{Интуиционистская логика}
Мы не хотим дурацких коснтрукций вроде парадокса брадобрея.
Мы не хотим странных, но логически верных утверждений вроде $A\to B \lor B \to A$.
Интуиционисткая логика предлагает свою математику, в которой своя интерпретация логических связок.
ВНК-интерпретация (Брауер-Гейтинг-Колмогоров).

\begin{itemize}\itemsep=-1mm
    \item $\alpha, \beta, \gamma \ldots$ --- это конструкции.
    \item $\alpha \land \beta$ если мы умеем строить и $\alpha$, и $\beta$.
    \item $\alpha \lor \beta$, если мы умеем строить $\alpha, \beta$ и знаем, что именно.
    \item $\alpha \to \beta$, если мы умеем перестроить $\alpha$ в $\beta$.
    \item $\bot $ --- не имеет построения
    \item $\neg \alpha \equiv \alpha \to \bot$
\end{itemize}

''Теория доказательств''.
Рассмотрим классическое исчисление высказываний и заменим схему аксиом 10 на следующую
\[ \alpha \to \neg \alpha \to \beta \]

В этой формализации мы следуем не сути интуиционисткой логики, а традиции.
В интуиционисткой логике формализм это не источник логики.

Примеры моделей.
\begin{enumerate}
    \item Модели КИВ подходят: корректны, но не полны ($\llbracket A \lor \neg A \rrbracket = \true$, но $\not\vdash_{\text{И}} A \lor \neg A$).
    \item Пусть $X$ топологическое пространство.

    Пусть истоинностные значения~--- все открыте пространства в классической топологии.
    \begin{itemize}
        \item $\llbracket \alpha \& \beta \rrbracket = \llbracket \alpha \rrbracket \cap \llbracket \beta \rrbracket$.
        \item $\llbracket \alpha \lor \beta \rrbracket = \llbracket \alpha \rrbracket \cup \llbracket \beta \rrbracket$.
        \item $\llbracket \alpha \to \beta \rrbracket = \left( X \setminus \llbracket \alpha \rrbracket \cup \llbracket \beta \rrbracket \right)^{o}$.
        \item $\llbracket \neg \alpha\rrbracket = \left(X \setminus \llbracket \alpha \rrbracket \right)^{o}$.
    \end{itemize}
\end{enumerate}
\begin{theorem}
    Топологические модели~--- корректные модели ИИВ.
\end{theorem}
\begin{statement}
    $\not\vdash_{\text{И}} A \lor \neg A $.
\end{statement}
\begin{proof}
    Пусть $A = (0, +\infty)$, $\neg A = (-\infty, 0)$, $A \lor \neg A = \R \setminus \{0\} \neq \R$.
\end{proof}

\endinput
