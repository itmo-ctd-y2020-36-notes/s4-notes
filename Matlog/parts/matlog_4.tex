\begin{definition}
    Алгебра Гейтинга~--- импликативная решетка с 0.

    Введем операцию $\sim a \equiv a \to 0$~--- дополнение до 0.
\end{definition}

\begin{definition}
    Булева алегбра~--- Алгебра Гейтинга, где $a + \sim a = 1$.
\end{definition}

\begin{example}
    Булева Алгебра

    \includegraphics[scale=0.8]{img/bool_algebra}
    
    \begin{itemize}
        \item $\cdot$ соответствует $\&$,
        \item $+$ соответствует $\vee$,
        \item $\to$ соответствует $\to$,
        \item $\sim$ соответствует $\neg$.
    \end{itemize}
\end{example}

Далее $\alpha, \beta$~--- выссказывания в ИИВ.

\begin{definition}
    $\alpha \leqslant \beta$, если $\alpha \vdash \beta$
\end{definition}

\begin{definition}
    $\alpha \approx \beta$, если $\alpha \leqslant \beta$ и $\beta \leqslant \alpha$
\end{definition}

\begin{definition}
    Пусть $\xi$~--- множество всех высказываний ИИВ.
    
    Тогда $[ \xi ]$~--- называется алгеброй Линденбаума $\mathcal{L}$.
\end{definition}

\begin{theorem}
    $\mathcal{L}$ --- Алгебра Гейтинга.
\end{theorem}

\begin{lemma}
    $\mathds{1} = [A \to A]$
\end{lemma}

\begin{proof}
    $\alpha \vdash A \to A$, верно (очевидно), то есть $[\alpha] \leqslant [A \to A]$, то есть $[A \to A] = 1$.
\end{proof}

\begin{theorem}
    $\mathcal{L}$~--- корректная модель ИИВ.
\end{theorem}

\begin{theorem}
    $\mathcal{L}$~--- полная модель ИИВ.

\end{theorem}

\begin{theorem}
    $\vDash \alpha$, то есть $[\alpha] = 1$.

    $1 = [A \to A]$, то есть $[\alpha] =1$, то есть $\beta \leqslant [\alpha]$ при всех $\beta$.
    
    Возьмем $\beta = A \to A$, $A \to A \vdash \alpha$, то есть $A \to A$, $(A \to A) \to \alpha$.

\end{theorem}

\begin{theorem}
    Алгебра Гейтинга --- полная и корректная модель ИИВ.
\end{theorem}

\begin{definition}
    Исчисление дизъюнктно, если для любых $\alpha, \beta\quad \vdash \alpha \lor \beta$ влечёт $\vdash \alpha$ или $\vdash \beta$.
\end{definition}

\begin{theorem}
    ИИВ дизъюнктно.
\end{theorem}

\begin{definition}
    Пусть существует $f: A \to B, \quad A, B$ -- алгебры Гейтинга.    

    $f$ -- гомоморфизм, если $f(0_A) = 0_B\quad f(1_A) = 1_B$ и $f(\alpha \star_A \beta) = f(\alpha) \star_B f(\beta)$
\end{definition}

\begin{definition}   [Геделева Алгебра]
    Это такая алгебра, где $a + b = 1$ влечет $a = 1$ или $b = 1$.
\end{definition}

\begin{definition}
    [$\Gamma (A)$]    
    Пусть $A$~--- алгебра Гейтинга.

    Определим $\gamma: A \to \Gamma(A)$ так:
    $\gamma(x) = \begin{cases}
        \omega, &x = 1_A\\
        x, & x < 1_A\\
    \end{cases}$
    и добавим $1_{\Gamma(A)}$: $t \leqslant 1_\Gamma(A)$, если $t \in \Gamma(A)$.    

    \begin{center}
        \includegraphics[scale=0.5]{img/gedelerisation.png}    
    \end{center}
\end{definition}

\begin{remark}
    $\Gamma (A)$ неофициально называется Геделеризацией.
\end{remark}

\begin{theorem}
  $\Gamma(A)$ -- Гёделева алгебра.  
\end{theorem}
\begin{proof}
    Пусть $a+b = 1_{\Gamma(A)}$, посмотрим на картинку.
\end{proof}

\begin{statement}
    $\Gamma(\mathcal L)$~--- Гёделева алгебра.
\end{statement}
\begin{proof}
    Определим каноническое отображение $g(x): \Gamma(\mathcal L) \to \mathcal L$\\ 
    $g(x) = \begin{cases}
        1&, x = 1 \text{ или } \omega\\
        x&, \text{ иначе}\\
    \end{cases} $
    \begin{statement}
        $g(x)$ -- гомоморфизм
    \end{statement}
    
\end{proof}

\begin{theorem}
    Рассмотрим ИИВ и алгебры Гейтинга $\mathcal L, \Gamma (\mathcal{L})$
\end{theorem}

\begin{statement}
    Если $g: A \to B$ и $\llbracket\alpha\rrbracket_A = 1_A$, то
    $\llbracket \alpha \rrbracket_B = g(1_A)$.
\end{statement}


\begin{proof}[Доказательство теоремы]

    Рассмотрим $\vdash \alpha \vee \beta$.

    $\Gamma (\mathcal L)$~--- Геделва алгеба, то есть алгебра Гейтинга.
    
    $\llbracket \alpha \lor \beta \rrbracket_{\Gamma(\mathcal L)} = 1_{\Gamma(\mathcal L)}$, 
    т.е. либо $\llbracket \alpha \rrbracket =1_\gamma{\mathcal L}$ либо $\llbracket \beta \rrbracket _{\Gamma(\mathcal L)} = 1_{\Gamma(\mathcal L)}$

    Рассмотрим $g : \Gamma(\mathcal{L}) \to \mathcal{L}$

    $\llbracket \alpha \rrbracket _{\Gamma(\mathcal L )} = 1_{\Gamma(\mathcal L)}$, тогда $\llbracket \alpha \rrbracket _{\mathcal L} = g(1_{\Gamma(\mathcal L)}) = 1_{\mathcal L}$

    т.е. $\vdash \alpha$.
\end{proof}

\begin{definition}
    Модель ИИВ называется табличной, если\begin{itemize}
        \item $\mathbb{V} = \mathcal{S}$;
        \item $\llbracket \alpha \star \beta \rrbracket = f_\star \left( \llbracket \alpha \rrbracket, \llbracket \alpha, \beta \rrbracket \right)$,
        \item  Существует $\true \in \mathcal{S}$ -- выделенная истина $\llbracket \alpha \rrbracket = \true$ тогда и только тогда, когда $\vDash \alpha$
    \end{itemize}
\end{definition}
    \begin{definition}[Модель Крипки]
        Некоторые факты, появившиеся на оси времени в истинном или ложном виде и больше не меняется
        
        \includegraphics[scale=0.7]{img/kripke_model_greate_ferma_theorem}
    \end{definition}
    
    \begin{note}
        $W$ -- частично упорядоченное множество миров.
    \end{note}

    \begin{definition}
        $\Vdash$

\includegraphics[]{img/forced_variable_worlds}

        \begin{enumerate}
            \item Вынужденность переменной A определяется моделью. При этом, если $W_x \leqslant W_y$, $ W_x\Vdash A$, то $W_y \vDash A$. 
            \item Доопределим $\Vdash$ на все выражения:
            \begin{enumerate}
                \item $W \Vdash A\land B$, если $W \Vdash A$ и $W \Vdash B$
                \item $W \Vdash A\lor B$, если $W \Vdash A$ или $W \Vdash B$
                \item $W \Vdash \neg A$, если нет $W \leqslant W_x$, что $W \Vdash A$
                \item $W \Vdash A \to B$, если во всех $W \leqslant W_x $ из $W_x \Vdash A$ следует $W_x \Vdash B$
            \end{enumerate}
        \end{enumerate}
    \end{definition}

    \begin{theorem}
        У ИИВ нет полной конечной табличной модели.
    \end{theorem}
    \begin{proof}
        $\varphi(u)\bigvee\limits_{i\neq j}^n A_i \to A_{j}$ 

        Пусть $T$~--- модель, $|\mathbb{V}| = n$.

        Рассмотрим $\varphi(n+1)$.  По принципу Дирихле. Есть $A_j$ и $A_i$: $\llbracket A_j \rrbracket = \llbracket A_i \rrbracket$.

        Несложно показать $\llbracket A_i \to A_j = \true$. $\llbracket \varphi(n + 1) \rrbracket = \true$. 
    \end{proof}

\begin{theorem}
    Модель Крипке~--- корректная модель ИИВ.
\end{theorem}

\subsection{Изоморфизм Кари--Ховарда}

\begin{statement}
    $\tau, \sigma$~-- типы. 
    
    $\tau \to \sigma$
    \begin{lstlisting}[mathescape=true]
    f(x : $\tau$): $\sigma$ {
        return g(x);
    }\end{lstlisting}

    $\tau \& \sigma$
    \begin{lstlisting}[mathescape=true]
    f(x: $\tau$, y: $\sigma$)\end{lstlisting}

    $\tau \vee \sigma$
    \begin{lstlisting}[mathescape=true]
    f(x: std:variant<$\tau$, $\sigma$>)\end{lstlisting}

\end{statement}

\begin{definition}
    [Изоморфизм Кари--Ховарда]

    Программа соответствует доказательству. Тип соответствует утверждению. ...

    (всё в интуиционисткой логике)
\end{definition}

\begin{note}
    $f: \neg\neg \alpha \to \alpha $ -- потом подумаем как это интерпретировать.
\end{note}

\endinput