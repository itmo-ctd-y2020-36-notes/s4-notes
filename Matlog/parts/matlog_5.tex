\section{Исчисление предикатов}

Нам нужен новый язык. В текущем языке всё хорошо, но он имеет малую выразитеьную силу. Косвенным свидетельством этого является то, что в нём всё легко разрешается.

В чём была исходная цель Гильберта: формализовать всю математику и доказывать всё, не боясь того, что будет противоречие где-нибудь.


\begin{example}
    $\frac{\text{Каждый человек смертен}\quad  \text{Сократ человк}}{\text{Сократ смертен}}$    

    $\frac{\text{Каждый объект, если он -- человек, то он -- смертен}\quad  \text{Сократ -- человк}}{\text{Сократ -- смертен}}$    

    Цель: {\color{blue} кванторы} и {\color{red}предикаты}.

    \[
    \frac{\forall x.H(x) \to S(x)\quad H(\text{Сократ})}{S(\text{Сократ})}
    .\] 
\end{example}

Идея: нам нужно построить некоторый язык и затем поверх него построить теорию моделей и теорию доказательств.

\begin{example}
    ${\color{blue} \forall} {\color{red}x}. {\color{red}\sin x} = {\color{red}0} \lor {\color{red}(\sin^2 x) + 1} > {\color{red}1}$.

    \begin{itemize}
        \item Предметные (здесь: числовые) выражения
        \begin{itemize}
            \item Предметные переменные ${\color{red}x}$.
            \item Одно-- и двуместные функциональные символы <<синусы>>, <<возведение в квадрат>> и <<сложение>>.
            \item Нульместные <...>
        \end{itemize}
        \item Логическе выражения
        \begin{itemize}
            \item Предикатные символы <<равно>> и <<больше>>.
        \end{itemize}
    \end{itemize}
\end{example}

Язык исчисления предикатов:
\begin{enumerate}
    \item Два типа: предметные и логические выражения
    \item Предметные выражения: матепеременная $\Theta$
    \begin{itemize}
        \item Предметные переменные: ${\color{blue} a}, {\color{blue} b}, {\color{blue} c}$, \ldots, метапеременне {\color{blue} x}, {\color{blue} y}.
        \item Функциональные выражения: ${\color{blue} f(\Theta_1, \ldots, \Theta_n)}$, метапеременные ${\color{blue} f, g, \ldots}$
    \end{itemize}
    \item Логические выражения: метапеременные ${\color{blue}\alpha}, {\color{blue}\beta}, {\color{blue}\gamma}, \ldots$.
    \begin{itemize}
        \item Предикатные выражения: метампеременная, имена
        \item Связки
        \item Кванторы
    \end{itemize}
\end{enumerate}

Сокращенные записи, метаязык
\begin{enumerate}
    \item Метепаременные:
    \begin{itemize}
        \item $\psi, \phi, \pi$ -- формулы
        \item $P, Q$ -- предикатные символы
        \item $\Theta$ -- функциональные символы
        \item <...>
    \end{itemize}

    \item Скобки -- как в И.В.; квантор -- жадный:
     \[ {\color{blue}(\forall a. \underbrace{A \lor B \lor C \to \exists b. \underbrace{D \& \neg E}}) \& F} \]

    \item Дополнительные обозначения при необходимости:
    \begin{itemize}
        \item ${\color{blue} (\Theta_1 = \Theta_2)} $ вместо ${\color{blue} E(\Theta_1, \Theta_2)}$.
        \item ${\color{blue} (\Theta_1 + \Theta_2)} $ вместо ${\color{blue} p(\Theta_1, \Theta_2)}$.
        \item ${\color{blue} 0} $ вместо ${\color{blue} z}$.
    \end{itemize}
\end{enumerate}



Оценка формулы\ldots

< todo >

\begin{example}
    $\llbracket \forall x.\exists y.\neg x+1=y \rrbracket $

    Зададим оценку:
    \begin{itemize}
        \item $D:=\N $
        \item $F_1:=1, F_{+}$ -- сложение в $\N $
        \item 
    \end{itemize}

    Фиксируем $x\in \N $ Тогда:
    \[
    \llbracket x + 1 = y \rrbracket ^{y:=x} = \false
    .\] 
    Поэтому при любом $x\in \N $:
    \[
...
    .\] 
    Итого:
    \[ \llbracket \forall x. \exists y. \neg x + 1 = y \rrbracket = \true \]
\end{example}

\begin{example}
    Странная интерпретация

   \[ \llbracket \forall x. \exists y. \neg x + 1 = y \rrbracket \]

    Зададим интерпретацию:
    \begin{itemize}
        \item $D := \{ \square \}$;
        \item $F_{(1)} := \{ \square \}, ~ F_{(+)} (x, y) = \square$;
        \item $P_{(=)} (x, y) = \true$
    \end{itemize}

    Тогда $\llbracket  x + 1 = y \rrbracket^{x\in D, y\in D} = \true$.
   x
    Итого $\llbracket \forall x. \exists y.  \neg x + 1 = y \rrbracket^{x\in D, y\in D} = \false$.

    Поэтому формулам оценки предикатов верить нельзя. Никакой интуиции за ними может и не стоять
\end{example}

\begin{definition}
    Формула общезначима, если истинна при любой оценке
\end{definition}

\begin{theorem}
    \[
    \llbracket \forall x.Q(f(x))\lor \neg Q(f(x)) \rrbracket 
    .\] 
\end{theorem}
\begin{proof}
    Перебор случаев
\end{proof}

\begin{definition}
    Рассмотрим формулу $\forall x.\psi$ (или $\exists x.\psi$). Здесь переменная $x$ связзана в $\psi$ Все вхождения переменой $x$ в $\psi$ -- связанные
\end{definition}
\begin{definition}
    Переменная $x$ входит свободно в $\psi$ <..аа>
\end{definition}


\begin{definition}
    Терм $\Theta$ свободен для подстановки вместо $x$ в $\psi$ ($\psi [x:=\Theta]$), если ни одно свободное вхождение переменной в $\Theta$ не станет связным после подстановки.
\end{definition}
Свобода есть: $(\forall x. P(y)) [y:=z]$ или $(\forall x. \forall y. P(x))$.

Cвободы нет: $(\forall x.P(y))[y:=x]$ и $(\forall y.\forall x.P(t))[t:=y]$   .

\section{Теория доказательств}

11. $(\forall x . \phi) \to \phi [x := \Theta]$

12 $(\exists x . \phi) \to $ fix me

Добавим еще два правила вывода (здесь везде $x$ не входит свободно в $\varphi$):
\begin{itemize}
    \item Введение $\forall$. $\dfrac{2}{2}$.
\end{itemize}

\begin{definition}
    Доказыуемость, выводисость, полнота, корректность --- аналогично исчислению высказыаваний.
\end{definition}


\section{Теорема о едукции для исчисления предикатов}

\begin{theorem}
    Если    $\Gamma \vdash \alpha \to \beta$, то $\Gamma, \alpha \vdash \beta$.
    Если $\Gamma, \alpha\vdash \beta$ и в доказательстве не применяются правила для кванторов по свободным переменным из $\alpha$, то $\gamma\vdash \alpha \to \beta$
\end{theorem}
\begin{proof}
    \begin{itemize}
        \item []
        \item [$\implies $] также как в К.И.В
        \item [$\impliedby $] та же схема, два новых случая.аксиомы те же, но нужно обработать два новых правила вывода.
    \end{itemize}
\end{proof}

валмлдоыауова лдоваы оафлдов флва пло щзвдфыфа ф
спасите 

хехехе модус понос

короче все докажется

\paragraph{Следование}

\begin{definition}[Следование]
    $\gamma_1, \gamma_2, \dots, \gamma_n \vDash \alpha$, если выполнено два условия:
    \begin{enumerate}
        \item $\alpha$ выполнено всегда, когда выполнено
    \end{enumerate}
    ...
\end{definition}

Влажность второго условия. 


Зачем нам это потребовалось?  Мы будем пользоваться, но не злоупотреблять.


Мы не хотим заранее сильно ограничивать язык. ПОэтому мы выбираем такой вариант, чтобы он разрешал некоторые.


\begin{example}
    $\vdash \exists x. P(x) \to P(x)$\ldots

    (1) $A \to (A \to A) \to A \implies P()$
\end{example}

\begin{example}
    $\vdash x. P(x) \to P(x)$
\end{example}

% строка переноса
