\section{Исчисление предикатов}

Нам нужен новый язык. В текущем языке всё хорошо, но он имеет малую выразитеьную силу. Косвенным свидетельством этого является то, что в нём всё легко разрешается.

В чём была исходная цель Гильберта: формализовать всю математику и доказывать всё, не боясь того, что будет противоречие где-нибудь.


\begin{example}
    $\frac{\text{Каждый человек смертен}\quad  \text{Сократ человк}}{\text{Сократ смертен}}$

    $\frac{\text{Каждый объект, если он -- человек, то он -- смертен}\quad  \text{Сократ -- человк}}{\text{Сократ -- смертен}}$

    Цель: {\color{blue} кванторы} и {\color{red}предикаты}.

    \[
    \frac{\forall x.H(x) \to S(x)\quad H(\text{Сократ})}{S(\text{Сократ})}
    .\]
\end{example}

Идея: нам нужно построить некоторый язык и затем поверх него построить теорию моделей и теорию доказательств.

\begin{example}
    ${\color{blue} \forall} {\color{red}x}. {\color{red}\sin x} = {\color{red}0} \lor {\color{red}(\sin^2 x) + 1} > {\color{red}1}$.

    \begin{itemize}
        \item Предметные (здесь: числовые) выражения
        \begin{itemize}
            \item Предметные переменные ${\color{red}x}$.
            \item Одно-- и двуместные функциональные символы <<синусы>>, <<возведение в квадрат>> и <<сложение>>.
            \item  Нульместные функциональные символы <<ноль>> ($\color{red}0$) и <<один>> ($\color{red}1$).
        \end{itemize}
        \item Логическе выражения
        \begin{itemize}
            \item Предикатные символы <<равно>> и <<больше>>.
        \end{itemize}
    \end{itemize}
\end{example}

\subsection{Язык исчисления предикатов}
\begin{enumerate}
    \item Два типа: предметные и логические выражения
    \item Предметные выражения: метапеременная {\color{blue}$\theta$}
    \begin{itemize}
        \item Предметные переменные: {\color{blue}$a$}, {\color{blue}$b$}, {\color{blue}$c$}, ..., метапеременные {\color{blue}$x$}, {\color{blue}$y$}.
        \item Функциональные выражения: {\color{blue}$f(\theta_1,\dots,\theta_n)$}, метапеременные {\color{blue}$f$}, {\color{blue}$g$}, \ldots
        \item Примеры: {\color{blue}$r$},
        {\color{blue}$q(p(x,s),r)$}
    \end{itemize}
    \item Логические выражения: метапеременные {\color{blue}$\alpha$}, {\color{blue}$\beta$}, {\color{blue}$\gamma$}, \ldots
    \begin{itemize}
        \item Предикатные выражения: {\color{blue}$P(\theta_1,\dots,\theta_n)$}, метапеременная {\color{blue}$P$}.\\
        Имена: {\color{blue}$A$}, {\color{blue}$B$}, {\color{blue}$C$}, \ldots,
        \item Связки: {\color{blue}$(\varphi\vee\psi)$}, {\color{blue}$(\varphi\to\psi)$}, {\color{blue}$(\varphi\rightarrow\psi)$},
        {\color{blue}$(\neg\varphi)$}
        \item Кванторы: {\color{blue}$(\forall x.\varphi)$} и {\color{blue}$(\exists x.\varphi)$}.
    \end{itemize}
\end{enumerate}

Сокращенные записи, метаязык
\begin{enumerate}
    \item Метепаременные:
    \begin{itemize}
        \item $\color{blue}\psi$, $\color{blue}\phi$, $\color{blue}\pi$, ... --- формулы
        \item $\color{blue}P$, $\color{blue}Q$, ... --- предикатные символы
        \item $\color{blue}\theta$, ... --- термы
        \item $\color{blue}f$, $\color{blue}g$, ... --- функциональные символы
        \item $\color{blue}x$, $\color{blue}y$, ... --- предметные переменные
        \end{itemize}

    \item Скобки --- как в И.В.; квантор --- жадный:
    \[ {\color{blue}(\forall a.} \underbrace{{\color{blue}A \vee B \vee C \rightarrow \exists b.}
                        \underbrace{\color{blue}D \& \neg E}_{\exists b.\dots}}_{\forall a.\dots} \color{blue}) \& F \]

    \item Дополнительные обозначения при необходимости:
    \begin{itemize}
        \item ${\color{blue} (\theta_1 = \theta_2)} $ вместо ${\color{blue} E(\theta_1, \theta_2)}$.
        \item ${\color{blue} (\theta_1 + \theta_2)} $ вместо ${\color{blue} p(\theta_1, \theta_2)}$.
        \item ${\color{blue} 0} $ вместо ${\color{blue} z}$.
    \end{itemize}
\end{enumerate}

Напомним формулу:\vspace{-0.3cm}

$$\forall x.\sin x = 0 \vee (\sin x)^2+1 > 1$$

Без синтаксического сахара:\vspace{-0.3cm}

\[ {\color{blue}\forall} {\color{red}x}.{\color{blue}E}({\color{red}f} ({\color{red}x}),{\color{red}z}) {\color{blue}~\vee~} {\color{blue}G}
   ({\color{red}p}({\color{red}q}({\color{red}s} ({\color{red}x})),{\color{red}o}), {\color{red}o}) \]

\subsection{Два вида значений}
\begin{enumerate}
\item {\color{blue}Истинностные (логические) значения: }
\begin{enumerate}
\item предикаты (в том числе пропозициональные переменные = нульместные предикаты);
\item логические связки и кванторы.
\end{enumerate}

\item{\color{red}Предметные значения:}
\begin{enumerate}
\item предметные переменные;
\item функциональные символы (в том числе константы = нульместные функциональные символы)
\end{enumerate}
\end{enumerate}

\subsection{Оценка исчисления предикатов}

\begin{definition}
    Оценка --- упорядоченная четвёрка $\langle D, F, T, E \rangle$, где:
    \begin{enumerate}
        \item $D$ --- предметное множество;
        \item $F$ --- оценка для функциональных символов. Пусть $f_n$ --- $n$-местный функциональный символ:
        \[ F_{f_n}: D^n \rightarrow D \]

        \item $T$ --- оценка для предикатных символов. Пусть $P_n$ --- $n$-местный предикатный символ:
        \[ T_{P_n}: D^n \rightarrow V~~~~ V = \{\true, \false\} \]

        \item $E$ --- оценка для свободных предметных переменных.
         $$E(x) \in D$$
        \end{enumerate}
\end{definition}

Запись и сокращения записи подобны исчислению высказываний:
\[ \llbracket \phi \rrbracket \in V,\quad
      \llbracket E(x,f(x))\vee R\rrbracket^{x := 1, f(t) := t^2, R := \true} = \true \]

\begin{enumerate}
\item Правила для связок $\vee$, $\&$, $\neg$, $\rightarrow$ остаются прежние;
\item $\llbracket f_n (\theta_1, \theta_2, \dots, \theta_n) \rrbracket = F_{f_n} (\llbracket\theta_1\rrbracket,
          \llbracket\theta_2\rrbracket, \dots, \llbracket\theta_n\rrbracket)$
\item $\llbracket P_n (\theta_1, \theta_2, \dots, \theta_n) \rrbracket = T_{P_n} (\llbracket\theta_1\rrbracket,
          \llbracket\theta_2\rrbracket, \dots, \llbracket\theta_n\rrbracket)$
\item $\llbracket \forall x.\phi \rrbracket = \left\{\begin{array}{ll}
   \true, & \text{если } \llbracket\phi\rrbracket^{x := t} = \true\text{ при всех } t \in D\\
   \false, & \text{если найдётся } t \in D, \text{ что } \llbracket\phi\rrbracket^{x := t} = \false
  \end{array}\right.$
\item $\llbracket \exists x.\phi \rrbracket = \left\{\begin{array}{ll}
   \true, & \text{если найдётся } t \in D, \text{ что } \llbracket\phi\rrbracket^{x := t} = \true\\
   \false, & \text{если } \llbracket\phi\rrbracket^{x := t} = \false\text{ при всех } t \in D
  \end{array}\right.$
\end{enumerate}

\begin{example}
    $\llbracket \forall x.\exists y.\neg x+1=y \rrbracket $

    Зададим оценку:
    \begin{itemize}
    \item $D := \N$;
    \item $F_{1} := 1$, $F_{(+)}$ --- сложение в $\N$;
    \item $P_{(=)}$ -- равенство в $\N$.
    \end{itemize}

    Фиксируем $x\in\N$.
    Тогда $\llbracket x+1 = y \rrbracket^{y := x} = \false$
    поэтому при любом $x \in \N$:
    \[ \llbracket \exists y.\neg x+1 = y \rrbracket = \true. \]

    Итого: $\llbracket \forall x.\exists y.\neg x+1 = y \rrbracket = \true$
\end{example}

\begin{example}
    Странная интерпретация
    $ \llbracket \forall x. \exists y. \neg (x + 1 = y) \rrbracket$.

    Зададим интерпретацию:
    \begin{itemize}
    \item $D := \{\square\}$;
    \item $F_{(1)} := \square$, $F_{(+)} (x,y) := \square$;
    \item $P_{(=)} (x,y) := \true$.
    \end{itemize}\vspace{0.5cm}

    Тогда:
    $\llbracket x+1 = y \rrbracket^{x \in D, y \in D} = \true$.

    Итого:
    $\llbracket \forall x.\exists y.\neg x+1 = y \rrbracket = \false$.

    Поэтому формулам оценки предикатов верить нельзя. Никакой интуиции за ними может и не стоять.
\end{example}

\begin{definition}
    Формула общезначима, если истинна при любой оценке.
\end{definition}

\begin{statement}
    $\llbracket \forall x.Q(f(x))\lor \neg Q(f(x)) \rrbracket = \true$.
\end{statement}
\begin{proof}
    Фиксируем $D$, $F$, $P$, $E$.  Пусть $x \in D$.
    Обозначим $P_{Q}(F_{f}(E_x))$ за $t$.
    Ясно, что $t \in V$. Разберём случаи.
    \begin{itemize}
    \item Если $t = \true$, то $\llbracket P(f(x))\rrbracket^{P(f(x)):=t} = \true$,
      потому $\llbracket P(f(x))\vee\neg P(f(x))\rrbracket^{P(f(x)):=t} = \true$.
    \item Если $t = \false$, то $\neg P(f(x))\rrbracket^{P(f(x)):=t} = \true$ потому
      всё равно $\llbracket P(f(x))\vee\neg P(f(x))\rrbracket^{P(f(x)):=t} = \true$.
    \end{itemize}\end{proof}

\subsection{Подстановки, свобода и связность}
\begin{definition}
    Рассмотрим формулу $\forall x.\psi$ (или $\exists x.\psi$). Здесь переменная $x$ связзана в $\psi$. Все вхождения переменой $x$ в $\psi$ -- \textbf{связанные}.
\end{definition}
\begin{definition}
    Переменная $x$ входит свободно в $\psi$, если не находится в области действия никакого квантора по $x$.
Все её вхождения в $\psi$ --- \textbf{свободные}.
\end{definition}
\begin{example}
    $\exists y.(\forall x.P(x))\vee P({\color{blue}x}) \vee Q(y)$.

    Единственное свободное вхождение прееменной $x$ помеченно синим цветом.
\end{example}

\begin{definition} Подстановка~--- это \ldots
    \[ \psi[x := \theta] := \left\{\begin{array}{ll}\psi, & \psi\equiv y, y \not\equiv x\\
    \psi, & \psi\equiv\forall x.\pi \textrm{ или } \psi\equiv\exists x.\pi\\
    \pi[x := \theta] \star \rho [x := \theta], & \psi\equiv \pi\star\rho\\
    \theta, & \psi\equiv x\\
    \forall y.\pi[x := \theta], & \psi \equiv \forall y.\pi \textrm{ и } y \not\equiv x\\
    \exists y.\pi[x := \theta], & \psi \equiv \exists y.\pi \textrm{ и } y \not\equiv x
\end{array}\right. \]
\end{definition}


\begin{definition}
    Терм $\theta$ свободен для подстановки вместо $x$ в $\psi$ ($\psi [x:=\Theta]$), если ни одно свободное вхождение переменной в $\Theta$ не станет связным после подстановки.
\end{definition}

Свобода есть: $(\forall x. P(y)) [y:=z]$ или $(\forall x. \forall y. P(x)) [y:=z]$.

Cвободы нет: $(\forall x.P(y))[y:=x]$ и $(\forall y.\forall x.P(t))[t:=y]$.

\subsection{Теория доказательств}
Рассмотрим язык исчисления предикатов. АксиомыЁ--- все схемы аксиом для классического исчисления высказываний
в данном языке.

\begin{tabular}{ p{7cm} p{7cm} }
    \begin{enumerate}
        \item $\alpha \to \beta \to \alpha$
        \item $(\alpha \to \beta) \to (\alpha \to \beta \to \gamma) \to (\alpha \to \gamma)$
        \item $\alpha \land \beta \to \alpha$
        \item $\alpha \land \beta \to \beta$
        \item $\alpha \to \beta \to \alpha \land \beta$
    \end{enumerate} &
    \begin{enumerate}
        \addtocounter{enumi}{5}
        \item $\alpha \to \alpha \lor \beta$
        \item $\beta \to \alpha \lor \beta$
        \item $(\alpha \to \gamma) \to (\beta \to \gamma) \to (\alpha \lor \beta \to \gamma)$
        \item $(\alpha \to \beta) \to (\alpha \to \neg \beta) \to (\neg \alpha)$
        \item $\neg \neg \alpha \to \alpha$
    \end{enumerate}
\end{tabular}

Добавим ещё две схемы аксиом (здесь везде $\theta$ свободен для подстановки вместо $x$ в $\varphi$):

\begin{enumerate}
\addtocounter{enumi}{10}
\item $(\forall x.\varphi) \rightarrow \varphi[x:=\theta]$
\item $\varphi[x:=\theta] \rightarrow \exists x.\varphi$
\end{enumerate}

Добавим ещё два правила вывода (до этого был только Modus Ponens):
\begin{enumerate}
    \item Введение $\forall$: $\dfrac{\varphi\rightarrow\psi}{\varphi\rightarrow\forall x.\psi}$,
    \item Введение $\exists$: $\dfrac{\psi\rightarrow\varphi}{(\exists x.\psi)\rightarrow\varphi}$.
\end{enumerate}
В обоих правилах $x$ не входит свободно в $\varphi$  не входит свободно в.

\begin{example}
$\dfrac{x = 5 \to x^2 = 25}{x = 5 \to \forall x. x^2 = 25}$. Между $x$ и $x^2$ была связь, которую мы нарушили ограничение.
\end{example}

\begin{statement}
    Доказыуемость, выводимость, полнота, корректность --- аналогично исчислению высказыаваний.
\end{statement}


\subsection{Теорема о дедукции для исчисления предикатов}
\begin{theorem}
    Если    $\Gamma \vdash \alpha \to \beta$, то $\Gamma, \alpha \vdash \beta$.
    Если $\Gamma, \alpha\vdash \beta$ и в доказательстве не применяются правила для кванторов по свободным переменным из $\alpha$, то $\Gamma\vdash \alpha \to \beta$
\end{theorem}
\begin{proof}
    \begin{itemize}
        \item []
        \item [$\implies $] также как в К.И.В
        \item [$\impliedby $] та же схема. У нас появились два новых случая аксиом. Ничего страшного, с ним проблем не возникнет.

Однако таже слоедует обработать два новых правила вывода.

Перестроим: $\delta_1, \delta_2, \dots, \delta_n \equiv \beta$ в $\alpha\rightarrow\delta_1, \alpha\rightarrow\delta_2, \dots, \alpha\rightarrow\delta_n$.

Дополним: обоснуем $\alpha\rightarrow\delta_n$, если предыдущие уже обоснованы (по индукции).

Два новых похожих случая: правила для $\forall$ и $\exists$. Рассмотрим $\forall$. Для квантора существования аналогично.

Доказываем переходи к $(n)$. $\alpha\rightarrow\psi\rightarrow\forall x.\varphi$ (правило для $\forall$), значит, доказано
на шаге $k$, что $\alpha\rightarrow\psi\rightarrow\varphi$.

\begin{tabular}[t]{p{2.8cm} p{4.6cm} p{7cm}}
$(n-0.9)-(n-0.8)$ &
$(\alpha\rightarrow\psi\rightarrow\varphi)\rightarrow(\alpha\&\psi)\rightarrow\varphi$ & Т. о полноте КИВ (формула общезначима)\\
$(n-0.6)$ & $(\alpha \& \psi)\rightarrow\varphi$ & M.P. $k$,$n-0.8$\\
$(n-0.4)$ & $(\alpha \& \psi)\rightarrow\forall x.\varphi$ & Правило для $\forall$, $n-0.6$\\
\parbox{2.8cm}{$(n-0.3)-(n-0.2)$\\} & \parbox{4.6cm}{$((\alpha\&\psi)\rightarrow\forall x.\varphi)\rightarrow$\\$(\alpha\rightarrow\psi\rightarrow\forall x.\varphi)$} & \parbox{7cm}{Т. о полноте КИВ \\($((\alpha\&\psi)\rightarrow\beta)\rightarrow(\alpha\rightarrow\psi\rightarrow\beta)$)}\\
$(n)$ & $\alpha\rightarrow\psi\rightarrow\forall x.\varphi$ & M.P. $n-0.4$, $n-0.2$
\end{tabular}
    \end{itemize}
\end{proof}

\subsection{Отношение следования}

\begin{definition}[Следование]
    $\gamma_1,\gamma_2,\dots,\gamma_n\models\alpha$, если выполнено два условия:
\begin{enumerate}
\item $\alpha$ выполнено всегда, когда выполнено $\gamma_1,\gamma_2,\dots,\gamma_n$;
\item $\alpha$ не использует кванторов по переменным, входящим свободно в $\gamma_1,\gamma_2,\dots,\gamma_n$.
\end{enumerate}
\end{definition}

\begin{theorem}
    Если $\Gamma\vdash\alpha$ и в доказательстве не используется кванторов по свободным
переменным из $\Gamma$, то $\Gamma\models\alpha$.
\end{theorem}

Следующий пример обращает внимание на важность второго условия.
\begin{example}
    Покажем, что $\Gamma\models\alpha$ ведёт себя неестественно, если
в $\alpha$ используются кванторы по переменным, входящим свободно в $\Gamma$.

Легко показать, что $P(x)\vdash \forall x.P(x)$.

\begin{tabular}{lll}
(1) & $P(x)$ & Гипотеза \\
(2) & $P(x)\rightarrow (A\rightarrow A \rightarrow A) \rightarrow P(x)$ & Сх. акс. 1\\
(3) & $(A\rightarrow A \rightarrow A) \rightarrow P(x)$ & M.P. 1, 2\\
(4) & $(A\rightarrow A \rightarrow A) \rightarrow \forall x.P(x)$ & Правило для $\forall$, 3\\
(5) & $(A\rightarrow A \rightarrow A)$ & Сх. акс. 1\\
(6) & $\forall x.P(x)$ & M.P. 5, 4
\end{tabular}

Пусть $D = \mathbb{Z}$ и $P(x) = x > 0$. Тогда не будет выполнено $P(x)\models \forall x.P(x)$.
\end{example}

Зачем нам это потребовалось?  Мы будем пользоваться, но не злоупотреблять.

Мы не хотим заранее сильно ограничивать язык. Поэтому мы выбираем такой вариант, чтобы он разрешал некоторые.

% строка переноса
