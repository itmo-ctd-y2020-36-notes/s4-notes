\subsection{Машина Тьюринга}
\begin{definition}
    Машина Тьюринга --- упорядоченная тройка:
\begin{enumerate}
\item Внешний алфавит $q_1, \dots, q_n$
\item Внутренний алфавит (состояний) $s_1, \dots, s_k$; $s_s$ --- начальное, $s_f$ --- конечное.
\item Таблица переходов $\langle k, s \rangle \Rightarrow \langle k', s', \leftrightarrow \rangle$
\end{enumerate}
\end{definition}

\begin{definition}
    Состояние машины Тьюринга --- упорядоченная тройка:
\begin{enumerate}
\item Бесконечная лента с символом-заполнителем $q_\varepsilon$, текст конечной длины.
\item Головка над определённым символом
\item Символ состояния (состояние в узком смысле) --- символ внутреннего алфавита.
\end{enumerate}
\end{definition}


\begin{example}
    [Машина, меняющая все 0 на 1, а все 1 --- на 0]
\begin{enumerate}
    \item Внешний алфавит $\varepsilon, 0, 1$
    \item Внутренний алфавит $s_s, s_f$ (начальное и завершающее состояния соответственно).
    \item Переходы:

    \begin{tabular}{l|lll}
    & $\varepsilon$ & 0 & 1\\\hline
    $s_s$ & $\langle s_f,\varepsilon,\cdot\rangle$ & $\langle s_s,1,\rightarrow\rangle$ & $\langle s_s,0,\rightarrow\rangle$\\
    $s_f$ & $\langle s_f,\varepsilon,\cdot\rangle$ & $\langle s_f,0,\cdot\rangle$ & $\langle s_f,1,\cdot\rangle$
\end{tabular}
\end{enumerate}

Пусть головка --- на первом символе $0 1 1$, состояние $s_s$.

$\underline{0} 1 1$  $\Rightarrow 1 \underline{1} 1$  $\Rightarrow 1 0 \underline{1}$  $\Rightarrow 1 0 0 \underline{\varepsilon}$

Состояние $s_f$, завершающее.
\end{example}



\subsubsection{Разрешимость языка Машины Тьюринга}

\begin{definition}
    Язык~--- множество строк.
\end{definition}
\begin{definition}
    Язык $L$ разрешим, если существует машина Тьюринга, которая для любого слова $w$ возвращает ответ <<да>>, если $w \in L$,
и <<нет>>, если $w \not\in L$.
\end{definition}

\subsubsection{Неразрешимость задачи останова}
\begin{definition}
    Рассмотрим все возможные описания машин Тьюринга. Составим упорядоченные пары: описание машины Тьюринга и входная строка.
Из них выделим язык останавливающихся на данном входе машин Тьюринга.
\end{definition}

\begin{theorem}
    Язык всех останавливающихся машин Тьюринга неразрешим
\end{theorem}
\begin{proof}
    От противного. Пусть $S(x,y)$ --- машина Тьюринга, определяющая, остановится ли машина $x$, примененная к строке $y$.
\begin{center}
    W(x) = if (S(x,x)) \{ while (true); return 0; \} else \{ return 1; \}
\end{center}
Что вернёт $S(code(W),code(W))$?
\end{proof}

\paragraph{Как закодировать состояние машины?}
\begin{enumerate}
    \item внешний алфавит: $n$ 0-местных функциональных символов $q_1, \dots, q_n$; $q_\varepsilon$ --- символ-заполнитель.
    \item список: $\varepsilon$ и $c(l,s)$; <<abc>> представим как $c(q_a,c(q_b,c(q_c,\varepsilon)))$;
    \item положение головки: <<ab.pq>> как $(c(q_b,c(q_a,\varepsilon)), c(q_p,c(q_q,\varepsilon)))$.
    \item внутренний алфавит: $k$ 0-местных функциональных символов $s_1, \dots, s_k$. Из них выделенные $s_s$~--- начальное и $s_f$ --- завершающее состояние.
\end{enumerate}

\paragraph{Достижимые состояния}
Предикатный символ $F_{x,y}(w_l,w_r,s)$: если у машины $x$ с начальной строкой $y$ состояние $s$ достижимо на строке $rev(w_l) @ w_r$.

Будем накладывать условия: семейство формул $C_m$.
Очевидно, начальное состояние достижимо:
\[ C_0 = F_{x,y}(\varepsilon,x,s_s). \]


\paragraph{Кодируем переходы}
\begin{enumerate}
    \item Занумеруем переходы.
    \item Закодируем переход $m$: $\langle k, s \rangle \Rightarrow \langle k', s', \rightarrow \rangle$.

    \[ C_m = \forall w_l.\forall w_r.F_{x,y}(w_l,c(q_k,w_r),s_s) \rightarrow F_{x,y}(c(q_{k'},w_l),w_r,s_{s'}). \]
    \item Переход посложнее:
$ \langle k, s \rangle \Rightarrow \langle k', s', \leftarrow \rangle. $

\begin{align*}
    C_m = &\forall w_l.\forall w_r.\forall t.F_{x,y}(c(t,w_l),c(q_k,w_r),s_s) \rightarrow F_{x,y}(w_l,c(t,c(q_{k'},w_r)),s_{s'})\, \& \\
    &\forall w_l.\forall w_r.F_{x,y}(\varepsilon,c(q_k,w_r),s_s)
\rightarrow F_{x,y}(\varepsilon,c(q_\varepsilon,c(q_{k'},w_r)),s_{s'}). \end{align*}

    \item и т.п.
\end{enumerate}

Итоговая формула: $C = C_0 \& C_1 \& \dots \& C_n$
<<правильное начальное состояние и правильные переходы между состояниями>>.

\begin{theorem}
состояние $s$ со строкой $rev(w_l)@w_r$ достижимо тогда и только тогда, когда
$C \vdash F_{x,y}(w_l,w_r,s)$
\end{theorem}
\begin{proof}
$(\Leftarrow)$ Рассмотрим модель: предикат $F_{x,y}(w_l,w_r,s)$ положим истинным, если состояние достижимо.
Это~--- модель для $C$ (по построению $C_m$).
Значит, доказуемость влечёт истинность (по корректности).

$(\Rightarrow)$ Индукция по длине лога исполнения.
\end{proof}


\subsubsection{Неразрешимость исчисления предикатов: доказательство}
\begin{theorem}
    Язык всех доказуемых формул исчисления предикатов неразрешим

    Т.е. нет машины Тьюринга, которая бы по любой формуле $s$ определяла, доказуема ли она.
\end{theorem}
\begin{proof}
    $s_f$ --- завершающее состояние.

    Умение определять истинность формулы $\exists w_l.\exists w_r.F_{x,y}(w_l,w_r,s_f)$ разрешает задачу останова.
\end{proof}


\section{Формальная арифметика и Аксиоматика Пеано}

\textit{Какие мы знаем числа?}

\begin{enumerate}
    \item Вещественные ($\R$).  $X = \{ A, B \}$, где $A,B \subseteq \Q$ --- дедекиндово сечение, если:
    \begin{enumerate}
        \item $A\cup B = \Q$
        \item Если $a \in A$, $x \in \Q$ и $x \le a$, то $x \in A$
        \item Если $b \in B$, $x \in \Q$ и $b \le x$, то $x \in B$
        \item $A$ не содержит наибольшего.
    \end{enumerate}

    $\R$ --- множество всех возможных дедекиндовых сечений.
    \item Рациональные ($\Q$).
        $Q = \Z \times \N$ --- множество всех простых дробей.

        $\langle p,q \rangle$ --- то же, что $\frac{p}{q}$

        $\langle p_1,q_1 \rangle \equiv \langle p_2, q_2 \rangle$, если $p_1q_2 = p_2q_1$.

        $\Q = Q/_\equiv$
\end{enumerate}


\textbf{А что такое целые числа?}

<<Бог создал целые числа, всё остальное~--- дело рук человека.>>~--- Леопольд Кронеккер

\[ \Z: \dots -3, -2, -1, 0, 1, 2, 3, \dots. \]

Определим целые числа так:
\begin{itemize}
\item     $Z = \{\langle x, y \rangle\ |\ x,y\in \N_0\}$
\item Интуиция: $\langle x,y\rangle = x-y$
\item $$\begin{array}{rcl}
        \langle a, b \rangle + \langle c, d \rangle & = & \langle a + c, b + d \rangle \\
        \langle a, b \rangle - \langle c, d \rangle & = & \langle a + d, b + c \rangle
    \end{array}$$
\item     Пусть $\langle a, b \rangle \equiv \langle c,d\rangle$, если $a + d = b + c$. Тогда $\Z = Z/_\equiv$
\item      $0 = [\langle 0,0 \rangle],\; 1 = [\langle 1,0\rangle],\; -7 = [\langle 0,7 \rangle]$

\end{itemize}


\textit{А что такое натуральные числа?}

\[ \N: 1, 2, \dots \mbox{ или } \N_0: 0, 1, 2, \dots \]

\subsection{Акиоматика Пеано}
Определим натуральные числа так:
\begin{definition}
    $N$ (или, более точно, $\langle N, 0, (')\rangle$) \emph{соответствует} аксиоматике Пеано,
    если следующее определено/выполнено:
    \begin{enumerate}
        \item Операция <<штрих>> $('): N \to N$, причём нет $a,b \in N$, что $a \ne b$, но $a' = b'$.

            Если $x = y'$, то $x$ назовём следующим за $y$, а $y$ --- предшествующим $x$.
        \item Константа $0 \in N$: нет $x \in N$, что $x' = 0$.
        \item Индукция. Каково бы ни было свойство (<<предикат>>) $P: N \to V$, если:
            \begin{enumerate}
            \item $P(0)$
            \item При любом $x\in N$ из $P(x)$ следует $P(x')$
            \end{enumerate}
            то при любом $x \in N$ выполнено $P(x)$.
    \end{enumerate}
\end{definition}
Как построить? Например, в стиле алгебры Линденбаума:
\begin{enumerate}
\item $N$ --- язык, порождённый грамматикой $\nu ::= \texttt{0}\ |\ \nu \texttt{<<'>>}$
\item $0$ --- это $\texttt{<<0>>}$, $x'$ --- это $x \doubleplus \texttt{<<'>>}$
\end{enumerate}

\begin{example}
    Что не соответствует аксиомам Пеано?
\begin{enumerate}
\item $\Z$, где $x' = x^2$. Функция <<штрих>> не инъективна: $-3^2 = 3^2 = 9$.

\item Кольцо вычетов $\Z/{7\Z}$, где $x' = x+1$. $6' = 0$, что нарушает свойства $0$.

\item $\mathbb{R^+}\cup\{0\}$, где $x' = x+1$. Пусть $P(x)$ означает <<$x \in \Z$>>: \begin{enumerate}
\item $P(0)$ выполнено: $0 \in \Z$.
\item Если $P(x)$, то есть $x \in \Z$, то и $x+1 \in \Z$ --- так
что и $P(x')$ выполнено.
\end{enumerate}
Однако, $P(0.5)$ ложно.
\end{enumerate}
\end{example}

Докажем, например, что 0 единственный.
\begin{theorem}
    0 единственен: если $t$ таков, что при любом $y$
выполнено $y' \ne t$, то $t = 0$.
\end{theorem}
\begin{proof}

\begin{itemize}
    \item Определим $P(x)$ как <<либо $x = 0$, либо $x = y'$ для некоторого $y \in N$>>.
    \begin{enumerate}
        \item $P(0)$ выполнено, так как $0 = 0$.
        \item Если $P(x)$ выполнено, то возьмём $x$ в качестве $y$: тогда для $P(x')$ будет выполнено $x' = y'$.
    \end{enumerate}
Значит, $P(x)$ для любого $x \in N$.

\item Рассмотрим $P(t)$: <<либо $t = 0$, либо $t = y'$ для некоторого $y \in N$>>.
Но так как такого $y$ нет, то неизбежно $t = 0$.
\end{itemize}
\end{proof}

\begin{definition}
$1 = 0'$, $2 = 0''$, $3 = 0'''$, $4 = 0''''$, $5 = 0'''''$, $6 = 0''''''$,
$7 = 0'''''''$, $8 = 0''''''''$, $9 = 0'''''''''$
\end{definition}

\begin{definition}
$$a + b = \left\{ \begin{array}{ll} a, & \mbox{если } b = 0\\
                                    (a + c)', & \mbox{если } b = c'
                    \end{array}\right.$$
\end{definition}

Например, $$2 + 2 = 0'' + 0'' =  (0'' + 0')' =  ((0'' + 0)')' =  ((0'')')' = 0'''' = 4$$

\begin{definition}
$$a \cdot b = \left\{ \begin{array}{ll} 0, & \mbox{если } b = 0\\
                                    a \cdot c + a, & \mbox{если } b = c'
                    \end{array}\right.$$
\end{definition}


{Пример: коммутативность сложения (лемма 1)}

\begin{lemma}[1]
    $a + 0 = 0 + a$
\end{lemma}

\begin{proof} Пусть $P(x)$ --- это $x + 0 = 0 + x$.

    \begin{enumerate}
\item Покажем $P(0)$. $0 + 0 = 0 + 0$
\item Покажем, что если $P(x)$, то $P(x')$. Покажем $P(x')$, то есть $x' + 0 = \dots$
\begin{center}
    \begin{tabular}{lrl} %Равенство & обоснование \\\hline
                        $\dots = x'$ & $a=x',b=0$: & $x' + 0 \Rightarrow x'$ \\
                        $\dots = (x)'$ & \\
                        $\dots = (x + 0)'$ & $a=x,b=0$: &$(x + 0) \Leftarrow (x)$ \\
                        $\dots = (0 + x)'$ & $P(x)$: &$(x + 0) \Rightarrow (0 + x)$ \\
                        $\dots = 0 + x'$ & $a=0,b=x'$: &$0 + x' \Leftarrow (0 + x)'$
    \end{tabular}
\end{center}
\end{enumerate}
Значит, $P(a)$ выполнено для любого $a \in N$.
\end{proof}


\begin{lemma}[2]
$a + b' = a' + b$
\end{lemma}
\begin{proof} $P(x)$ --- это $a + x' = a' + x$
    \begin{enumerate}
\item $a + 0' = (a + 0)' = (a)' = a' = a' + 0$
\item Покажем, что $P(x')$ следует из $P(x)$: $a + x'' = (a + x')' = (a' + x)' = a' + x'$
    \end{enumerate}
\end{proof}

\begin{theorem}
    $a + b = b + a$
\end{theorem}
\begin{proof}[Доказательство индукцией по $b$: $P(x)$ --- это $a + x = x + a$]
\begin{enumerate}
\item $a + 0 = 0 + a$ (лемма 1)
\item $a + x' = (a + x)' = (x + a)' = x + a' = x' + a$
\end{enumerate}
\end{proof}

\subsubsection{Уточнение исчисления предикатов}
\begin{itemize}
\item Пусть требуется доказывать утверждения про равенство. Введём $E(p,q)$ --- предикат <<равенство>>.
\item Однако, $\not\vdash E(p,q)\to E(q,p)$: если $D = \{0,1\}$ и $E(p,q) ::= (p>q)$,
то $\not\models E(p,q)\to E(q,p)$.
\item Конечно, можем указывать $\forall p.\forall q.E(p,q)\to E(q,p) \vdash \varphi$.
\item Но лучше добавим аксиому $\forall p.\forall q.E(p,q)\to E(q,p)$.
\item Добавив необходимые аксиомы, получим \emph{теорию первого порядка}.
\end{itemize}

\begin{definition}
Теорией первого порядка назовём исчисление предикатов с дополнительными (<<нелогическими>>
или <<математическими>>):
\begin{itemize}
\item предикатными и функциональными символами;
\item аксиомами.
\end{itemize}

Сущности, взятые из исходного исчисления предикатов, назовём \emph{логическими}
\end{definition}

\begin{tabular}{llll}
Порядок & Кванторы & Формализует суждения о... & Пример\\\hline
нулевой & запрещены & об отдельных значениях & И.В.\\
первый & по предметным переменным & о множествах & И.П.\\
    &   & $S = \{ t\ |\ \psi[x := t] \}$ \\
второй & по предикатным переменным & о множествах множеств \\
    &   & $S = \{ \{t\ |\ P(t)\}\ |\ \varphi[p := P] \}$ \\
    & ...
\end{tabular}


\subsubsection{Формальная арифметика}
\begin{definition}
Формальная арифметика --- теория первого порядка, со следующими добавленными нелогическими ...
\begin{itemize}
\item двуместными функциональными символами $(+)$, $(\cdot)$; одноместным функциональным символом $(')$,
нульместным фукнциональным символом $0$;
\item двуместным предикатным символом $(=)$;
\item восемью нелогическими \emph{аксиомами}:\vspace{0.1cm}
\begin{tabular}{ll}
(A1) $a=b \to a=c \to b=c$             &(A5) $a+0 = a$                     \\
(A2) $a=b \to a'=b'$                   &(A6) $a+b' = (a+b)'$               \\
(A3) $a'=b' \to a=b$                   &(A7) $a\cdot 0 = 0$                \\
(A4) $\neg a' = 0$                     &(A8) $a\cdot b' = a \cdot b + a$
\end{tabular}
\item нелогической схемой аксиом индукции $\psi[x:=0]\&(\forall x.\psi\to \psi[x:=x'])\to \psi$, с метапеременными $x$ и $\psi$.
\end{itemize}
\end{definition}


\begin{statement}
    $a=a$ в формальной арифметике.
\end{statement}
\begin{proof}
    Пусть $\top ::= 0=0\to 0=0 \to 0=0$, тогда:

\begin{tabular}{lll}
(1) & $a=b\to a=c \to b=c$ & (Акс. А1)\\
(2) & $(a=b\to a=c \to b=c) \to \top \to (a=b\to a=c \to b=c)$ & (Сх. акс. 1)\\
(3) & $\top \to (a=b\to a=c \to b=c)$ & (M.P. 1, 2)\\
(4) & $\top \to (\forall c.a = b\to a = c \to b = c)$ & (Введ. $\forall$)\\
(5) & $\top \to (\forall b.\forall c.a = b\to a = c \to b = c)$ & (Введ. $\forall$)\\
(6) & $\top \to (\forall a.\forall b.\forall c.a = b\to a = c \to b = c)$ & (Введ. $\forall$)\\
(7) & $\top$ & (Сх. акс 1)\\
(8) & $(\forall a.\forall b.\forall c.a = b\to a = c \to b = c)$ & (M.P. 7, 6)\\
(9) & $(\forall a.\forall b.\forall c.a = b\to a = c \to b = c) \to $\\
    & $\to (\forall b.\forall c.a+0 = b\to a+0 = c \to b = c)$ & (Сх. акс. 11)\\
(10) & $\forall b.\forall c.a+0 = b\to a+0 = c \to b = c$ & (M.P. 8, 9)\\
(12) & $\forall c.a+0 = a\to a+0 = c \to a = c$ & (M.P. 10, 11)\\
(14) & $a+0 = a\to a+0 = a \to a = a$ & (M.P. 12, 13)\\
(15) & $a+0 = a$ & (Акс. А5)\\
(16) & $a+0 = a \to a = a$ & (M.P. 15, 14)\\
(17) & $a = a$ & (M.P. 15, 16)
\end{tabular}
\end{proof}
