\subsection{Теоремы Гёделя о неполноте арифметики}


\begin{theorem}Первая теорема Гёделя о неполноте арифметики
    \begin{itemize}
    \item Если формальная арифметика непротиворечива, то $\not\vdash\sigma(\overline{\ulcorner\sigma\urcorner})$.
    \item Если формальная арифметика $\omega$-непротиворечива, то $\not\vdash\neg\sigma(\overline{\ulcorner\sigma\urcorner})$.
    \end{itemize}
\end{theorem}


\begin{note}
    $\sigma(x_1) := \forall p.\neg\omega_1(x_1,p)$. $W_1(\ulcorner\xi\urcorner,p)$ --- $p$ есть доказательство самоприменения $\xi$.
\end{note}

\begin{proof}[Доказательство теоремы Гёделя].
\begin{itemize}
\item Пусть $\vdash\sigma(\overline{\ulcorner\sigma\urcorner})$. Значит, $p$ --- номер доказательства.  Тогда
$\langle\ulcorner\sigma\urcorner,p\rangle \in W_1$.  Тогда $\vdash\omega_1(\overline{\ulcorner\sigma\urcorner},\overline{p})$.
Тогда $\vdash\exists p.\omega_1(\overline{\ulcorner\sigma\urcorner},p)$.  То есть
$\vdash\neg\forall p.\neg\omega_1(\overline{\ulcorner\sigma\urcorner},p)$.  То есть $\vdash\neg\sigma(\overline{\ulcorner\sigma\urcorner})$. Противоречие.

\item Пусть $\vdash\neg\sigma(\overline{\ulcorner\sigma\urcorner})$.  То есть $\vdash\exists p.\omega_1(\overline{\ulcorner\sigma\urcorner},p)$.
\begin{itemize}
\item
 Но найдётся ли натуральное число $p$, что $\vdash\omega_1(\overline{\ulcorner\sigma\urcorner},\overline{p})$?

\item Пусть нет. То есть $\vdash\neg\omega_1(\overline{\ulcorner\sigma\urcorner},\overline{0})$, $\vdash\neg\omega_1(\overline{\ulcorner\sigma\urcorner},\overline{1})$, \dots

\item По $\omega$-непротиворечивости $\not\vdash\exists p.\neg\neg\omega_1(\overline{\ulcorner\sigma\urcorner},p)$.
\end{itemize}
       Значит, найдётся натуральное $p$, что $\vdash\omega_1(\overline{\ulcorner\sigma\urcorner},\overline{p})$.
То есть, $\langle\ulcorner\sigma\urcorner, p\rangle\in W_1$.
То есть, $p$ --- доказательство самоприменения $W_1$: $\vdash\sigma(\overline{\ulcorner\sigma\urcorner})$.  Противоречие.
\end{itemize}
\end{proof}


\begin{theorem}
    Формальная арифметика с классической моделью~--- неполна.
\end{theorem}
\begin{proof}
Полная теория --- теория, в которой любая общезначимая формула доказуема.

Рассмотрим Ф.А. с классической моделью.
Из теоремы Гёделя имеем $\not\vdash\sigma(\overline{\ulcorner\sigma\urcorner})$.

Рассмотрим $\sigma(\overline{\ulcorner\sigma\urcorner}) \equiv \forall p.\neg\omega_1(\overline{\ulcorner\sigma\urcorner}),p$:
нет числа $p$, что $p$ --- номер доказательства $\sigma(\overline{\ulcorner\sigma\urcorner})$.

То есть, $\llbracket \forall p.\neg\omega_1(\overline{\ulcorner\sigma\urcorner}),p) \rrbracket = \text{И}$.
То есть, $\models \sigma(\overline{\ulcorner\sigma\urcorner})$.
\end{proof}

Почему мы должны требовать от нашей теории $w$--непротиворечивости? Неужели наша формальная ариметика недостаточно содержательна в том, чтобы сформулировать, какие  натуральные числа можно использовать? Этот вопрос занимал математиков и вскоре было предложено решение

\paragraph{Первая теорема Гёделя о неполноте в форме Россера}
\begin{definition}
    $\theta_1\le\theta_2 \equiv \exists p.p+\theta_1=\theta_2\quad\quad\theta_1<\theta_2\equiv\theta_1\le\theta_2\&\neg\theta_1=\theta_2$.
\end{definition}
\begin{definition}
    Пусть $\langle \ulcorner\xi\urcorner,p\rangle \in W_2$, если $\vdash\neg\xi(\overline{\ulcorner\xi\urcorner})$.
Пусть $\omega_2$ выражает $W_2$ в формальной арифметике.
\end{definition}
\begin{theorem}
    Рассмотрим $\rho(x_1) = \forall p.\omega_1(x_1,p)\rightarrow\exists q.q \le p \& \omega_2 (x_1, q)$.

    Тогда $\not\vdash\rho(\overline{\ulcorner\rho\urcorner})$ и $\not\vdash\neg\rho(\overline{\ulcorner\rho\urcorner})$.
\end{theorem}
\begin{note}
    Смысл $\rho(\overline{\ulcorner\rho\urcorner})$ примерно такой: <<Меня легче опровергнуть, чем доказать>>.
\end{note}

\textbf{А есть ли более формальное доказательство?}

Неполнота варианта теории, изложенной выше, формально доказана на Coq, Russell O'Connor, 2005:\\
``My proof, excluding standard libraries and the library for Pocklington’s criterion,
consists of 46 source files, 7 036 lines of specifications, 37 906 lines of proof,
and 1 267 747 total characters. The size of the gzipped tarball (gzip -9) of all
the source files is 146 008 bytes, which is an estimate of the information content of my proof.''

Утрмерждение теоремы, записанное на языке Coq.
\begin{verbatim}
    Theorem Incompleteness : forall T : System,
        Included Formula NN T ->
        RepresentsInSelf T ->
        DecidableSet Formula T ->
        exists f : Formula,
        Sentence f/\(SysPrf T f \/ SysPrf T (notH f) -> Inconsistent LNN T).
\end{verbatim}

\paragraph{А что мы можем сказать про противоречивать Ф.А.?}

\begin{definition}
Обозначим за $\psi(x,p)$ формулу, выражающую в формальной арифметике рекурсивное
отношение Proof: $\langle \ulcorner\xi\urcorner,p\rangle \in \text{Proof}$, если $p$ --- гёделев номер
доказательства $\xi$.

Обозначим $\pi(x)\equiv\exists p.\psi(x,p)$.
\end{definition}
\begin{definition}
    Формулой Consis назовём формулу $\neg \pi(\overline{\ulcorner 1=0 \urcorner})$.
\end{definition}
\begin{note}
    Неформальный смысл Consis: <<формальная арифметика непротиворечива>>.
\end{note}

\begin{theorem}
    [Вторая теорема Гёделя о неполноте арифметики]
    Если Consis доказуем, то формальная арифметика противоречива.
\end{theorem}
\begin{proof}[Неформальное доказательство]

    Формулировка 1 теоремы Гёделя о неполноте арифметики:
<<если Ф.А. непротиворечива, то недоказуемо $\sigma(\overline{\ulcorner\sigma\urcorner})$>>.

То есть, $\forall p.\neg\omega_1(\overline{\ulcorner\sigma\urcorner},p)$.

То есть,
если $\text{Consis}$, то $\sigma(\overline{\ulcorner\sigma\urcorner})$.

То есть, если $\text{Consis}$, то $\sigma(\overline{\ulcorner\sigma\urcorner})$, --- и это можно доказать,
то есть $\vdash\text{Consis}\rightarrow\sigma(\overline{\ulcorner\sigma\urcorner})$.

Однако, если формальная арифметика непротиворечива, то $\not\vdash\sigma(\overline{\ulcorner\sigma\urcorner})$.
\end{proof}

\begin{definition}
Будем говорить, что формула $\psi$, выражающая отношение Proof,
формула $\pi$ и формула Consis соответствуют
условиям Гильберта-Бернайса-Лёфа, если следующие условия выполнены для любой формулы $\alpha$:

\begin{enumerate}
\item $\vdash \alpha$ влечет $\vdash \pi(\overline{\ulcorner\alpha\urcorner})$
\item $\vdash \pi (\overline{\ulcorner\alpha\urcorner}) \rightarrow \pi(\overline{\ulcorner\pi(\overline{\ulcorner\alpha\urcorner})\urcorner})$
\item $\vdash \pi (\overline{\ulcorner\alpha\rightarrow \beta\urcorner}) \rightarrow \pi(\overline{\ulcorner\alpha\urcorner}) \rightarrow \pi(\overline{\ulcorner\beta\urcorner})$
\end{enumerate}
\end{definition}

\begin{lemma}
    Лемма об автоссылках. Для любой формулы $\phi(x_1)$ можно построить такую замкнутую формулу $\alpha$ (не использующую неаксиоматических предикатных и функциональных символов), что $\vdash \phi(\overline{\ulcorner\alpha\urcorner}) \leftrightarrow \alpha$.
\end{lemma}

\begin{note}
    $\leftrightarrow$ означает, что это можно доказать и слева направо, и справо налево.
\end{note}

\begin{theorem}
    Существует такая замкнутая формула $\gamma$, что если Ф.А. непротиворечива, то
$\not\vdash \gamma$, а если Ф.А. $\omega$-непротиворечива, то и $\not\vdash\neg\gamma$.
\end{theorem}
\begin{proof}
Рассмотрим $\phi(x_1) \equiv \neg\pi(x_1)$. Тогда по лемме об автоссылках существует
$\gamma$, что $\vdash \gamma \leftrightarrow \neg\pi(\overline{\ulcorner\gamma\urcorner})$.

\begin{itemize}
\item Предположим, что $\vdash \gamma$. Тогда $\vdash \gamma \rightarrow \neg\pi(\overline{\ulcorner\gamma\urcorner})$, то есть $\not\vdash\gamma$
\item Предположим, что $\vdash \neg\gamma$. Тогда $\vdash \pi(\overline{\ulcorner\gamma\urcorner})$,
то есть $\vdash \exists p.\psi(\overline{\ulcorner\gamma\urcorner},p)$. Тогда по $\omega$-непротиворечивости
найдётся $p$, что $\vdash \psi(\overline{\ulcorner\gamma\urcorner},\overline{p})$, то есть $\vdash \gamma$.
\end{itemize}
\end{proof}

\begin{proof}
[Доказательство второй теоремы Гёделя]

\begin{enumerate}
\item Пусть $\gamma$ таково, что $\vdash \gamma \leftrightarrow \neg\pi(\overline{\ulcorner\gamma\urcorner})$.
\item Покажем $\pi(\overline{\ulcorner\gamma\urcorner})\vdash \pi(\overline{\ulcorner 1=0\urcorner})$.

\begin{enumerate}
\item По условию 2, $\vdash \pi(\overline{\ulcorner\gamma\urcorner}) \rightarrow \pi(\overline{\ulcorner\pi(\overline{\ulcorner\gamma\urcorner})\urcorner})$.
По теореме о дедукции $\pi(\overline{\ulcorner\gamma\urcorner})\vdash \pi(\overline{\ulcorner\pi(\overline{\ulcorner\gamma\urcorner})\urcorner})$;

\item Так как $\vdash \pi(\overline{\ulcorner\gamma\urcorner})\rightarrow\neg\gamma$, то
по условию 1 $\vdash \pi(\overline{\ulcorner\pi(\overline{\ulcorner\gamma\urcorner})\rightarrow\neg\gamma\urcorner})$;

\item По условию 3, $\pi(\overline{\ulcorner\gamma\urcorner})\vdash \pi(\overline{\ulcorner\pi(\overline{\ulcorner\gamma\urcorner})\urcorner})
\rightarrow \pi(\overline{\ulcorner\pi(\overline{\ulcorner\gamma\urcorner})\rightarrow\neg\gamma\urcorner})\rightarrow
\pi(\overline{\ulcorner\neg\gamma\urcorner})$;

\item Таким образом, $\pi(\overline{\ulcorner\gamma\urcorner})\vdash\pi(\overline{\ulcorner\neg\gamma\urcorner})$;

\item Однако, $\vdash \gamma\rightarrow\neg\gamma\rightarrow 1=0$. Условие 3 (применить два раза) даст $\pi(\overline{\ulcorner\gamma\urcorner})\vdash \pi(\overline{\ulcorner 1=0 \urcorner})$.
\end{enumerate}

\item $\neg\pi(\overline{\ulcorner 1=0 \urcorner})\rightarrow\neg\pi(\overline{\ulcorner\gamma\urcorner})$ (т. о дедукции, контрапозиция).
\item $\vdash \neg\pi(\overline{\ulcorner 1=0 \urcorner})\rightarrow\gamma$ (определение $\gamma$).
\end{enumerate}
\end{proof}
